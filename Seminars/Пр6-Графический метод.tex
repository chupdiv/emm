%encoding=utf8%
\documentclass[a5paper,11pt]{extarticle}

% Русская кодировка
\usepackage[T2A]{fontenc}
\usepackage[utf8]{inputenc}
\usepackage[russian]{babel}

\usepackage[pdftex]{graphicx}
\usepackage[pdftex,colorlinks,urlcolor=black, linkcolor=black, citecolor=black]{hyperref}
%        \pdfcompresslevel=9 % сжимать PDF

% необходимые модули
\usepackage{amssymb,amsmath,amsthm}
\usepackage{tabularx,multirow,hhline}
\usepackage{indentfirst}
\usepackage{tikz}
\usepackage[margin=5mm]{geometry}
\usepackage{enumitem}
\usepackage{multicol}
\setlist{noitemsep,leftmargin=\parindent}

\AddEnumerateCounter{\Asbuk}{\@Asbuk}{\CYRM}
\AddEnumerateCounter{\asbuk}{\@asbuk}{\cyrm}
\newlist{dotenumerate}{enumerate}{10}
\setlist[enumerate,1]{label=\arabic*., ref=\arabic*} %списки со скобками
\setlist[enumerate,2]{label=\asbuk*), ref=(\asbuk*)} %списки со скобками

\usepackage{environ}

\NewEnviron{LOOP}[1]{
  \newcounter{nclone}
  \setcounter{nclone}{#1} 
  \par
  \loop
    \BODY
  \addtocounter{nclone}{-1}
  \ifnum \value{nclone}>0 \repeat}

%Полуторный интервал
\renewcommand{\baselinestretch}{1.00}
\pagestyle{empty}
\newcolumntype{C}{>{\centering\arraybackslash}X}


\sloppy
\binoppenalty=10000
\relpenalty=10000
\begin{document}
\begin{enumerate}
  {\item[]\centering {\small Практическое занятие \textnumero~6 \par}
          \bfseries Линейное программирование. Методы решения ЗЛП
          \par\vspace{1mm}
  }

%\item Решить систему методом Гаусса---Жордана. Найти все базисные решения
%
%\textbf{a)}~$
%\left\lbrace
%\begin{aligned}
%  x_1+2x_2-x_3&=5,\\
%  2x_1-x_2-3x_3&=-4;\\
%\end{aligned}
%\right.
%$
%\textbf{б)}~$
%\left\lbrace
%\begin{aligned}
%  x_1+x_2+x_3+x_4&=2,\\
%  2x_1+2x_2-x_3+2x_4&=-2,\\
%  x_1-x_2-x_4&=2.\\
%\end{aligned}
%\right.
%$

\item Совхозу требуется не более 10~трехтонных автомашин и не более 8 пятитонных.
Отпускная цена автомашины первой марки 2\,000~ден.\;ед., второй марки 4\,000~ден.\;ед.
Совхоз может выделить для приобретения машин 40\,000~ден.\;ед. Сколько следует
приобрести автомашин каждой марки в отдельности, чтобы их суммарная
грузоподъёмность была максимальной.

\item Решите задачи графически ($x_{i} \geqslant 0$)

\textbf{a)}~% Неограниченная область
$F=3x_{1}+2x_{2} \to \max$,\quad 
$\left\lbrace
  \begin{aligned}
    x_{1}+x_{2} &\geqslant 8\\
    2x_{1}-x_{2} &\leqslant 1\\
    x_{1}-2x_2 &\leqslant 2\\
  \end{aligned} 
  \right.$ 

\textbf{б)}~%отрезок
	$F=100-x-2y \to \min$, \quad
	$ \left\lbrace\begin{aligned}
		x - 4y &\leqslant 4;\\
		3x - y &\geqslant 0;\\
		x + 2y  &\geqslant 4\\
	  \end{aligned} \right.
	$

\textbf{в)}~%
$F=2x_{1}-x_{2} \to \min$,\quad
$
\left\lbrace
\begin{aligned}
  x_1-x_2+4x_3-2x_4&=2,\\
  3x_1+2x_2-x_3+4x_4&=3.\\
\end{aligned}
\right.
$

%\item
%Фирма выпускает ковбойские шляпы двух фасонов ($A$ и $B$). Трудоемкость изготовления шляпы фасона $A$ вдвое выше трудоемкости изготовления шляпы фасона~$B$. Если бы фирма выпускала только шляпы фасона~$A$, суточный объем производства мог бы составить 500 шляп. 
%Суточный объем сбыта шляп обоих фасонов ограничен диапазоном от 150 до 200 штук. Прибыль от продажи шляпы фасона~$A$ равна~\$\,8, а фасона $B$~--- \$\,5. Определить, какое количество шляп каждого фасона следует изготовить, чтобы максимизировать прибыль.



  \item 
  	Решить симплекс-методом:\\
  \textbf{а)}~$F  = -3x_1 + x_2 - x_3 - 2x_4  \to \max$\quad $
  \left\lbrace\begin{aligned}
    2x_1 + x_2 + 3x_3 + &x_4 \leqslant 5,\\
    2x_1 - x_2 - x_3 + &2x_4 \leqslant 1;
  \end{aligned}\right.
  $ 
  
  \textbf{б)}~$F = -x_1 +x_2 + 3x_3 +x_4  \to \min$,\quad $
  \left\lbrace\begin{aligned}
  -x_1 + x_2 - x_3 + 2x_4 &\leqslant 6,\\
  2x_1 + x_2 + 2x_3 - x_4 &\leqslant 4.
  \end{aligned}\right.
  $
\item
	 Краска для внутренних~($I$) и наружных~($E$) работ поступает в оптовую продажу. Для производства красок используются два исходных продукта~--- $A$ и~$B$. Суточные запасы которых~--- 6 и~8\;тонн соответственно. 
На тонну краски~$E$ идет 1\;т.~компонента~$A$ и~2\;т. компонента~$B$.
На тонну краски~$I$ идет 2\;т.~компонента~$A$ и~1\;т. компонента~$B$.
извесно, что суточный спрос на краску I не превышает спроса на Е более чем на 1~т., и не превышает 2 т. в сутки. 
Оптовые цены красок $E$ и $I$  равны 3000~ден.~ед. и 2000~ден.~ед соответственно. Какое количество краски каждого вида должна производить фабрика, чтобы доход от реализации продукции был максимальным?
\medskip
{\item[] \centering \bfseries Домашнее задание\par}
По пособию Ряттель А.\,В. Основы экономико-математического моделирования. Тема 5. Стр. 17--22 выполнить: 
%\item Решить методом Гаусса---Жордана: задание 2. стр 17.
\item Упражнения 1 и 3. стр. 18--19
\item Задания 1 и 3. стр 20--21

%\item Решите задачи графически ($x_{1} \geqslant 0,  x_{2} \geqslant 0$)\\
%\textbf{a)}~$ %4.4
%\begin{aligned}
%   &F=-2x_{1}+6x_{2} \to \min;\\ 
%%  &x_{1} \geqslant 0,  x_{2} \geqslant 0;\\
%  &\left\lbrace\begin{aligned}
%    x_{1}+ x_{2} &\geqslant 2;\\
%    -x_{1}+ 2x_{2} &\leqslant 4;\\
%    x_{1}+2x_{2} &\leqslant 8;
%  \end{aligned} \right.
%\end{aligned}$ 
%\textbf{б)}~$
%\begin{aligned} %4.6
%   &F=x_{1}+x_{2} \to \max;\\ 
%%  &x_{1} \geqslant 0,  x_{2} \geqslant 0;\\
%  &\left\lbrace\begin{aligned}
%    x_{1}- 4x_{2}-4 &\leqslant 0;\\
%    3x_{1}- x_{2} &\geqslant 0;\\
%    x_{1}+x_{2}-4 &\geqslant 0;
%  \end{aligned} \right.
%\end{aligned}$
%\textbf{в)}~$%4.8
%\begin{aligned}
%   &F=x_{1}-x_{2} \to \max;\\ 
%%  &x_{1} \geqslant 0,  x_{2} \geqslant 0;\\
%  &\left\lbrace\begin{aligned}
%    -2x_{1}+ x_{2} &\leqslant 2;\\
%    x_{1}- 2x_{2} &\leqslant -8;\\
%    x_{1}+x_{2} &\leqslant 5;
%  \end{aligned} \right.
%\end{aligned}$




%$%4.7
%\begin{aligned}
%   &F=2x_{1}-x_{2} \to \min;\\ 
%  &x_{1} \geqslant 0,  x_{2} \geqslant 0;\\
%  &\left\lbrace\begin{aligned}
%    x_{1}+ x_{2} &\geqslant 4;\\
%    2x_{1}- x_{2} &\geqslant 2;\\
%    x_{1}+2x_{2} &\leqslant 10;
%  \end{aligned} \right.
%\end{aligned}$\hfil


\end{enumerate}
\newpage

\end{document}	
