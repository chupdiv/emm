%encoding=utf8%
\documentclass[a4paper,12pt]{extarticle}
% \usepackage{pgfpages}

% Русская кодировка
\usepackage[T2A]{fontenc}
\usepackage[utf8]{inputenc}
\usepackage[russian]{babel}
%\usepackage[pdftex]{graphicx}
%\usepackage[pdftex,colorlinks,urlcolor=black, linkcolor=black, citecolor=black]{hyperref}
%        \pdfcompresslevel=9 % сжимать PDF
% необходимые модули
\usepackage{amssymb,amsmath,amsthm}
\usepackage{tabularx,multirow,hhline}
\usepackage{indentfirst}
\usepackage{tikz}
\usepackage{icomma}
\usepackage[margin=8mm]{geometry}
\usepackage[inline]{enumitem}
\usepackage{multicol}
\setlist{nosep,leftmargin=\parindent}

\AddEnumerateCounter{\Asbuk}{\@Asbuk}{\CYRM}
\AddEnumerateCounter{\asbuk}{\@asbuk}{\cyrm}
\newlist{dotenumerate}{enumerate}{10}
\setlist[enumerate,1]{label=\arabic*., ref=\arabic*} %списки со скобками
\setlist[enumerate,2]{label=\asbuk*), ref=(\asbuk*)} %списки со скобками

\usepackage{environ}

\NewEnviron{LOOP}[1]{
  \newcounter{nclone}
  \setcounter{nclone}{#1} 
  \par
  \loop
    \BODY
  \addtocounter{nclone}{-1}
  \ifnum \value{nclone}>0 \repeat}

%Полуторный интервал
\renewcommand{\baselinestretch}{1.00}
\pagestyle{empty}
\newcolumntype{C}{>{\centering\arraybackslash}X}


\sloppy
\binoppenalty=10000
\relpenalty=10000

\begin{document}
{\centering {\footnotesize Практическое занятие \textnumero~31 \par}
\bfseries   Эконометрические модели. Множественная регрессия
\par\vspace{1mm}
}

{\small
\begin{itemize}


\item Уравнение регрессии $Y=XB + \varepsilon$, где $B = (X^TX)^{-1}X^TY$, 
\item Оценка влияние факторов $x_j$ на зависимую переменную $y$:\quad $\beta_j = b_j \frac{s_{x_j}}{s_y}$,\quad $E_j=b_j \frac{\bar{x}_j}{\bar{y}}$
\item Остаточная дисперсия $s^2 = \frac{(Y-XB)^T(Y-XB)}{n-p-1}$
\item Коэффициенты детерминации $R^2 = 1-\frac{(Y-XB)^T(Y-XB}{(Y-\bar{Y})^T(Y-\bar{Y}}$,\qquad  $\hat{R}^2 = 1-\frac{s^2}{\hat{s}_y^2}$
\item Значимость коэффициента регрессии: $\frac{|b_j|}{s_{b_j}} > t_{1-\alpha;n-p-1}$, где $s^2_{b_j}=s^2[(X^TX)^{-1}]_{jj}$

\item Доверительный интервал для коэффициента регрессии: $|b^*_j-b_j| \leqslant t_{1-\alpha;n-p-1}s_{b_j}$, где $s^2_{b_j}=s^2[(X^TX)^{-1}]_{jj}$
\item Доверительный интервал для значения 
$y$: $|y^*_0-y_0| \leqslant t_{1-\alpha;n-p-1} s_{y_0}$, где $s^2_{y_0}=s^2[1+X_0^T(X^TX)^{-1}X)_{jj}$

\end{itemize}
}
\medskip
\hrule
\medskip 

\begin{enumerate}
    % \item Постройте регрессионную модель зависимости расходов на питание семьи от  общих расходов и размера семьи.

    % {\centering \begin{tabular}{cccc}
    % \hline
    %     № п.\,п.  & Расходы на питание ($y$) & Общие расходы ($x_1$) & Размер семьи ($x_2$) \\
    % \hline
    %  1 & 22  & 45  & 1,5 \\
    %  2 & 34  & 75  & 1,6 \\
    %  3 & 50  & 125 & 1,9 \\
    %  4 & 67  & 223 & 1,8 \\
    %  5 & 47  & 92  & 3,4 \\
    %  6 & 66  & 146 & 3,6 \\
    %  7 & 81  & 227 & 3,4 \\
    %  8 & 106 & 358 & 3,5 \\
    %  9 & 70  & 135 & 5,5 \\
    % 10 & 95  & 218 & 5,4 \\
    % 11 & 119 & 331 & 5,4 \\ 
    % 12 & 147 & 490 & 5,3 \\
    % 13 & 93  & 175 & 8,5 \\
    % 14 & 133 & 305 & 8,3 \\
    % 15 & 169 & 468 & 8,1 \\
    % 16 & 197 & 749 & 7,3 \\
    % \hline
    % \end{tabular}  \par}
  
    \item 
    Предположим, что коммерческий агент рассматривает возможность закупки небольших зданий под офисы в традиционном деловом районе. Агент может использовать множественный регрессионный анализ для оценки цены здания под офис на основе следующих переменных:
    \begin{enumerate*}
        \item[$y$~---] оценочная стоимость здания в условных единицах;
        \item[$x_1$~---] общая площадь в квадратных метрах;
        \item[$x_2$~---] количество офисов;
        \item[$x_3$~---] количество входов  ($0,5$ входа --- вход только для доставки корреспонденции);
        \item[$x_4$~---] возраст здания~--- время эксплуатации в годах.
    \end{enumerate*}
    Агент наугад выбирает 11 зданий из имеющихся 1500 и получает следующие данные:
    \par{\centering \begin{tabular}{cccccc}
        \hline
        №     & Площадь & Офисы & Входы & Возраст & Стоимость \\      
        п.\,п.& $x_1$& $x_2$ & $x_3$ & $x_4$ & $y$    \\      
        \hline
        1     & 2310 & 2     & 2     & 20    & 42000  \\
        2     & 2333 & 2     & 2     & 12    & 144000 \\
        3     & 2356 & 3     & 1,5   & 33    & 151000 \\
        4     & 2379 & 3     & 2     & 43    & 151000 \\
        5     & 2402 & 2     & 3     & 53    & 139000 \\
        6     & 2425 & 4     & 3     & 23    & 169000 \\
        7     & 2448 & 2     & 1,5   & 99    & 126000 \\
        8     & 2471 & 2     & 2     & 34    & 142000 \\
        9     & 2494 & 3     & 3     & 23    & 163000 \\
        10    & 2517 & 4     & 4     & 55    & 169000 \\
        11    & 2540 & 2     & 3     & 22    & 149000 \\
        \hline
    \end{tabular}\par}
    
    Требуется:
    \begin{enumerate}
        \item вычислить матрицу коэффициентов парной корреляции и проанализировать тесноту связи между показателями;
        \item построить линейную модель регрессии, описывающие зависимость $y_t$ от факторов $x_{1}$ и $x_{2}$;
        \item оценить качество модели вычислив коэффициент детерминации;
        \item проанализировать влияние факторов на зависимую переменную ($\beta$-коэффициент, коэффициентр эластичности) и оценить их значимость,  найти доверительный интервал;
        \item определить точечные прогнозные оценки ВНП для 5 наблюдений (объясняющие переменные задать самостоятельно);
    \end{enumerate}
    

\item Заданы три временных ряда: 

\par{\centering\begin{tabular}{lcccccccccc}
    \hline    
    Год &  1 &  2 &  3 &  4 &  5 &  6 &  7 &  8 &  9 & 10 \\
    \hline    
    Валовой национальный продукт (в млрд \$), $y_t$      &  3 & 11 & 10 & 11 & 15 & 17 & 21 & 25 & 23 & 19 \\
    Потребление (в млрд \$), $x_{1t}$   & 18 & 14 & 33 & 37 & 40 & 42 & 41 & 49 & 56 & 48 \\
    Инвестиции (в млрд \$), $x_{2t}$   & 20 & 22 & 14 & 26 & 25 & 32 & 35 & 34 & 39 & 45 \\
    \hline    
\end{tabular}\par}
Требуется:
\begin{enumerate}
    \item вычислить матрицу коэффициентов парной корреляции и проанализировать тесноту связи между показателями;
    \item построить линейную модель регрессии, описывающие зависимость $y_t$ от факторов $x_{1t}$ и $x_{2t}$;
    \item оценить качество модели, вычислив среднюю ошибку аппроксимации и коэффициент детерминации;
    \item проанализировать влияние факторов на зависимую переменную и оценить их значимость,
    \item определить точечные прогнозные оценки ВНП для 5 наблюдений (объясняющие переменные задать самостоятельно);
\end{enumerate}
Все полученные результаты необходимо интерпретировать.

\end{enumerate}
    
    

    
    
    
    
    
    
    
    
    
    
    
    
    
    
    

\end{document}	
