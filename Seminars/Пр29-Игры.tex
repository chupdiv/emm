%encoding=utf8%
\documentclass[a4paper,12pt]{extarticle}
% \usepackage{pgfpages}

% Русская кодировка
\usepackage[T2A]{fontenc}
\usepackage[utf8]{inputenc}
\usepackage[russian]{babel}
%\usepackage[pdftex]{graphicx}
%\usepackage[pdftex,colorlinks,urlcolor=black, linkcolor=black, citecolor=black]{hyperref}
%        \pdfcompresslevel=9 % сжимать PDF
% необходимые модули
\usepackage{amssymb,amsmath,amsthm}
\usepackage{tabularx,multirow,hhline}
\usepackage{indentfirst}
\usepackage{tikz}
\usepackage[margin=8mm]{geometry}
\usepackage[inline]{enumitem}
\usepackage{multicol}
\setlist{nosep,leftmargin=\parindent}

\AddEnumerateCounter{\Asbuk}{\@Asbuk}{\CYRM}
\AddEnumerateCounter{\asbuk}{\@asbuk}{\cyrm}
\newlist{dotenumerate}{enumerate}{10}
\setlist[enumerate,1]{label=\arabic*., ref=\arabic*} %списки со скобками
\setlist[enumerate,2]{label=\asbuk*), ref=(\asbuk*)} %списки со скобками

\usepackage{environ}

\NewEnviron{LOOP}[1]{
  \newcounter{nclone}
  \setcounter{nclone}{#1} 
  \par
  \loop
    \BODY
  \addtocounter{nclone}{-1}
  \ifnum \value{nclone}>0 \repeat}

%Полуторный интервал
\renewcommand{\baselinestretch}{1.00}
\pagestyle{empty}
\newcolumntype{C}{>{\centering\arraybackslash}X}


\sloppy
\binoppenalty=10000
\relpenalty=10000
\begin{document}
\begin{enumerate}
  {\item[]\centering {\footnotesize Практическое занятие \textnumero~29 \par}
          \bfseries Разные виды игр + Управление организационными системами
          \par\vspace{1mm}
  }

  \item 
  На столе лежат 6 спичек. Игроки берут спички по очереди, каждый может взять 1 или 2 спички. Тот, кто берет последнюю спичку, проирывает 1 очко. 
  Построить граф игры. Определить оптимальные стратегии игроков.

\item 
  Найти хотя бы одно решение бесконечной антагонистической игры на единичном квадрате (\(x,y\in [0;1]\)) со следующей функцией выигрыша
  \[
    H(x,y) = \begin{cases}
      x   & x\leqslant \frac{y}{2}\\
      y-x & \frac{y}{2} < x \leqslant y\\
      x-y &  y < X\\
    \end{cases}
  \]
    
  \item 
  Пусть имеется шесть потребителей, приоритеты которых определяются числами 7, 8, 12, 5, 9, 11. 
  Ресурс Центра составляет 67. 
  Определить равновесные стратегии (заявки) Потребителей, если ресурс распределяется в соответствии с механизмом обратных приоритетов.
  

  \item 
  Пять экспертов сообщили следующие оценки из промежутка $[40,80]$: 45, 70, 44, 75, 65. 
  Определить итоговое решение в соответствии с описанным механизмом. 
  Изменится ли результат, если пятый эксперт назовет вместо 65 оценку 55?

\item 
  Предприниматель намерен взять в аренду отель сроком на 1 год. 
  Имеются отели четырех типов: на 20, 30, 40 или 50 комнат. По условию аренды предприниматель должен оплатить все расходы, связанные с содержанием отеля. 
  Эти расходы (в у.\,е.) состоят из трех частей:
  \begin{enumerate}
    \item[] \textit{Расходы, не зависящие от выбора проекта отеля:}
      \item благоустройство территории~--- 10\,000;
      \item затраты на текущий ремонт и содержание~--- 1\,500;
      \item один ночной дежурный~--- 6\,000;
      \item один служащий для уборки территории~--- 8\,000.
    \item[] \textit{Расходы, пропорциональные числу комнат отеля:}
      \item меблировка одной комнаты~---  4\,000.;
      \item одна горничная на 10 комнат~---  6\,000;
      \item содержание одной комнаты~---  150;
      \item страхование одной комнаты~---  25.
    \item[] \textit{Расходы, пропорциональные среднему числу занятых комнат:}
      \item стирка, уборка~--- 5 в день;
      \item электричество, вода~--- 5 в день.
  \end{enumerate}
  Доход предпринимателя составляет 60 у.\,е. в день с каждой занятой комнаты. 
  Выбор какого проекта отеля следует считать оптимальным?  

  \item 
  Три игрока выбирают одного из четырех кандидатов в президенты компании. 
  Правило выбора таково:
  начиная с первого игрока, каждый игрок налагает вето на выбор одного из неотведенных кандидатов. 
  Единственный оставшийся кандидат считается избранным. 
  Функции выигрышей $U_i$ для каждого из игроков в зависимости от выбранного в президенты кандидата имеют вид:
  $$
    U_1 = (5,4,3,7),\qquad 
    U_2 = (6,7,5,4),\qquad
    U_3 = (3,8,5,4).
  $$
  Представить игру в виде дерева игры. Какой кандидат будет избран в президенты?  

  \item Восемь потребителей подали Центру свои заявки. Они таковы:
  11, 4, 6, 2, 8, 7, 12, 14. Центр обладает ресурсом $R = 50$. 
  Требуется распределить этот ресурс в соответствии с механизмом открытого управления. 

  \item Восемь экспертов сообщили следующие оценки из промежутка \([40,80]\): 45, 70, 44, 75, 65, 80, 66, 60. 
  Определить итоговое решение в соответствии с моделью открытого управления. 
  Изменится ли результат, если пятый эксперт назовет вместо 65 оценку 60? 
  А что будет, если второй эксперт назовет вместо 70 оценку 80?

  \item
  «Камень-ножницы-бумага и колодец тоже надо. Раз-два-три!» 
  Два игрока одновременно показывают ладонью одну из четырех фигур: камень, ножницы, бумагу или колодец. 
  Ножницы режут бумагу, тупятся об камень и тонут в колодце. 
  Бумага накрывает камень и закрывает колодец. 
  Камень тонет в колодце. 
  Если игроки показали одну и ту же фигуру, то ничья. 
  Решите игру в смешанных стратегиях.


   
\end{enumerate}
\end{document}	



Первый игрок выбирает в какой руке спрятать пуговицу. Затем второй игрок пытается угадать
в какой руке находится пуговица. За неугаданную руку второй платит первому два рубля. За
угаданную левую руку первый платит второму один рубль, за угаданную правую - три рубля.
а) С какой вероятностью монета следует прятать в левой руке?
б) В чью пользу эта игра?










\noindent\begin{enumerate*}
  \item $
  \begin{pmatrix}
  1.5 & 3\\
  2 & 1\\
  \end{pmatrix}
  $;
  \item $
  \begin{pmatrix}
  -1 & 1 \\
  1 & -1\\
  \end{pmatrix}
  $;
  \item $
  \begin{pmatrix}
  -1 & 1 & -1 & 2\\
  0 & -1 & 2 & -2\\
  \end{pmatrix}
  $;
  \item $
  \begin{pmatrix}
  1 & 4 \\
  3 & -2\\
  0 & 5\\
  \end{pmatrix}
  $.
  \end{enumerate*}
  \item 

  Чтобы никто не воспользовался хорошим мелом в ее отсутствие, учительница Сидорова М.\,И. (Марь Иванна) решила спрятать куски хорошего мела. 
  Она может спрятать их либо в самой классной комнате, либо в лаборантской. 
  Учитель математики Иванов И.\,П. будет искать мел или только в лаборантской, или только в классной комнате (у него урок уже начинается). 
  Если он находит мел, то получает полезность 1, если нет, то ему приходится писать плохим мелом, и он получает полезность 0. 
  Утром следующего дня Сидорова обнаруживает, что забыла, где спрятала мел. 
  Если она находит мел с первой попытки, то получает полезность 2, если только со второй попытки~--- полезность 1, и~полезность 0, если весь хороший мел использован Иваном Ивановичем.
  \begin{enumerate}
    \item Представьте игру в развернутой форме;
    \item Укажите информационные множества.
    \item Представьте игру в нормальной форме;
    \item Найдите все равновесия по Нэшу в чистых и смешанных стратегиях
  \end{enumerate}
  