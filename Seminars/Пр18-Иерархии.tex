%encoding=utf8%
\documentclass[a4paper,14pt]{extarticle}
%\usepackage{pgfpages}
%\pgfpagesuselayout{2 on 1}[a4paper,border shrink=5mm,landscape]

% Русская кодировка
\usepackage[T2A]{fontenc}
\usepackage[utf8]{inputenc}
\usepackage[russian]{babel}
%\usepackage[pdftex]{graphicx}
%\usepackage[pdftex,colorlinks,urlcolor=black, linkcolor=black, citecolor=black]{hyperref}
%        \pdfcompresslevel=9 % сжимать PDF

% необходимые модули
\usepackage{amssymb,amsmath,amsthm}
\usepackage{tabularx,multirow,hhline}
\usepackage{indentfirst}
\usepackage{tikz}
\usepackage[margin=6mm]{geometry}
\usepackage[inline]{enumitem}
\usepackage{multicol}
\setlist{itemsep=0pt,leftmargin=\parindent}

\AddEnumerateCounter{\Asbuk}{\@Asbuk}{\CYRM}
\AddEnumerateCounter{\asbuk}{\@asbuk}{\cyrm}
\newlist{dotenumerate}{enumerate}{10}
\setlist[enumerate,1]{label=\arabic*., ref=\arabic*} %списки со скобками
\setlist[enumerate,2]{label=\asbuk*), ref=(\asbuk*)} %списки со скобками

\usepackage{environ}

\NewEnviron{LOOP}[1]{
  \newcounter{nclone}
  \setcounter{nclone}{#1} 
  \par
  \loop
    \BODY
  \addtocounter{nclone}{-1}
  \ifnum \value{nclone}>0 \repeat}

%Полуторный интервал
\renewcommand{\baselinestretch}{1.00}
\pagestyle{empty}
\newcolumntype{C}{>{\centering\arraybackslash}X}


\sloppy
\binoppenalty=10000
\relpenalty=10000
\begin{document}
{\centering {\scriptsize Практическое занятие \textnumero~15 \par}
\bfseries Иерархии и приоритеты
\par\vspace{1mm}
}
 \begin{enumerate}
 	
    \item  
    Необходимо разрешить проблему распределения энергии в некоторой развитой стране между тремя ее крупнейшими пользователями: бытовым потреблением (БП), транспортом (ТР) и промышленностью (ПР). 
        
    Критериями, по которым оцениваются эти потребители, являются вклад в развитие экономики (Э), вклад в качество окружающей среды (С) и вклад в национальную безопасность (Б). 

    Общая цель~--- благоприятное социальное и политическое положение (Бл). 
    
    Пусть уже построена матрица попарных сравнений трех критериев: Э, С и Б в соответствии с их важностью для общей цели~--- Бл. 
    
    {\centering
    \begin{tabular}{c|ccc}
        Бл & Э   & С   &  Б   \\
        \hline
        Э  & 1   & 5   &  3   \\ 
        С  & 1/5 & 1   &  3/5 \\ 
        Б  & 1/3 & 5/3 &  1   \\
    \end{tabular}
    \par}

    Кроме того, с помощью экспертных оценок построены матрицы попарных сравнений отраслей по каждому из критериев Э, С и Б:

    {\centering
    \begin{tabular}{c|ccc}
        \textbf{Э}  & БП  & ТР  & ПР \\ 
        \hline
        БП & 1   & 3   & 5  \\
        ТР & 1/3 & 5   & 2  \\
        ПР & 1/5 & 1/2 & 1  \\
    \end{tabular}\quad
    \begin{tabular}{c|ccc}
        \textbf{С}  & БП  & ТР  & ПР\\ 
        \hline
        БП & 1   & 2   & 7 \\
        ТР & 1/2 & 1   & 5 \\
        ПР & 1/7 & 1/5 & 1 \\
    \end{tabular}\quad
    \begin{tabular}{c|ccc}
        \textbf{Б}  & БП  & ТР  & ПР\\ 
        \hline
        БП & 1   & 2   & 3 \\
        ТР & 1/2 & 1   & 2 \\
        ПР & 1/3 & 1/2 & 1 \\
    \end{tabular}
    \par}
    \item 
		Пусть есть три альтернативы  $A_1$ ,$A_2$, $A_3$.  Эксперт оценивает $A_1$ и $A_2$ как равнозначные; $A_2$ и $A_3$ как равнозначные, но $A_3$ имеет небольшое преимущество перед $A_1$ (2 балла).
		Постройте матрицу попарных сравнений, рассчитайте согласованность, определите приоритеты альтернатив.
		
	\item 	Предположим обнаружена новая альтернатива $A_4$, эксперт отдает ей незначительное предпочтение над перед $A_1$ (2балла) и $A_3$ (3 балла) и считает ее равнозначной $A_2$. 
	Как изменится согласованность матрицы сравнений  и приоритеты альтернатив?
	
	\item[] \textbf{Домашнее задание}
    \item 
    	Рассчитайте веса распределения времени между учебой, досугом и подработкой в соответствии с их общим вкладом в ваше личное благополучие через 7--10 лет, на которое влияют интересная работа, материальная обеспеченность и здоровье (семья). 
    	Считается, что первый уровень иерархии (цель)~--- благополучие, второй уровень (критерии)~--- интересная работа, материальная обеспеченность, здоровье (семья), третий уровень (альтернативы)~--- учеба, подработка, досуг.
 \end{enumerate}

\end{document}	