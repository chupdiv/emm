%encoding=utf8%
\documentclass[a5paper,10pt]{extarticle}
% \usepackage{pgfpages}

% Русская кодировка
\usepackage[T2A]{fontenc}
\usepackage[utf8]{inputenc}
\usepackage[russian]{babel}
%\usepackage[pdftex]{graphicx}
%\usepackage[pdftex,colorlinks,urlcolor=black, linkcolor=black, citecolor=black]{hyperref}
%        \pdfcompresslevel=9 % сжимать PDF

% необходимые модули
\usepackage{amssymb,amsmath,amsthm}
\usepackage{tabularx,multirow,hhline}
\usepackage{indentfirst}
\usepackage{tikz}
\usepackage[margin=8mm]{geometry}
\usepackage[inline]{enumitem}
\usepackage{multicol}
\setlist{itemsep=0pt,leftmargin=\parindent}

\AddEnumerateCounter{\Asbuk}{\@Asbuk}{\CYRM}
\AddEnumerateCounter{\asbuk}{\@asbuk}{\cyrm}
\newlist{dotenumerate}{enumerate}{10}
\setlist[enumerate,1]{label=\arabic*., ref=\arabic*} %списки со скобками
\setlist[enumerate,2]{label=\asbuk*), ref=(\asbuk*)} %списки со скобками

\usepackage{environ}

\NewEnviron{LOOP}[1]{
  \newcounter{nclone}
  \setcounter{nclone}{#1} 
  \par
  \loop
    \BODY
  \addtocounter{nclone}{-1}
  \ifnum \value{nclone}>0 \repeat}

%Полуторный интервал
\renewcommand{\baselinestretch}{1.00}
\pagestyle{empty}
\newcolumntype{C}{>{\centering\arraybackslash}X}


\sloppy
\binoppenalty=10000
\relpenalty=10000
\begin{document}
\begin{enumerate}
  {\item[]\centering {\footnotesize Практическое занятие \textnumero~27 \par}
          \bfseries Игры с нулевой суммой
          \par\vspace{1mm}
  }
\item Найдите решение игры
$
\begin{pmatrix}
  5 & 1 & 2 & 5 \\
  4 & 1 & 0 & 6 \\
  2 & 2 & 1 & 3 \\
  4 & 4 & 3 & 7 \\
\end{pmatrix}
$

% \item Алиса и Боб играют в следующую игру. Они  одновременно показывают от одного до трех пальцев. Выигрыш или проигрыш определяется числом
% показанных пальцев. Если сумма числа пальцев четная, то А получает от В платеж (в у.\,е.), равный этой сумме, если сумма пальцев нечетная, то А платит В.
% Определить оптимальные стратегии поведения сторон.
% Составьте матрицу игры. Определите верхнюю и нижнюю цену игры, седловую точку, если она есть.


\item Имеются две карты: туз и двойка. Игрок А (Алиса) наугад вынимает одну из карт; В (Боб) не видит, какую карту вынула А. 
Если А вынула туза, она заявляет: «у меня туз», и требует от противника 1 у.\,е. Если 
А вынула двойку, то он может либо сказать «у меня туз» и потребовать у противника 1 у.\,е., либо признаться, что у него двойка, и уплатить противнику 1 у.\,е.
Противник, если ему добровольно платят 1 у.\,е., может только принять сумму. 
Если же у него потребуют 1 у.\,е., то может либо поверить игроку А, что у него туз и отдать ему  1  у.\,е.,  либо  потребовать  проверки  с  тем,  чтобы  убедиться,  верно  ли утверждение игрока~А. Если в результате проверки окажется, что у А действительно туз, В должен уплатить 2 у.\,е. Если же окажется, что А обманывает и у него двойка, то игрок А уплачивает игроку В 2 у.е. Изобразить игру в виде дерева решений. 
Определить стратегии и матрицу платежей. 
Определить верхнюю и нижнюю цену игры, седловую точку, если она есть.
Проанализировать игру и найти оптимальную стратегию каждого игрока.

% \item Игра задана платежной матрицей. Определите верхнюю и нижнюю цену игры, седловую точку, если она есть:

% \noindent\begin{enumerate*}
% \item $
% \begin{pmatrix}
% 0.5 & 0.6 & 0.8\\
% 0.9 & 0.7 & 0.8\\
% 0.7 & 0.6 & 0.6\\
% \end{pmatrix}
% $;
% % \item $
% % \begin{pmatrix}
% % 0,3 & 0,6 & 0,8\\
% % 0,9 & 0,4 & 0,2\\
% % 0,7 & 0,5 & 0,4\\
% % \end{pmatrix}
% % $;

% \item 
% $
% \begin{pmatrix}
% 4 & 5 & 4\\
% 3 & 7 & 4\\ 
% 5 & 2 & 3\\
% \end{pmatrix}
% $.
% % \item 
% % $
% % \begin{pmatrix}
% % 8 & 9 & 9 & 4\\
% % 6 & 5 & 8 & 7\\
% % 3 & 4 & 5 & 6\\
% % \end{pmatrix}
% % $;

% % \item $
% % \begin{pmatrix}
% % 8 & 9 & 9 & 4\\
% % 6 & 5 & 8 & 7\\
% % 3 & 4 & 8 & 6\\
% % 8 & 9 & 9 & 4\\
% % \end{pmatrix}
% % $.
% \end{enumerate*}

\item Найдите решение матричной игры
\noindent\begin{enumerate*}
% \item $
% \begin{pmatrix}
% 1 & -1\\
% -1 & 1\\
% \end{pmatrix}
% $;

\item $
\begin{pmatrix}
2 & 5 & 8\\
7 & 6 & 10\\
12 & 10 & 8\\
\end{pmatrix}
$;


\item $
\begin{pmatrix}
   2 & 3 \\
  -1 & 7 \\
  0 & 5 \\
  4 & 2 \\
  6 & 1 \\
\end{pmatrix}$.


\end{enumerate*}
\item[] \textbf{Домашнее задание}
\item Найдите решение матричной игры:
\begin{enumerate*}
   \item
   $\begin{pmatrix}
    4 & 5 & 4\\
    3 & 7 & 4\\ 
    5 & 2 & 3\\
    \end{pmatrix}
    $
  \item 
    $\begin{pmatrix}
    -1 & 1 & -1 & 2\\
    0 & -1 & 2 & -2\\
  \end{pmatrix}$;
\end{enumerate*}

 \item (Уплата налогов). В конфликтной ситуации участвуют две стороны: $A$~--- государственная  налоговая  инспекция, $B$~--- налогоплательщик  с определенным годовым доходом, налог с которого составляет $T$ у.\,е. 

 У стороны $A$ имеется два способа поведения. Один из них, $A_1$, состоит в контролировании дохода налогоплательщика $B$ и взимании с него:
налога в размере $T$, если доход заявлен и соответствует действительному; налога в размере $T$ и штрафа в размере $W$, если заявленный в декларации доход меньше действительного или в случае сокрытия всего дохода.

Второй способ поведения $A_2$~--- не контролировать доход налогоплательщика.
У стороны В имеются три стратегии поведения: 
$B_1$–заявить о действительном  доходе; 
$B_2$–заявить  доход,  меньший  действительного, и,  следовательно, налог $T^*$ с заявленного дохода будет меньше $T$; 
$B_3$–скрыть доход, тогда не надо будет платить налог.
Составить  платежную  матрицу  игры  (матрицу  выигрышей $A$)  и  решить игру.

% \item Решите игру в смешанных стратегиях путем сведения к задаче линейного программирования:
% $$
% \begin{pmatrix}
% 1 & 2 & 4 & 5 \\ 
% 3 & 1 & 3 & 6 \\ 
% 2 & 3 & 1 & 5 \\ 
% \end{pmatrix}
% $$

\end{enumerate}
\end{document}	














\noindent\begin{enumerate*}
  \item $
  \begin{pmatrix}
  1.5 & 3\\
  2 & 1\\
  \end{pmatrix}
  $;
  \item $
  \begin{pmatrix}
  -1 & 1 \\
  1 & -1\\
  \end{pmatrix}
  $;
  \item $
  \begin{pmatrix}
  -1 & 1 & -1 & 2\\
  0 & -1 & 2 & -2\\
  \end{pmatrix}
  $;
  \item $
  \begin{pmatrix}
  1 & 4 \\
  3 & -2\\
  0 & 5\\
  \end{pmatrix}
  $.
  \end{enumerate*}
