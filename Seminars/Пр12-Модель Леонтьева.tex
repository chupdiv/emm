%encoding=utf8%
\documentclass[a5paper,10pt]{extarticle}
%\usepackage{pgfpages}
%\pgfpagesuselayout{2 on 1}[a4paper,border shrink=5mm,landscape]

% Русская кодировка
\usepackage[T2A]{fontenc}
\usepackage[utf8]{inputenc}
\usepackage[russian]{babel}
%\usepackage[pdftex]{graphicx}
%\usepackage[pdftex,colorlinks,urlcolor=black, linkcolor=black, citecolor=black]{hyperref}
%        \pdfcompresslevel=9 % сжимать PDF

% необходимые модули
\usepackage{amssymb,amsmath,amsthm}
\usepackage{tabularx,multirow,hhline}
\usepackage{indentfirst}
\usepackage{tikz}
\usepackage[margin=6mm]{geometry}
\usepackage[inline]{enumitem}
\usepackage{multicol}
\setlist{itemsep=0pt,leftmargin=\parindent}

\AddEnumerateCounter{\Asbuk}{\@Asbuk}{\CYRM}
\AddEnumerateCounter{\asbuk}{\@asbuk}{\cyrm}
\newlist{dotenumerate}{enumerate}{10}
\setlist[enumerate,1]{label=\arabic*., ref=\arabic*} %списки со скобками
\setlist[enumerate,2]{label=\asbuk*), ref=(\asbuk*)} %списки со скобками

\usepackage{environ}

\NewEnviron{LOOP}[1]{
  \newcounter{nclone}
  \setcounter{nclone}{#1} 
  \par
  \loop
    \BODY
  \addtocounter{nclone}{-1}
  \ifnum \value{nclone}>0 \repeat}

%Полуторный интервал
\renewcommand{\baselinestretch}{1.00}
\pagestyle{empty}
\newcolumntype{C}{>{\centering\arraybackslash}X}


\sloppy
\binoppenalty=10000
\relpenalty=10000
\begin{document}
{\centering {\scriptsize Практическое занятие \textnumero~12 \par}
\bfseries Балансовые модели
\par\vspace{1mm}
}
\begin{enumerate}

% \item  Народное хозяйство представлено тремя отраслями:
%   $P_1$~--- тяжелая промышленность;  $P_2$~--- легкая промышленность;
%   $P_3$~--- сельское хозяйство.
%   За отчетный год получены следующие данные:

%   {\centering \begin{tabular}{|c|c|c|c|c|c|c|}
%     \hline
%              & $P_1$ & $P_2$ & $P_3$ & $Y_0$ & $Y_0$ & $Y$ \\
%     \hline
%     $P_1$    & 80    & 15    & 25    &  80   & 300 & 150 \\
%     \hline
%     $P_2$    & 10    & 60    & 5     &  225  & 400 & 300 \\
%     \hline
%     $P_3$    & 10    & 30    & 30    &  30   & 400 &  50   \\ 
%     \hline
%     \hline
%     $t$      & 9     &  8    &  3    &  310     &     &     \\
%     \hline
% \end{tabular}
% \par}  
  
%   Требуется определить:
%   \begin{enumerate}
%       \item матрицу коэффициентов прямых материальных затрат $А$, матрицу «затраты-выпуск» $(Е-А)$ и вектор конечного потребления $Y$ для вектора валовых выпусков $X$;
%       \item матрицу коэффициентов полных материальных затрат $В$ и валовые объемы выпуска $X$ для вектора конечного потребления $Y$. Определить также валовые объемы межотраслевых поставок $x_{ij}$ ;
%       \item приросты валовых объемов выпуска, если конечное потребление должно измениться на 
%       $\Delta Y = (10\%,-10\%,50\%)$ по сравнению с Y;
%   \end{enumerate}
% %   г) матрицу полных затрат ресурсов S для матрицы М ее прямых затрат и
% %   суммарную потребность μ в ресурсах для вектора конечного потребления (от-
% %   четного Y 0 и планового Y);
% %   д) найти потребность в продукции всех отраслей материального производства (в руб.) для получения единицы конечного продукта j-того вида;
% %   е) найти полные затраты труда для заданного вектора прямых затрат тру-
% %   да в часах t =(t 1 , t 2 , t 3 );
% %   ж) определить цены на продукцию отраслей при заданной норме часовой
% %   ставки заработной платы P t ;
% %   з) с учетом различных категорий трудовых ресурсов, которые заданы
% %   матрицей L xj при часовой ставке P xj , рабочих k-й категории, определить пол-
% %   ные затраты труда каждой категории и цены продукции всех отраслей для век-
% %    6000 1800 2400 
% %   
% %   
% %   9000
% %   2700
% %   3600
% %   
% %   
% %   тора валового выпуска X, если L xj  
% %   , P kt =(2; 1,5; 2,2; 1,3).
% %   3000 900 1200 
% %    
% %    
% %   4500
% %   1800
% %   240


\item Экономическая система состоит из трех отраслей: 
$P_1$~--- промышленность, 
$P_2$~--- сельское хозяйство, 
$P_3$ – транспорт (в млрд. руб.).

{\centering \begin{tabular}{|c|c|c|c|c|c|c|}
    \hline
    & $P_1$ & $P_2$ & $P_3$ & $\Sigma$ & $Y$ & $X$ \\
    \hline
    $P_1$    & 20    & 50    &       &          & 200 & 300 \\
    \hline
    $P_2$    & 10    & 0     & 40    &          &     & 500 \\
    \hline
    $P_3$    & 0     &       &       &          & 240 &     \\ 
    \hline
    $\Sigma$ &       &       &       &  310     &     &     \\
    \hline
    $Z$      &       & 390   &       &          &     &     \\
    \hline
    $X$      &       &       &       &          &     &     \\
    \hline
\end{tabular}
\par}

Задание:
    \begin{enumerate}
        \item завершить составление баланса;
        \item рассчитать матрицу коэффициентов прямых затрат, полных затрат, косвенных затрат первого порядка;
        \item рассчитать валовые выпуски промышленности и сельского хозяйства и конечный продукт транспорта на планируемый период при условии увеличения конечного продукта первых двух отраслей на 3\%, оставив без изменения объем валового продукта транспорта;
        \item рассчитать новую производственную программу каждой отрасли.
    \end{enumerate}

\item Дана матрица прямых затрат 
$A=\begin{pmatrix}
        0.18 & 0.08 & 0.44\\
        0.36 & 0.25 & 0.70\\
        0.48 & 0.11 & 0.28\\
\end{pmatrix}$
и вектор конечного потребления $Y=(570; 280; 180)$.
\begin{enumerate}
    \item Проверить продуктивность матрицы~$A$.
    \item Найдите соответствующие объёмы валового выпуска каждой отрасли.
    \item Пусть надо удвоить выпуск конечного продукта второй
    отрасли. На сколько процентов должны измениться объёмы валового выпуска третьей отрасли?    
\end{enumerate}

\item Два цеха предприятия выпускают продукцию двух видов: цех № 1 – продукцию $B$, цех № 2~--- продукцию $C$. Часть производимой продукции направляется на внутреннее потребление, а остальная является конечным продуктом. Коэффициенты прямых затрат заданы матрицей 
$A=\begin{pmatrix}
0.14 & 0.41\\
0.3 & 0.13\\
\end{pmatrix}
$.
Плановый объем конечного потребления продукции $B$ составляет  600 тонн, а продукции $C$~--- 300 тонн. 
Составьте плановую модель выпуска продукции (валового и конечного продукта) с учетом внутреннего потребления. 

\item Задание~1 стр. 29. А.\,В. Ряттель <<Основы экономико-математического моделирования>>
\item Для желающих. Упражнение~2 стр. 28. А.\,В. Ряттель <<Основы экономико-математического моделирования>>
\end{enumerate}
%\newpage
%{\centering {\scriptsize Практическое занятие \textnumero~13 \par}
%\bfseries Управление запасами
%\par\vspace{1mm}
%}
%\begin{enumerate}
%    \item Ежедневный спрос на некоторый продукт составляет 100 единиц.
%    Затраты на приобретение каждой партии продукта равны 100 ден.\;ед. не зависимо от объема партии.
%    Затраты на хранение единицы продукта 0.02~ден.~ед. в~сутки. 
%    Определить наиболее экономичный объем партии и интервал между поставками.
%    \item Предположим теперь, что возможен дефицит товара, который влечет штраф в размере 0,03~ден.~ед. на~единицу продукции. Определить наиболее экономичный объем хранимой на складах продукции.
%    \item Фирма может производить изделие или покупать его. Если фирма сама выпускает изделие, то каждый запуск его в производство обходится в~20~руб. Интенсивность производства составляет 120~шт. в день. Если изделие закупается, то затраты на осуществление заказа равны 15 руб. Затраты на содержание изделия в запасе независимо от того, закупается оно или производится, равны 2 коп. в день. Потребление изделия фирмой оценивается в 26000~шт. в~год.
%    Предполагая, что фирма работает без дефицита, определите, что выгоднее: закупать или производить изделие (в~месяце 22~рабочих дня).
%    \item На некотором станке производятся детали в количестве 2000 штук в месяц. Эти детали используются для производства продукции на другом станке с интенсивностью 500 шт. в месяц. По оценкам специалистов компании, издержки хранения составляют 50 коп. в год за одну деталь. Стоимость производства одной детали равна 2,50~руб., а~стоимость на подготовку производства составляет 1000 руб. Каким должен быть размер партии деталей, производимой на первом станке, с какой частотой следует запускать производство этих партий?
%\end{enumerate}

\end{document}	