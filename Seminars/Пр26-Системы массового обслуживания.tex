%encoding=utf8%
\documentclass[a5paper,11pt]{extarticle}
% \usepackage{pgfpages}
% \pgfpagesuselayout{2 on 1}[a4paper,border shrink=0mm,landscape]

% Русская кодировка
\usepackage[T2A]{fontenc}
\usepackage[utf8]{inputenc}
\usepackage[russian]{babel}


% необходимые модули
\usepackage{amssymb,amsmath,amsthm}
\usepackage{tabularx,multirow,hhline}
\usepackage{indentfirst}
\usepackage{tikz}
\usepackage[margin=8mm,left=8mm]{geometry}
\usepackage{paralist}
\usepackage[inline]{enumitem}
\usepackage{multicol}
\usepackage{diagbox}
\setlist{noitemsep,leftmargin=\parindent}

\AddEnumerateCounter{\Asbuk}{\@Asbuk}{\CYRM}
\AddEnumerateCounter{\asbuk}{\@asbuk}{\cyrm}
\newlist{dotenumerate}{enumerate}{10}
\newlist{denumerate}{enumerate}{1}
\setlist[enumerate,1]{itemsep=1pt,label=\arabic*., ref=\arabic*} %списки со скобками
\setlist[enumerate,2]{label=\textbf{(\asbuk*)}, ref=\arabic*} %списки со скобками
\setlist[denumerate,1]{noitemsep,label={Д.\,\arabic*.}, ref=\arabic*} %списки со скобками

\usepackage{environ}

\NewEnviron{LOOP}[1]{
  \newcounter{nclone}
  \setcounter{nclone}{#1} 
  \par
  \loop
    \BODY
  \addtocounter{nclone}{-1}
  \ifnum \value{nclone}>0 \repeat}

%Полуторный интервал
\renewcommand{\baselinestretch}{1.00}
\pagestyle{empty}
\newcolumntype{C}{>{\centering\arraybackslash}X}


\sloppy
\binoppenalty=10000
\relpenalty=10000

\begin{document}
{\centering {\scriptsize Практическое занятие \par}
\bfseries Системы массового обслуживания \par}
\begin{enumerate}
     \item 
    В результате наблюдений за потоком покупателей в течение 10 дней работы магазина были получены следующие данные (регистрация числа покупателей в магазине осуществлялась каждый час):
    
    {\small\centering\begin{tabular}{|c|cccccccc|}
        \hline
        \backslashbox{День}{Час}  & 1 & 2 & 3 & 4 & 5 & 6 & 7 & 8 \\  
        \hline
        1 & 2 & 4 & 2 & 3 & 4 & 3 & 5 & 2 \\  
        2 & 3 & 2 & 3 & 2 & 7 & 2 & 3 & 3 \\  
        3 & 1 & 3 & 4 & 3 & 4 & 6 & 4 & 2 \\  
        4 & 4 & 4 & 4 & 5 & 9 & 3 & 4 & 4 \\  
        5 & 2 & 1 & 3 & 7 & 3 & 6 & 2 & 3 \\  
        6 & 3 & 2 & 3 & 4 & 5 & 5 & 3 & 2 \\  
        7 & 4 & 3 & 4 & 3 & 8 & 3 & 4 & 3 \\  
        8 & 1 & 2 & 2 & 4 & 3 & 4 & 2 & 4 \\  
        9 & 3 & 4 & 6 & 3 & 4 & 2 & 4 & 2 \\  
       10 & 2 & 2 & 3 & 5 & 6 & 4 & 2 & 5 \\ 
       \hline
    \end{tabular}\par}
    \begin{enumerate}
        \item Определить интенсивность входящего потока покупателей в расчете на час работы магазина;
        \item используя критерий Пирсона с уровнем значимости $\alpha=0.05$, обосновать предположение, что поток описывается пуассоновским законом распределения.
        \end{enumerate}
    
    
    \item Булочная «Горячий хлеб» имеет одного контролера-кассира. В течение часа приходят в среднем 54 покупателя. Среднее время обслуживания контролером-кассиром одного покупателя составляет 1 мин. Определить выручку от продажи, характеристики СМО и провести анализ ее работы.

    \medskip
    \item[] \textbf{Домашнее задание}
    
    \item Статистическими исследованиями в результате наблюдения установлено, что интенсивность потока телефонных звонков коммерческому директору фирмы $\lambda = 1.2$ вызова в минуту, средняя продолжительность разговора $t_\text{обсл.} = 2.5\;\text{минуты}$ и все потоки событий имеют характер простейших пуассоновских потоков. 
    Определить предельную пропускную способность СМО, вероятность отказа, а также полное число обслуженных и необслуженных заявок в течение одного часа работы.
    
    \item Коммерческая фирма занимается посреднической деятельностью по продаже автомобилей и осуществляет часть переговоров по трем телефонным линиям. В среднем поступает 75 звонков в час. Среднее время предварительных переговоров справочного характера составляет 2 минуты. Определить характеристики СМО, дать оценку работы СМО.


\end{enumerate}
\newpage
\end{document}
