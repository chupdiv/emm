%encoding=utf8%
\documentclass[a4paper,12pt]{extarticle}

% Русская кодировка
\usepackage[T2A]{fontenc}
\usepackage[utf8]{inputenc}
\usepackage[russian]{babel}

\usepackage[pdftex]{graphicx}
\usepackage[pdftex,colorlinks,urlcolor=blue, citecolor=magenta]{hyperref}
%        \pdfcompresslevel=9 % сжимать PDF

% необходимые модули
\usepackage{amssymb,amsmath,amsthm}
\usepackage{indentfirst}
\usepackage{tikz}
\usepackage[margin=9mm]{geometry}
\usepackage[inline]{enumitem}
\setlist{noitemsep,leftmargin=\parindent}

\AddEnumerateCounter{\Asbuk}{\@Asbuk}{\CYRM}
\AddEnumerateCounter{\asbuk}{\@asbuk}{\cyrm}
\newlist{dotenumerate}{enumerate}{10}
\setlist[enumerate,1]{label=\arabic*., ref=\arabic*} %списки со скобками
\setlist[enumerate,2]{label=\asbuk*), ref=(\asbuk*)} %списки со скобками

\usepackage{tabularx}
\newcolumntype{C}{>{\centering\arraybackslash}X}


%Полуторный интервал
\renewcommand{\baselinestretch}{1.00}
\pagestyle{empty}
\begin{document}
{\centering \small Домашнее задание к практическому занятию \textnumero~3 \par
\bfseries \large Сетевые модели \par}


\textbf{Кейс 1.} 
В таблице приведены данные о сроках и стоимости работ
\par{\centering
\begin{tabular}{ccccc}
\hline
Работа & \multicolumn{2}{c}{Нормальные сроки} & \multicolumn{2}{c}{Сжатые сроки}\\
 & $t_\text{норм.}$	& $c_\text{норм.}$ & $t_\text{форс.}$ & $c_\text{форс.}$ \\
\hline
(1,2) & 7 &  95 & 4 & 100 \\
(1,3) & 6 &  90 & 3 &  97 \\
(1,4) & 5 &  86 & 4 & 104 \\
(2,3) & 7 &  92 & 7 &  98 \\
(2,4) & 6 &  87 & 5 &  93 \\
(3,4) & 7 & 112 & 5 & 120 \\
(3,5) & 8 & 101 & 5 & 113 \\
(4,5) & 9 &  97 & 6 & 109 \\
\hline
\end{tabular}\par}

Требуется:
\begin{enumerate}
\item построить сетевой график;
\item определить критический путь и стоимость проекта при нормальных значениях продолжительности всех работ;
\item определить срок завершения, критический путь и стоимость проекта при минимально возможной продолжительности всех работ;
\item найти минимальную стоимость проекта при том же сроке его завершения;
\end{enumerate}

%%textbf{Кейс 1} 
%Бензозаправочная станция должна быть построена на уже подготовленном месте. Она состоит из двух объектов: заправочной площадки и здания конторы. 
%
%На заправочной площадке располагаются касса и насосы, а помещение для административного персонала, воздушный компрессор и туалеты общего пользования находятся в здании конторы. 
%
%К насосам примыкает бетонированная яма, содержащая резервуары с бензином.
%Компания, которой принадлежит станция, желает установить счетчик-расходомер и соединить его с магистралью. 
%
%Бензонасосы необходимо получить у изготовителя, а после монтажа подсоединить к резервуарам с бензином и электросети.
%
%Страховая компания настаивает на том, чтобы для охраны здания конторы была предусмотрена эффективная система сигнализации. 
%После установки этой системы страховая компания проводит свою инспекцию. 
%Сигнализация работает от основной электрической сети. 
%
%Все оборудование для конторы и туалетов должно быть получено по специальному заказу и после доставки должно храниться в укрытии во избежание его порчи. 
%
%Некоторые предметы требуют окраски уже после установки на место.
%
%Сжатый воздух для накачивания шин обеспечивается электрическим компрессором, который перед началом работы должен быть проверен компетентным специалистом.
%
%Воздухозаборная линия компрессора монтируется совместно с основными подземными коммуникациями, а устройства для забора воздуха устанавливаются на стене ограждения.
%
%На прилегающем к бензозаправочной станции участке дороги устанавливаются соответствующие дорожные знаки и указатели, информирующие водителей о местоположении станции. 
%Предполагается, что дорожные знаки и указатели будут установлены тогда, когда станция будет
%готова к эксплуатации.
%
%Подготовка к началу работ включает, кроме обычных приготовлений, доставку на место специального прицепа для хранения инструментов, оборудования и т.\,п.: в прицепе располагаются также диспетчерская и подсобная службы. 
%
%После завершения работ прицеп должен быть удален, равно как и все строительные леса и вспомогательное оборудование. Эта операция называется «вывоз использованного оборудования и уборка территории».
%
%
%В таблице приведены нормальные и форсированные стоимости. 
%
%
%
%\par{\centering
%\begin{tabular}{p{9cm}cccc}
%\hline
%Работа & \multicolumn{2}{c}{Длительность, нед.} & \multicolumn{2}{c}{Стоимость, долл.}\\
% & $t_\text{норм.}$	& $t_\text{форс.}$ & $c_\text{норм.}$ & $c_\text{форс.}$ \\
%\hline
%Рытье котлована для заправочной площадки 	&  1 &  1   &  150 &  150 \\
%Строительство площадки 						&  1 &  0,8 &  200 &  225 \\
%Строительство кассы 						&  2 &  1,6 & 1000 & 1200 \\
%Получение насосов 							& 16 & 10   & 1500 & 1500 \\
%Установка насосов 							&  1 &  1   &  300 &  300 \\
%Подсоединение насосов 						&  2 &  0,4 &  100 &  120 \\
%Инспекция установки насосов 				&  2 &  2   &    0 &    0 \\
%Получение оборудования 						&  8 &  1   &  100 &  200 \\
%Окраска и оборудование конторы и туалетов 	&  2 &  1   &  200 &  250 \\
%Проведение освещения 						&  1 &  0,4 &    0 &    0\\
%Рытье котлована для конторы 				&  1 &  1   &  300 &  300 \\
%Строительство конторы 						&  1 &  1   &  150 &  150  \\
%Установка оборудования 						&  2 &  1,4 & 9000 & 9950 \\
%Установка сигнализации 						&  1 &  0,6 &  200 &  250  \\
%Подключение сигнализации 					&  2 &  1   &  100 &  120  \\
%\hline
%\end{tabular}\par}
%
%Выполнение проекта позже запланированного времени ведет к штрафу 500 долл./день, а за каждый сэкономленный день назначена премия 200 долл. 
%
%Накладные расходы подрядчика равны 40 долл./день. Определите длительность проекта, при которой прибыль подрядчика будет максимальной.

\bigskip
\textbf{Кейс 2.} 
31 марта Мэри Джексон ворвалась в гостиную своих родителей и объявила, что она и ее друг по колледжу Лэрри Адамс решили пожениться. Оправившись от первого потрясения, мама обняла Мэри, поздравила ее и спросила: <<И когда же свадьба?>>. За этим вопросом последовал  диалог:
\begin{description}
\item[Мэри:] 22 апреля.
\item[Мама:] Что?!
\item[Папа:] Свадьба будет главным событием года. Зачем же так спешить?
\item[Мэри:] Потому что в это время цветут вишни, и свадебные фотографии
получатся очень красивыми!
\item[Мама:] Но, дорогая, мы не успеем все подготовить. Помнишь, сколько всего
нам пришлось делать, когда выходила замуж твоя сестра? Даже если мы
приступим к работе завтра, все равно потребуется время чтобы заказать
церемонию венчания в церкви и зал для приемов. Они требуют, чтобы заявку
подавали минимум за 17 дней. На украшение церкви также уйдет не меньше трех
дней. Хотя... Думаю, если заплатить долларов на 100 больше, то этот срок можно
было бы сократить до 10 дней.
\item[Мэри:] Да, знаете, я хочу, чтобы свидетельницей на моей свадьбе была
Джейн Саммерс.
\item[Папа:] Но она же сейчас в Гватемале? Ей понадобится не меньше 10 дней,
чтобы собраться и приехать сюда.
\item[Мэри:] Да, но если она полетит самолетом, то будет здесь уже через два дня.
А это всего какие-то 500 долларов. Ей все равно стоит приехать раньше, ведь надо
еще подогнать ее наряд.
\item[Мама:] А угощенье! Чтобы отобрать продукты, приготовить еду и
сервировать столы, нам понадобится два дня, а в ресторане <<У Джека>> требуют,
чтобы заявка подавалась не менее чем за 10 дней до пробного ужина (накануне
свадьбы).
\item[Мэри:] Да, мам, а можно, я буду в твоем платье?
\item[Мама:] Ну, конечно, но надо будет заменить кружева. Их можно заказать в
Нью-Йорке, ведь нам все равно придется заказывать материал для платьев
свидетельниц. На заказ и получение материала уйдет восемь дней. Но сначала
надо, чтобы они прислали образцы, а это еще три дня.
\item[Папа:] Если сделать заказ авиапочтой, а это будет стоить на 25 долларов
дороже, можно будет получить материал и кружева уже через пять дней.
\item[Мэри:] И я хочу, чтобы платья сшила миссис Уатсон.
\item[Папа:] Но она берет 120 долларов в день!
\item[Мама:] Если бы мы все сшили сами, на это ушло бы 11 дней. А если бы
миссис Уатсон согласилась помочь, то этот срок можно было бы сократить до
шести дней, и мы заплатили бы по 120 долларов только за эти дни.
\item[Мэри:] Но я хочу, чтобы это делала именно она, и никто другой!
\item[Мама:] Два дня займет последняя примерка. Обычно еще два дня уходит на
то, чтобы почистить и отгладить новые платья, но если заплатить в нашей новой
химчистке дополнительно 30 долларов, это сделают за день.
\item[Папа:] Все должно быть сделано ко дню накануне свадьбы, через 21 день.
\item[Мама:] Но мы кое о чем забыли. О приглашениях.
\item[Папа:] Надо заказать их в типографии Боба, а у него на изготовление
приглашений обычно уходит 12 дней. Но я совершенно уверен, что, если мы
заплатим ему на 35 долларов больше, он сделает все за пять дней.
\item[Мама:] Дня три уйдет на то, чтобы выбрать образец приглашения. И, кроме
того, надо, чтобы на всех приглашениях был напечатан наш адрес.
\item[Мэри:] Да, это будет элегантно!
\item[Мама:] Приглашения необходимо отправить не позже, чем за 10 дней до
свадьбы. Если этот срок задержать, то некоторые родственники получат их
9слишком поздно и не успеют приехать, и это, конечно, их страшно огорчит.
Бьюсь об заклад, что если мы не разошлем приглашения хотя бы за восемь дней
до свадьбы, тетя Этель не сможет приехать и наверняка уменьшит свой
свадебный подарок долларов на 200.
\item[Папа:] Ничего себе!
\item[Мама:] Надо будет отвезти приглашения прямо на почту, а это тоже займет
день. На то, чтобы надписать все адреса, потребуется дня четыре, если только мы
не наймем кого-нибудь помочь нам. И, кроме того, мы не сможем начать эту
работу до тех пор, пока приглашения не будут напечатаны. А если кого-то нанять,
мы сможем сэкономить два дня, потратив по 25 долларов за каждый из этих дней.
\item[Мэри:] Надо еще приготовить подарки для подружек. Я думаю, что смогу
сделать это за день.
\item[Мама:] Прежде чем начинать писать приглашения, надо составить список
гостей. Боже, это займет не меньше четырех дней, и только я могу разобраться в
наших адресных данных.
\item[Мэри:] Ах, мама, я так взволнована! Мне кажется, что можно было бы
привлечь к подготовке родственников.
\item[Мама:] Дорогая, я просто не представляю, как мы это можем сделать. Итак,
нам надо выбрать образец приглашений, заказать церемонию в церкви и...
\item[Папа:] Послушай, Мэри, а почему бы тебе просто не взять 1500 долларов и
не сбежать с женихом? Свадьба твоей сестры стоила 1200 долларов, а ведь нам не
пришлось никого привозить из Гватемалы, пользоваться авиадоставками и прочее
в этом роде.
\end{description}

\textbf{Вопросы:}
\begin{enumerate}
\item С учетом необходимых действий и их последовательности постройте
сетевой график подготовки свадьбы.
\item Определите полные пути проекта. Какие из них являются критическими?
\item Каков срок выполнения работ без дополнительных затрат.
\item Какие минимальные дополнительные затраты понадобятся для того, чтобы сыграть
свадьбу 22 апреля?
\end{enumerate}


\end{document}
