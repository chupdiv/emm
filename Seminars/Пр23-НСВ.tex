%encoding=utf8%
\documentclass[a4paper,17pt]{extarticle}
% \usepackage{pgfpages}
% \pgfpagesuselayout{2 on 1}[a4paper,border shrink=0mm,landscape]

% Русская кодировка
\usepackage[T2A]{fontenc}
\usepackage[utf8]{inputenc}
\usepackage[russian]{babel}


% необходимые модули
\usepackage{amssymb,amsmath,amsthm}
\usepackage{tabularx,multirow,hhline}
\usepackage{indentfirst}
\usepackage{tikz}
\usepackage[margin=15mm]{geometry}
\usepackage{paralist}
\usepackage[inline]{enumitem}
\usepackage{multicol}
\setlist{nosep,leftmargin=\parindent}

\AddEnumerateCounter{\Asbuk}{\@Asbuk}{\CYRM}
\AddEnumerateCounter{\asbuk}{\@asbuk}{\cyrm}
\newlist{dotenumerate}{enumerate}{10}
\newlist{denumerate}{enumerate}{1}
\setlist[enumerate,1]{itemsep=5pt,label=\arabic*., ref=\arabic*} %списки со скобками
\setlist[enumerate,2]{label=\textbf{(\asbuk*)}, ref=\arabic*} %списки со скобками
\setlist[denumerate,1]{noitemsep,label={Д.\,\arabic*.}, ref=\arabic*} %списки со скобками

\usepackage{environ}

\NewEnviron{LOOP}[1]{
  \newcounter{nclone}
  \setcounter{nclone}{#1} 
  \par
  \loop
    \BODY
  \addtocounter{nclone}{-1}
  \ifnum \value{nclone}>0 \repeat}

%Полуторный интервал
\renewcommand{\baselinestretch}{1.00}
\pagestyle{empty}
\newcolumntype{C}{>{\centering\arraybackslash}X}


\sloppy
\binoppenalty=10000
\relpenalty=10000

\newcommand{\bitem}{\textbf{\item}}
\begin{document}
{\centering 
{\centering {\scriptsize Практическое занятие \textnumero~23 \par}
\bfseries Непрерывные случайные величины \par}
\begin{enumerate}
    \item Дана функция распределения случайной величины~$X$: 
    $F(x)= \begin{cases}
      0, & x \leqslant 0,\\
      \frac{x^2}{4}, & 0< x \leqslant 2,\\
      1,  & x>2.
    \end{cases}
    $
    \begin{enumerate*}
        \item Найти плотность вероятности $f(x)$. 
        \item Построить графики $F(x)$ и~$f(x)$. 
        \item Найти вероятности $P(1)$, $P(X<1)$, $P(1<X<2)$;
        \item Вычислить медиану $\text{Me}(X)$.
        % \item Вычислить $M(X)$, $D(X)$, $\sigma(X)$, $Me(X)$.
    \end{enumerate*}
    
    \item Цена деления шкалы измерительного прибора равна $0.2$. Показания округляют до ближайшего целого числа. Полагая, что при отсчёте ошибка округления распределена по равномерному закону, найти: 
    \begin{enumerate*}
        \item математическое ожидание, дисперсию и среднее квадратическое отклонение этой случайной величины; 
        \item вероятность того, что ошибка округления меньше $0.04$.    
    \end{enumerate*}
    
    
    \item Вес пойманной рыбы подчиняется нормальному закону распределения с параметрами $\mu = 3775$\,г, $\sigma=25$\,г. 
    Найти вероятность того, что вес одной рыбы будет от~300 до~425\,г.
%    \item  Математическое ожидание и среднее квадратическое отклонение нормально распределенной случайной величины $X$ соответственно равны 10 и~2. Найти вероятность того, что в результате испытания $X$ примет значение, заключённое в интервале $(12,14)$.
    
    \item Станок-автомат изготовляет валики, причём контролирует их диаметр~$X$. Считая, что $X$~--- нормально распределенная случайная величина с математическим ожиданием 100\,мм и средним квадратическим отклонением 0,1\,мм, найти интервал, симметричный относительно математического ожидания, в котором с вероятностью 0,9973 будут заключены диаметры изготовленных валиков.
    \item Цена некой ценной бумаги нормально распределена. В те­чение последнего года
      20\% рабочих дней она была ниже 88 ден. ед., а 75\% -выше 90 ден. ед.  Найти: 
    \begin{enumerate*}
      \item математическое ожидание и сред­нее квадратическое отклонение цены ценной бумаги;
      \item вероятность того, что в день покупки цена будет заключена в пределах от~83  до~96~ден.\;ед.;
      \item с надежностью  0,95 определить максимальное отклонение цены ценной бумаги от среднего (прогнозного) значения (по абсолют­ной величине).
    \end{enumerate*}
%     \item Автомат штампует детали. Контролируется длина детали $X$, которая распределена нормально с математическим ожиданием (проектная длина), равным 50\,мм. Фактически длина изготовленных деталей не менее 32 и~не более 68\,мм. Найти вероятность того, что длина наудачу взятой детали: а) более 55 мм; б) меньше 40 мм.

\end{enumerate}
\newpage
\end{document}
