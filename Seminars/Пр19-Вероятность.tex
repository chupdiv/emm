%encoding=utf8%
\documentclass[a4paper,14pt]{extarticle}
%\usepackage{pgfpages}
%\pgfpagesuselayout{2 on 1}[a4paper,border shrink=5mm,landscape]

% Русская кодировка
\usepackage[T2A]{fontenc}
\usepackage[utf8]{inputenc}
\usepackage[russian]{babel}
%\usepackage[pdftex]{graphicx}
%\usepackage[pdftex,colorlinks,urlcolor=black, linkcolor=black, citecolor=black]{hyperref}
%        \pdfcompresslevel=9 % сжимать PDF

% необходимые модули
\usepackage{amssymb,amsmath,amsthm}
\usepackage{tabularx,multirow,hhline}
\usepackage{indentfirst}
\usepackage{tikz}
\usepackage[margin=10mm]{geometry}
\usepackage[inline]{enumitem}
\usepackage{multicol}
\setlist{itemsep=0pt,leftmargin=\parindent}

\AddEnumerateCounter{\Asbuk}{\@Asbuk}{\CYRM}
\AddEnumerateCounter{\asbuk}{\@asbuk}{\cyrm}
\newlist{dotenumerate}{enumerate}{10}
\setlist[enumerate,1]{label=\arabic*., ref=\arabic*} %списки со скобками
\setlist[enumerate,2]{label=\asbuk*), ref=(\asbuk*)} %списки со скобками

\usepackage{environ}

\NewEnviron{LOOP}[1]{
  \newcounter{nclone}
  \setcounter{nclone}{#1} 
  \par
  \loop
    \BODY
  \addtocounter{nclone}{-1}
  \ifnum \value{nclone}>0 \repeat}

%Полуторный интервал
\renewcommand{\baselinestretch}{1.00}
\pagestyle{empty}
\newcolumntype{C}{>{\centering\arraybackslash}X}


\sloppy
\binoppenalty=10000
\relpenalty=10000
\begin{document}
{\centering
 {\small Практическое занятие \textnumero~19 \par}
{\bfseries События и вероятности}
\par}
\begin{enumerate}
% \item  
% Все натуральные числа от 1 до 30 записаны на одинаковых карточках и
% помещены в урну. После тщательного перемешивания карточек из урны
% извлекается одна карточка. Какова вероятность того, что число на взятой
% карточке окажется кратным~5?

\item[]{\centering\itshape Классическое определение вероятности\par}
\item 
Брошены две игральные кости. Найти вероятности событий: а) сумма
выпавших очков равна семи; б) сумма выпавших очков равна пяти, а
произведение – четырем.

 \item 
 Какова вероятность того, что в наудачу выбранном: а) двузначном; б) четырехзначном числе цифры одинаковы?

\item 
Монета брошена два раза. Найти вероятность того, что хотя бы один раз
появится герб. Как изменится вероятность, если монету бросают~$3$~раза?






% \item 
% В группе 30 человек. Необходимо выбрать старосту и его заместителя. Сколько существует способов это сделать? Какова вероятность того, что это будут Петя И. и~Вова~С.?

% \item 
% Алфавит племени Мумбо-Юмбо состоит из трех букв А, Б, В. Словом является любая последовательность, состоящая не более, чем из 4 букв. Сколько слов в языке племени Мумбо-Юмбо?

% \item 
% Сколько существует семизначных чисел, состоящих из цифр 4, 5 и 6, в которых цифра 4 повторяется 3 раза, а цифры 5 и 6 – по 2 раза?

\item 
Буквы \texttt{Т, Е, И, Я, Р, О} написаны на отдельных карточках. Ребенок берет
карточки в случайном порядке и прикладывает одну к другой: а) 3~карточки; б) все карточки. Какова вероятность того, что получится
слово: а) «\texttt{ТОР}»; б) «\texttt{ТЕОРИЯ}»?

 \item 
 На отдельных карточках написаны буквы: три буквы \texttt{А}, две буквы \texttt{Н} и~одна буква~\texttt{С}. Ребенок берет карточки в случайном порядке и прикладывает
 одну к другой. Какова вероятность того, что получится слово «\texttt{АНАНАС}»?


% \item 
% Набирая номер телефона, абонент забыл: а) последнюю цифру; б) три
% последних цифры номера телефона. Найти вероятность того, что абонент
% дозвонится до нужного номера с первого раза, если в б) он помнит, что
% цифры различны (могут повторяться).

\item 
Гость забыл код подъездного замка и пытается его угадать. Какова
вероятность того, что с первого раза будет набран верный код, если он
содержит: а) две; б) три цифры (считается, чтобы открыть дверь
подъезда, необходимо нажать цифры кода одновременно).


% \item 
% В лифт на 1-м этаже девятиэтажного дома вошли 4 человека, каждый из
% которых может выйти независимо друг от друга на любом этаже с 2-го по
% 9-й. Какова вероятность того, что все пассажиры выйдут: а) на 6-м
% этаже; б) на одном этаже?


\item 
    Собрание, на котором присутствуют 25 человек, в том числе 5 женщин, выбирает делегацию из 3 человек. Найти вероятность того, что в делегацию войдут 2 женщины и 1 мужчина. 
\item Из 200 рабочих опаздывают на работу трое рабочих. Найти вероятность того, что два случайно выбранных рабочих опоздали сегодня на работу.
\item 
В ящике находятся 15 красных, 9 голубых и 6 зеленых шаров. Наудачу
вынимают 6 шаров. Какова вероятность того, что вынуты 1 зеленый, 2
голубых и 3 красных шара?

{\item[] \centering\itshape Геометрическое определение вероятности\par}
\item 
На отрезке длины 20 см помещён меньший в 2 раза отрезок. Найти
вероят\-ность того, что точка, случайным образом поставленная на
больший отрезок, попадёт также и на меньший отрезок. 
\item 
Плоскость разграфлена на клетки параллельными прямыми, находящиеся друг от друга
на расстоянии $2a$. На плоскость случайным образом брошена
монета радиуса $r$ ($r<a$). Найти вероятность того, что монета не пересечёт ни одной из прямых.

% \item 
% 	В круг вписан квадрат. В круг случайным образом бросается точка. Какова вероятность того, что точка попадёт в квадрат?
% \item 
% На отрезке $OA$ длины $L$ числовой оси $Ox$ случайным
% образом поставлены две точки: $B(x)$ и $C(y)$.
% Найти вероятность того, что из трех получившихся от\-резков можно
% построить треугольник.
\item 
Случайным образом взяты два положительных числа $x$ и $y$, каждое из кото\-рых не превышает~2. 
Найти вероятность того, что произведение $xy$ будет не больше единицы, а частное  ${\frac{y}{x}}$ не больше 2.

\item Автобус подходит к остановке каждые 20 минут, а тролейбус~--- каждые 15 мин. Найдите вероятность того, что пассажир, пришедший однажды на остановку, будет ожидать транспорт не более 10 мин.

\pagebreak
{\item[]\centering \itshape Домашнее задание\par}
\item 
Куб, все грани которого окрашены, распилен на тысячу кубиков
одинакового размера, которые затем тщательно перемешаны. Найти
вероятность того, что наудачу извлеченный кубик имеет: а) одну; б) две;
в) три окрашенных грани.

\item 
По условиям лотереи «Спортлото 6 из 45» участник лотереи, угадавший
4,5,6 видов спорта из отобранных при случайном розыгрыше 6 видов спорта
из 45, получает денежный приз. Найти вероятность того, что будут
угаданы; а) 6 цифр; б) 4 цифры.



 \item 
 Пятитомное собрание сочинений на полке в случайном порядке. Какова
 вероятность того, что книги стоят слева направо в порядке нумерации
 томов (от 1 до 5)?

% \item 
% В алфавите племени Мумбо-Юмбо шесть букв. Словом является любая последовательность из шести букв, в которой есть хотя бы две одинаковые буквы. Какова вероятность получить слово выбирая буквы броском кубика?

\item 
Сколькими способами можно поставить на шахматную доску белую и черную ладьи так, чтобы они не били друг друга? Какова вероятность того, что они стоят в соседних по диагонали клетках?

\item 
Среди 25 студентов, из которых 15 девушек, разыгрываются четыре
билета, причем каждый может выиграть только один билет. Какова
вероятность того, что среди обладателей билета окажутся: а) 4 девушки;
б) 3 юноши и 1 девушка?

% {\item[] \centering\itshape Статистическое определение вероятности\par}
% \item 
% При стрельбе по мишени частота попаданий равна 0,75. Найти число
% попа\-даний при 40 выстрелах.
% \item 
% Частота нормального всхода семян 0,97. Из высеянных семян взошло 970.
% Сколько семян было высеяно?
% \item 
% На отрезке натурального ряда от 1 до 20 найти частоту простых чисел.


% \item 
% Быстро вращающийся диск разделен на чётное число равных секторов,
% по\-переменно окрашенных в белый и чёрный цвет. По диску произведён
% выстрел. Найти вероятность того, что пуля попадёт в один из белых
% секторов.
\item В квадрат с вершинами (0;0), (0;1), (1;1), (1;0) случайным образом
брошена точка. Найти вероятность того, что координаты этой точки
удовлетворяют неравенству $y > \frac{1}{2}x$.
\item Два человека $X$ и $Y$ условились встретиться в определенном месте между двумя и тремя часами дня. Пришедший первым ждет другого в течение 10 минут, после чего уходит. Чему равна вероятность встречи этих лиц, если каждый из них может прийти в любое время в течение указанного часа независимо от другого?
% Два парохода должны подойти к одному и тому же причалу. Время прихода
% обоих пароходов независимо и равновозможно в течение суток. Определить
% вероятность того, что одному из пароходов придётся ожидать освобождения
% причала, если время стоянки первого парохода равно 1 часу, а второго –
% 2 ча\-сам.
% \item 
% Случайным образом взяты два положительных числа
% $x$ и $y$, каждое из кото\-рых не превышает~1. Найти вероятность того, что сумма
% $x+y$ не превышает 1, а произведение $xy$ не меньше 0,99.
\end{enumerate}


\end{document}	