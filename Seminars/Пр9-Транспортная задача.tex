%encoding=utf8%
\documentclass[a5paper,11pt]{extarticle}
%\usepackage{pgfpages}
%\pgfpagesuselayout{2 on 1}[a4paper,border shrink=5mm,landscape]

% Русская кодировка
\usepackage[T2A]{fontenc}
\usepackage[utf8]{inputenc}
\usepackage[russian]{babel}
%\usepackage[pdftex]{graphicx}
%\usepackage[pdftex,colorlinks,urlcolor=black, linkcolor=black, citecolor=black]{hyperref}
%        \pdfcompresslevel=9 % сжимать PDF

% необходимые модули
\usepackage{amssymb,amsmath,amsthm}
\usepackage{tabularx,multirow,hhline}
\usepackage{indentfirst}
\usepackage{tikz}
\usepackage[margin=6mm]{geometry}
\usepackage[inline]{enumitem}
\usepackage{multicol}
\setlist{itemsep=0pt,leftmargin=\parindent}

\AddEnumerateCounter{\Asbuk}{\@Asbuk}{\CYRM}
\AddEnumerateCounter{\asbuk}{\@asbuk}{\cyrm}
\newlist{dotenumerate}{enumerate}{10}
\setlist[enumerate,1]{label=\arabic*., ref=\arabic*} %списки со скобками
\setlist[enumerate,2]{label=\asbuk*), ref=(\asbuk*)} %списки со скобками

\usepackage{environ}

\NewEnviron{LOOP}[1]{
  \newcounter{nclone}
  \setcounter{nclone}{#1} 
  \par
  \loop
    \BODY
  \addtocounter{nclone}{-1}
  \ifnum \value{nclone}>0 \repeat}

%Полуторный интервал
\renewcommand{\baselinestretch}{1.00}
\pagestyle{empty}
\newcolumntype{C}{>{\centering\arraybackslash}X}


\sloppy
\binoppenalty=10000
\relpenalty=10000
\begin{document}
\begin{enumerate}
  {\item[]\centering {\scriptsize Практическое занятие \textnumero~9--10 \par}
          \bfseries Транспортная задача и динамическое программирование.
          \par\vspace{1mm}
  }
%\item Решите транспортную задачу
%\begin{tabular}{c|cccc}
%	& 200 & 200 & 300 & 400\\
%\hline
%	200 & 4 & 3 & 2 & 1\\
%	300 & 2 & 3 & 5 & 6\\
%	500 & 6 & 7 & 9 & 12\\
%\end{tabular}

%\item Решите транспортную задачу
%\begin{tabular}{c|ccc}
%	& 60 & 60 & 50 \\
%\hline
%	50 & 2 & 3 & 2 \\
%	70 & 2 & 4 & 5 \\
%	60 & 6 & 5 & 7\\
%\end{tabular}

\item  Менеджер лесной компании должен решить как снабжать их три лесозавода древесиной, срубленной на трех лесосеках. 
Расстояния между лесозаводами и лесосеками приведены в табл. 
{\par\centering
	\begin{tabular}{c|ccc}
		Лесосека & Лесозавод 1&  Лесозавод 2 &  Лесозавод 3\\
		\hline
		1& 80& 150& 500\\
		2& 100& 170& 200\\
		3&  300&  250&  150\\
	\end{tabular}\par
}
Транспортные затраты на вывозку древесины лесовозами (одной модели)~--- 10\;руб. за км. 
Каждый завод требует непрерывного снабжения древесиной, причем минимальное ежедневное снабжение каждого из них~--- 25~лесовозов. 
Ежедневный максимальный объем вырубаемой древесины по лесосекам (в лесовозах) следующий: 
первая~--- 25; вторая~--- 30; третья~--- 25.
Требуется принять решение по количеству ежедневно отгружаемой древесины с лесосек к каждому лесозаводу с целью минимизации транспортных затрат.

\item 
	Решить прямую и двойственную задачи:
	$F = -2x_1 - x_2 + x_3 -2x_4  \to \max$,  при условиях 
	$
	\left\lbrace\begin{aligned}
		x_1 + 3x_2 - x_3 + 4x_4 &\leqslant 5\\
		-x_1 - x_2 + x_3 + 2x_4 &\leqslant 3
	\end{aligned}\right.
	$

\item 
Производственное объединение выделяет четырем входящим в него
предприятиям кредит в сумме 100 млн.\,ден. ед. для расширения производства и~увеличения выпуска продукции. 
По каждому предприятию известен возможный прирост $z_i(u_i)$ ( $i = \overline{1,4}$) выпуска продукции (в денежном выражении) в зависимости от выделенной ему суммы~$u_i$ . Выделяемые суммы кратны
20\;млн\,ден.\,ед. 
%При этом предполагаем, что прирост выпуска продукции
%на $i$-м предприятии не зависит от суммы средств, вложенных в другие предприятия, а общий прирост выпуска в производственном объединении равен сумме приростов, полученных на каждом предприятии объединения.

{\centering
\begin{tabular}{c|cccc}
$u_i$&$z_1(u_i)$&$z_2(u_i)$&$z_3(u_i)$&$z_4(u_i)$\\
\hline
20 & 10 & 12& 11& 16\\
40 & 31 & 26& 36& 37\\
60 & 42 & 36& 45& 46\\
80 & 62 & 54& 60& 63\\
100& 76 & 78& 77& 80\\
\end{tabular}\par}
\medskip
\hrule
\medskip
{\item[] \bfseries Домашнее задание\par}
По пособию Ряттель А.\,В. Основы экономико-математического моделирования.
\item \textit{Транспортная задача}: Задание 1. Стр. 25.
\item \textit{Динамическое программирование}: Задания 1,2 стр. 27.







 





 








%\item[2.1.] Завод имеет три цеха $A$, $B$ и $C$ и пять складов \textnumero~1, 2, 3, 4, 5. Цех $А$ производит 200 тыс. шт. изделий, цех $В$~--- 205~тыс.~шт., цех $C$~--- 225~тыс.~шт. Пропускная способность складов за то же время характеризуется следующими показателями; склад \textnumero~1~--- 190~тыс.~шт.,
%склад \textnumero~2~--- 130~тыс.~шт., склад \textnumero~3~--- 80~тыс.~шт., склад \textnumero~4~--- 100~тыс.~шт., склад~\textnumero~5~--- 130~тыс.~шт. Стоимости перевозки 1~тыс.~шт. изделий из~цеха~$А$ в~склады No~1, 2, 3, 4, 5 соответственно равны~5, 7, 4, 9, 5~ден.~ед., из цеха $В$~--- 7, 4, 3, 4, 7 ден.~ед., а~из цеха $С$~– 9, 10, 6, 8, 7. Составьте такой план перевозки изделий, при котором расходы на перевозку изделий были бы наименьшими.

%\item[2.2.] На трех складах А, В, С находится сортовое зерно соответственно 200, 300 и 100 т, которое надо доставить в четыре пункта: №~1~--- 150~т, №~2~--- 150~т, №~3~--- 250~т и №~4~--- 50~т.
%Стоимости доставки 1~т со склада~А в указанные пункты соответственно равны 5, 4, 6, 3
%ден. ед., со склада В~--- 1, 10, 2, 1 ден. ед. и со склада С~--- 2, 3, 3, 1 ден. ед. составьте опти-
%мальный план перевозки зерна в четыре пункта, минимизирующий стоимость перевозок.


%\item На предприятии имеется три группы станков, каждая из которых может выполнять пять операций по обработке деталей. 
%Максимальное время работы каждой группы станков равно 100, 250, и 180 час. соответственно. 
%Время выполнения каждой операции составляет 100, 120, 70, 110 и~130 час. соответственно. 
%Определить, сколько времени и на какой операции нужно использовать каждую группу станков, чтобы обработать максимальное количество деталей. Производительность каждой группы станков на каждой операции заданы матрицей:
%$$\begin{pmatrix}
%3 & 5 & 11 & 10 & 5\\
%5 & 10 & 15 & 3 & 2\\
%4 & 8 & 6 & 12 & 10\\
%\end{pmatrix}
%$$


%\item Существуют 4 базы $A_1$,$A_2$,$A_3$,$A_4$ и~4 торговые точки $B_1$, $B_2$, $B_3$, $B_4$. Расстояния от баз до торговых точек заданы матрицей:
%$$
%\begin{pmatrix}
%10 & 20 & 12 & 5\\
%3 & 14 & 9 & 1\\
%13 & 8 & 6 & 9\\
%7 & 15 & 8 & 10\\
%\end{pmatrix}
%$$
%Нужно так прикрепить базы к~торговым точкам, чтобы суммарное расстояние было минимальным.

%\item 
%Мастер должен назначить на 4 типовых операций 4 рабочих. Время, которое затрачивают рабочие на выполнение каждой операции, приведено в таблице.
%{\centering
%\begin{tabular}{c|cccc}
% & \multicolumn{4}{c}{Операции}\\
%Рабочие & $O_1$ & $O_2$ & $O_3$ & $O_4$ \\ 
%\hline
%$P_1$ & 7 & 7 & 3 & 6 \\
%$P_2$ & 4 & 9 & 5 & 4 \\ 
%$P_3$ & 5 & 5 & 4 & 5 \\ 
%$P_4$ & 6 & 4 & 7 & 2 \\ 
%\end{tabular}
%\par}
%
%Определите расстановку рабочих по операциям, при которой суммарное время на выполнение всех работ будет наименьшим. 
\end{enumerate}
\end{document}	














\lint $
\begin{aligned}
 &F=2x_{1}+x_{2} \to \max \\
 &x_{1} \geqslant 0, x_{2} \geqslant 0\\
 &\left\lbrace
  \begin{aligned}
    x_{1}+3x_{2} &\leqslant 21\\
    3x_{1}+2x_{2} &\leqslant 21\\
    x_{1} &\leqslant 5\\
  \end{aligned} 
  \right.
\end{aligned}$ 
\hfil
\lint  $
\begin{aligned}
   &F=4x_{1}+6x_{2} \to \min;\\ 
  &x_{1} \geqslant 0,  x_{2} \geqslant 0;\\
  &\left\lbrace\begin{aligned}
    3x_{1}+ x_{2} &\geqslant 9;\\
    x_{1}+ 2x_{2} &\geqslant 8;\\
    x_{1}+6x_{2} &\geqslant 12;\\
  \end{aligned} \right.
\end{aligned}$\hfil\\[5mm]
\lint  $
\begin{aligned}
   &F=2x_{1}-3x_{2} \to \min;\\ 
  &x_{1} \geqslant 0,  x_{2} \geqslant 0;\\
  &\left\lbrace\begin{aligned}
    x_{1}+ x_{2} &\geqslant 4;\\
    -x_{1}+ 2x_{2} &\leqslant 2;\\
    x_{1}+2x_{2} &\leqslant 10;
  \end{aligned} \right.
\end{aligned}$\hfil
\lint  $
\begin{aligned}
   &F=2x_{1}-6x_{2} \to \max;\\ 
  &x_{1} \geqslant 0,  x_{2} \geqslant 0;\\
  &\left\lbrace\begin{aligned}
    x_{1}+ x_{2} &\geqslant 2;\\
    -x_{1}+ 2x_{2} &\leqslant 4;\\
    x_{1}+2x_{2} &\leqslant 8;
  \end{aligned} \right.
\end{aligned}$\hfil\\[5mm]
\lint  $
\begin{aligned}
   &F=x_{1}+x_{2} \to \max;\\ 
  &x_{1} \geqslant 0,  x_{2} \geqslant 0;\\
  &\left\lbrace\begin{aligned}
    x_{1}- 4x_{2} &\leqslant 0;\\
    3x_{1}- x_{2} &\geqslant 0;\\
    x_{1}+x_{2}-4 &\geqslant 0;
  \end{aligned} \right.
\end{aligned}$\hfil
\lint  $
\begin{aligned}
   &F=x_{1}-x_{2} \to \max;\\ 
  &x_{1} \geqslant 0,  x_{2} \geqslant 0;\\
  &\left\lbrace\begin{aligned}
    -2x_{1}+ x_{2} &\leqslant 2;\\
    x_{1}- 2x_{2} &\leqslant -8;\\
    x_{1}+x_{2} &\leqslant 5;
  \end{aligned} \right.
\end{aligned}$\hfil\\[5mm]
\lint Составить двойственные задачи к каждому из заданий 1--4.\\
\lint 
