%encoding=utf8%
\documentclass[a4paper,14pt]{extarticle}
% \usepackage{pgfpages}

% Русская кодировка
\usepackage[T2A]{fontenc}
\usepackage[utf8]{inputenc}
\usepackage[russian]{babel}
%\usepackage[pdftex]{graphicx}
%\usepackage[pdftex,colorlinks,urlcolor=black, linkcolor=black, citecolor=black]{hyperref}
%        \pdfcompresslevel=9 % сжимать PDF
% необходимые модули
\usepackage{amssymb,amsmath,amsthm}
\usepackage{tabularx,multirow,hhline}
\usepackage{indentfirst}
\usepackage{tikz}
\usepackage[margin=8mm]{geometry}
\usepackage[inline]{enumitem}
\usepackage{multicol}
\setlist{nosep,leftmargin=\parindent}

\AddEnumerateCounter{\Asbuk}{\@Asbuk}{\CYRM}
\AddEnumerateCounter{\asbuk}{\@asbuk}{\cyrm}
\newlist{dotenumerate}{enumerate}{10}
\setlist[enumerate,1]{label=\arabic*., ref=\arabic*} %списки со скобками
\setlist[enumerate,2]{label=\asbuk*), ref=(\asbuk*)} %списки со скобками

\usepackage{environ}

\NewEnviron{LOOP}[1]{
  \newcounter{nclone}
  \setcounter{nclone}{#1} 
  \par
  \loop
    \BODY
  \addtocounter{nclone}{-1}
  \ifnum \value{nclone}>0 \repeat}

%Полуторный интервал
\renewcommand{\baselinestretch}{1.00}
\pagestyle{empty}
\newcolumntype{C}{>{\centering\arraybackslash}X}


\sloppy
\binoppenalty=10000
\relpenalty=10000
\begin{document}
\begin{enumerate}
  {\item[]\centering {\footnotesize Практическое занятие \textnumero~28 \par}
          \bfseries Биматричные игры
          \par\vspace{1mm}
  }

\item 
Предприятие должно определить ассортимент сезонной продукции. Точная величина спроса на продукцию неизвестна, но ожидается, что она может принять одно из пяти возможных значений:  $P_1$--$P_5$ в зависимости от погоды. Предприятие выбрало пять стратегий производства обеспечивающих  наилучший уровень предложения в данных условиях. Данная ситуация представлена в виде платежной матрицы игры.
$$
\begin{array}{c|ccccc}
    & P_1 & P_2 &P_3 &P_4 &P_5 \\
\hline
A_1 & 48  & 9   & 15 & 87 & 6  \\     
A_2 & 7   & 48  & 61 & 37 & 85 \\    
A_3 & 42  & 78  & 10 & 95 & 66 \\     
A_4 & 79  & 87  & 97 & 49 & 75 \\     
A_5 & 45  & 5   & 31 & 58 & 64 \\   
\end{array}
$$
Требуется определить наилучшую стратегию поведения предприятия на рынке:
\begin{enumerate*}
    \item Пользуясь критерием Вальда;
    \item Пользуясь критерием Сэвиджа
    \item считая, что ранее проведенные маркетинговые исследования  определили вероятности возникновения ситуаций  $P_1$--$P_5$: $q_1 =0.15$, $q_2 =0.2$, $q_3 =0.35$, $q_4 =0.25$, $q_5 =0.05$.
\end{enumerate*}


\item Найдите все равновесия по Нэшу в игре
$$
\begin{array}{c|cc}
     & s & c \\
     \hline
    s & (1; 1) & (-2; -1) \\
    c & (-1; -2) & (-1; -1) \\
\end{array}
$$

\item (Всеобщность знания) В городе $N$ серы дома и хмуры... Там  живут игроки двух типов: «безумцы» и рациональные. При встрече в городе $N$ принято играть в игру из предыдущей задачи.
Рациональные игроки играют стратегию, приносящую наибольший платеж, а безумцы --- стратегию $c$.
Как-то случилось Пете попасть в этот город и встретится с одним рациональным аборигеном. Они никогда раньше не виделись и никогда больше не увидятся. Петя знает, что абориген --- рационален.
Абориген знает, что Петя --- рационален. Петя ошибочно полагает, что абориген считает его безумцем. Абориген знает о Петиной ошибке.
Какое из равновесий по Нэшу будет сыграно?

\item
Найти ситуации равновесные по Нэшу для следующих биматричных игр (вспомните, что в равновесных стратегиях игроки злонамеренны):
$$
\begin{array}{c|cc}
    \text{\bfseries a)} & B_1 & B_2 \\
     \hline
    A_1 & (-7; -7) & (1; -9) \\
    A_2 & (-9; 1) & (0; 0) \\
\end{array}\qquad
\begin{array}{c|ccc}
    \text{\bfseries б)}& B_1    & B_2    & B_3\\
    \hline
A_1 & (2; 8) & (1; 4) & (9; 20) \\
A_2 & (7; 7) & (6; 8) & (2; 4) \\
\end{array}
$$


	    
\item Двум воронам как-то бог послал по кусочку сыра. На ель вороны взгромоздясь одновременно выбирают: либо позавтракать своим кусочком, либо провести стремительное нападение на соседку и, похитив ее кусочек сыра, позавтракать в другом месте...
$$
\begin{array}{c|cc}
     & \text{blitz krieg} & \text{breakfast} \\
     \hline
    \text{blitz krieg} & (-a; -a) & (2; 0) \\
    \text{breakfast} & (0; 2) & (1; 1) \\
\end{array}
$$
\begin{enumerate*}
    \item Найдите все равновесия по Нэшу в этой игре;
    \item Как зависит от параметра~$a$ вероятности блицкрига и завтрака в смешанном равновесии?
\end{enumerate*}

% \item Найдите все равновесные по Нэшу смешанные стратегии (вспомните, что в равновесных стратегиях игроки злонамеренны).
% б) Какая стратегия является оптимальным ответом второго игрока на стратегию $l_2$ ?
% в) Какая стратегия является оптимальным ответом второго игрока на стратегию $\frac{1}{3}l_1 + \frac{2}{3} l_2$ ?
% \item 
% На столе лежат 6 спичек. Игроки берут спички по очереди, каждый может взять 1 или 2 спички. Тот, кто берет последнюю спичку, проирывает 1 очко. Построить граф игры. Определить оптимальные стратегии игроков.

\end{enumerate}
\end{document}	














\noindent\begin{enumerate*}
  \item $
  \begin{pmatrix}
  1.5 & 3\\
  2 & 1\\
  \end{pmatrix}
  $;
  \item $
  \begin{pmatrix}
  -1 & 1 \\
  1 & -1\\
  \end{pmatrix}
  $;
  \item $
  \begin{pmatrix}
  -1 & 1 & -1 & 2\\
  0 & -1 & 2 & -2\\
  \end{pmatrix}
  $;
  \item $
  \begin{pmatrix}
  1 & 4 \\
  3 & -2\\
  0 & 5\\
  \end{pmatrix}
  $.
  \end{enumerate*}
