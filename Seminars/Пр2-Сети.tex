%encoding=utf8%
\documentclass[a4paper,14pt]{extarticle}

% Русская кодировка
\usepackage[T2A]{fontenc}
\usepackage[utf8]{inputenc}
\usepackage[russian]{babel}

\usepackage[pdftex]{graphicx}
\usepackage[pdftex,colorlinks,urlcolor=blue, citecolor=magenta]{hyperref}
%        \pdfcompresslevel=9 % сжимать PDF

% необходимые модули
\usepackage{amssymb,amsmath,amsthm}
\usepackage{indentfirst}
\usepackage{tikz}
\usepackage[margin=9mm]{geometry}
\usepackage[inline]{enumitem}
\setlist{noitemsep,leftmargin=\parindent}

\AddEnumerateCounter{\Asbuk}{\@Asbuk}{\CYRM}
\AddEnumerateCounter{\asbuk}{\@asbuk}{\cyrm}
\newlist{dotenumerate}{enumerate}{10}
\setlist[enumerate,1]{label=\arabic*., ref=\arabic*} %списки со скобками
\setlist[enumerate,2]{label=\asbuk*), ref=(\asbuk*)} %списки со скобками

\usepackage{tabularx}
\newcolumntype{C}{>{\centering\arraybackslash}X}


%Полуторный интервал
\renewcommand{\baselinestretch}{1.00}
\pagestyle{empty}
\begin{document}
{\centering \small Практическое занятие \textnumero~2 \par
\bfseries \large Сетевые модели \par}

\begin{enumerate}
\item Упорядочить сетевой граф:

{\centering
\begin{tikzpicture}[y=7.5mm, x=8mm,above,sloped]
\node[draw,circle] (0) at (0,0) {0};
\node[draw,circle] (1) at (2,2) {1};
\node[draw,circle] (2) at (2,0) {2};
\node[draw,circle] (3) at (2,-2) {3};
\node[draw,circle] (4) at (5,2) {4};
\node[draw,circle] (5) at (5,0) {5};
\node[draw,circle] (6) at (5,-2) {6};
\node[draw,circle] (7) at (7,-1) {7};
\node[draw,circle] (8) at (7, 1) {8};
\node[draw,circle] (10) at (7,-3) {10};
\node[draw,circle] (9) at (9,0) {9};
\node[draw,circle] (11) at (11,0) {11};
\draw[-stealth] (0) -- (1);
\draw[-stealth] (0) -- (3);
\draw[-stealth] (1) -- (2);
\draw[-stealth] (1) -- (4);
\draw[-stealth] (1) -- (5);
\draw[-stealth] (2) -- (3);
\draw[-stealth] (2) -- (5);
\draw[-stealth] (2) -- (7);
\draw[-stealth] (3) -- (7);
\draw[-stealth] (3) -- (6);
\draw[-stealth] (3) -- (10);
\draw[-stealth] (4) -- (8);
\draw[-stealth] (5) -- (7);
\draw[-stealth] (5) -- (8);
\draw[-stealth] (6) -- (10);
\draw[-stealth] (7) -- (6);
\draw[-stealth] (7) -- (8);
\draw[-stealth] (7) -- (9);
\draw[-stealth] (7) -- (10);
\draw[-stealth] (8) -- (9);
\draw[-stealth] (9) -- (11);
\draw[-stealth] (10) -- (9);
\draw[-stealth] (10) -- (11);

\end{tikzpicture}
\par}

\item Для сетевой модели определить следующие характеристики:

{\centering
\begin{tikzpicture}[y=9mm, x=12mm,above,sloped]
\node[draw,circle] (1) at (0,0) {1};
\node[draw,circle] (2) at (2,-2) {2};
\node[draw,circle] (3) at (2, 2) {3};
\node[draw,circle] (4) at (4, 2) {4};
\node[draw,circle] (5) at (4,-2) {5};
\node[draw,circle] (6) at (4, 0) {6};
\node[draw,circle] (7) at (6, 2) {7};
\node[draw,circle] (8) at (6, -2) {8};
\node[draw,circle] (9) at (8, 0) {9};
\draw[-stealth] (1) -- node{11} (2);
\draw[-stealth] (1) -- node{15} (3);
\draw[-stealth] (1) -- node{9} (4);
\draw[-stealth] (1) -- node{13} (6);
\draw[-stealth] (2) -- node{12} (5);
\draw[-stealth] (2) -- node{12} (6);
\draw[-stealth] (3) -- node{11} (4);
\draw[-stealth] (3) -- node{5} (6);
\draw[-stealth] (4) -- node{14} (7);
\draw[-stealth] (5) -- node{10} (8);
\draw[-stealth] (6) -- node{14} (8);
\draw[-stealth] (6) -- node{13} (9);
\draw[-stealth] (7) -- node{11} (9);
\draw[-stealth] (8) -- node{10} (9);
\end{tikzpicture}
\par}
\begin{enumerate}
\item ранние и поздние сроки совершения событий;
\item резервы времени событий;
\item критический путь;
\item для некритических работ найти полные и свободные резервы времени. 
\end{enumerate}
На основе проведенных расчетов установить, как повлияет на срок выполнения работ и полный резерв времени работы (6,8) увеличение продолжительности работы (6,9), например, на 4 ед
\item 

Данные по основным операциям проекта представлены в следующей таблице:

{\centering
\begin{tabular}{ccc}
\hline
Работа & Необходимо выполнение & Длительность, недели\\
\hline
$A$ & ---	& 4 \\
$B$ & ---	& 6 \\
$C$ & $A,B$ & 7 \\
$D$ & $B$	& 3 \\
$E$ & $C$	& 4 \\
$F$ & $D$	& 5 \\
$G$ & $E,F$ & 3 \\
\hline
\end{tabular}
\par}

Постройте сетевую модель проекта, определите критические пути модели и проанализируйте, как влияет на ход выполнения проекта задержка работы $D$ на 4 недели.















%Для улучшения финансового состояния фирме необходимо увеличить спрос на выпускаемый
%цемент марки М400 и расширить потребительский рынок. Фирма считает целесообразным размещать
%цемент в специализированной таре. Для переоснащения цеха необходимо установить оборудование по
%производству специализированной тары. Предполагается выполнить следующее:
%
%
%%\begin{tabular}{|cp{8cm}|cc|cc|}
%%%\hline
%%%	№ п.\,п. & Yfbvtyjdf
%%\hline
%%	1 & Подготовка и выпуск технического задания на переоборудование цеха 
%%	  & 20 & 20 & 18 & 26 \\
%%	2 & Разработка мероприятий по технике безопасности						
%%	  & 25 & 30 & 20 & 37 \\
%%	3 & Подбор кадров
%%	  & 10 & 5 & 9 & 7 \\
%%	4 & Заказ и поставка необходимого оборудования
%%	  & 30 & 60 & 23 & 64 \\ 
%%	5 & Заказ и поставка электрооборудования
%%	  & 40 & 65 & 32 & 78 \\
%%	6 & Установка оборудования
%%	  & 50 & 90 & 43 & 100 \\
%%	7 & Установка электрооборудования
%%	  & 45 & 80 & 41 & 85 \\
%%	8 & Обучение персонала
%%	  & 15 & 5 & 9 & 10 \\
%%	9 & Испытание и сдачу в эксплуатацию линии	
%%	  & 25 & 50 & 21 & 57 \\
%%\hline
%%\end{tabular}
%%Ожидается, что производительность вводимой линии по производству тары составит 1000
%%мешков в день при односменном режиме работы. 
%%Стоимость 1 мешка~--- 25 р., 
%%выручка от реализации тары в смену составит 25 тыс. р., 
%%из которых чистая прибыль фирмы равна 50 тыс. р. 
%%Деньги на покупку оборудования и переоснащение цеха в размере 5 500 тыс. р. взяты в банке под 30% годовых из расчета
%%5000 тыс. р. на оборудование и 500 тыс. р. на его установку.
%%Затраты на проведение работ и их продолжительность в нормальном и максимальном режимах
%%указаны в табл. 30.8


\item \textbf{На дом:} задания 2,3 практического занятия 2 из пособия Ряттель А.\,В. Основы экономико-математического
моделирования стр. 13. Ищите здесь: \url{https://cloud.mail.ru/public/3soD/y91HFQd3H}

\end{enumerate}
\end{document}
