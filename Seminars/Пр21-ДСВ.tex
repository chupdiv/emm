%encoding=utf8%
\documentclass[a4paper,14pt]{extarticle}
% \usepackage{pgfpages}
% \pgfpagesuselayout{2 on 1}[a4paper,border shrink=0mm,landscape]

% Русская кодировка
\usepackage[T2A]{fontenc}
\usepackage[utf8]{inputenc}
\usepackage[russian]{babel}


% необходимые модули
\usepackage{amssymb,amsmath,amsthm}
\usepackage{tabularx,multirow,hhline}
\usepackage{indentfirst}
\usepackage{tikz}
\usepackage[margin=20mm]{geometry}
\usepackage{paralist}
\usepackage[inline]{enumitem}
\usepackage{multicol}
\setlist{noitemsep,leftmargin=\parindent}

\AddEnumerateCounter{\Asbuk}{\@Asbuk}{\CYRM}
\AddEnumerateCounter{\asbuk}{\@asbuk}{\cyrm}
\newlist{dotenumerate}{enumerate}{10}
\newlist{denumerate}{enumerate}{1}
\setlist[enumerate,1]{itemsep=5pt,label=\arabic*., ref=\arabic*} %списки со скобками
\setlist[denumerate,1]{noitemsep,label={Д.\,\arabic*.}, ref=\arabic*} %списки со скобками

\usepackage{environ}

\NewEnviron{LOOP}[1]{
  \newcounter{nclone}
  \setcounter{nclone}{#1} 
  \par
  \loop
    \BODY
  \addtocounter{nclone}{-1}
  \ifnum \value{nclone}>0 \repeat}

%Полуторный интервал
\renewcommand{\baselinestretch}{1.00}
\pagestyle{empty}
\newcolumntype{C}{>{\centering\arraybackslash}X}


\sloppy
\binoppenalty=10000
\relpenalty=10000

\newcommand{\bitem}{\textbf{\item}}
\begin{document}
{\centering 
{\centering {\scriptsize Практическое занятие \textnumero~21 \par}
\bfseries Дискретные случайные величины \par}
\begin{enumerate}
\item  По мишени производится 4 независимых выстрела с вероятностью попадания при каждом выстреле 0,8. $X$~--- число попаданий. Требуется найти: А) Найти закон распределения вероятностей~$X$. Б) Построить многоугольник распределения. В) Найти вероятность события $x\geqslant 1$. Г) Найти числовые характеристики $X$, моду.
\item В урне 7 шаров, из которых 4 белых, а остальные чёрные. Из этой урны наудачу извлекаются 3 шара. $X$~--- число белых шаров Требуется найти: А) Найти закон распределения вероятностей~$X$. Б) Построить многоугольник распределения. В) Найти вероятность события $1\leqslant x\leqslant 3$. Г) Найти числовые характеристики $X$.
%\item Учебник издан тиражом $100\;000$ экземпляров. Вероятность того, что учебник сброшюрован неправильно, равна $0,0001$. Требуется найти закон распределения Д.С.В. X – число бракованных книг.
\item Два стрелка делают по одному выстрелу в одну мишень. Вероятность попадания для первого стрелка при одном выстреле~--- $0.5$, для второго~--- $0.4$. Дискретная случайная величина $X$~--- число попаданий в мишень. А) Найти закон распределения вероятностей $X$. Б) Построить многоугольник распределения. В) Найти вероятность события $x\geqslant 1$. Г) Найти числовые характеристики.
\item ДСВ $X$ принимает три значения: $x_1=4$ c~вероятностью $p_1=0.5$; $x_2=6$ c вероятностью $p_2=0.3$. Найти $x_3$ и~$p_3$, зная, что $M(X)=8$.
\item  Найти дисперсию Д.С.В. $X$~---числа появлений события А в двух независимых испытаниях, если вероятности появления события в этих испытаниях одинаковы и известно, что $M(X)=1.2$.
% {\item[] \centering \bfseries Домашнее задание\par}
\item Прибор состоит из 5 независимо работающих элементов. Вероятность отказа элемента в момент включения прибора равна $0.2$. Найти: А) Найти закон распределения вероятностей X – число отказавших элементов. Б) Построить многоугольник распределения. В) Найти вероятность события $x\geqslant 2$. Г) Найти числовые характеристики $X$.
%\item Магазин получил 1000 бутылок минеральной воды. Вероятность того, что при перевозке бутылка окажется разбитой, равна 0,003. Требуется найти закон распределения Д.С.В. $X$~--- число разбитых бутылок.
\item  В партии из 6 деталей имеется 4 стандартных. Наудачу отобраны 3 детали. Требуется найти: а) закон распределения ДСВ. $X$~--- число стандартных среди отобранных; б) найти вероятности событий: $1\leqslant x\leqslant 3$, $x>3$; в) найти числовые характеристики~$X$; 
\item  ДСВ~$X$ имеет только два возможных значения: $x_1$ и $x_2$, причем $x_2 >x_1$. Вероятность того, что $X$ примет значение $x_1$, равна $0.2$. Найти закон распределения величины~$X$, если $M(X)=2.6$, $D(X)=0.8$.
\end{enumerate}
\newpage
\end{document}
