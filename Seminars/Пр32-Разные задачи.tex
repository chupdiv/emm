%encoding=utf8%
\documentclass[a4paper,14pt]{extarticle}
% \usepackage{pgfpages}

% Русская кодировка
\usepackage[T2A]{fontenc}
\usepackage[utf8]{inputenc}
\usepackage[russian]{babel}
%\usepackage[pdftex]{graphicx}
%\usepackage[pdftex,colorlinks,urlcolor=black, linkcolor=black, citecolor=black]{hyperref}
%        \pdfcompresslevel=9 % сжимать PDF
% необходимые модули
\usepackage{amssymb,amsmath,amsthm}
\usepackage{tabularx,multirow,hhline}
\usepackage{indentfirst}
\usepackage{tikz}
\usepackage{icomma}
\usepackage[margin=8mm]{geometry}
\usepackage[inline]{enumitem}
\usepackage{multicol}
\setlist{nosep,leftmargin=\parindent}

\AddEnumerateCounter{\Asbuk}{\@Asbuk}{\CYRM}
\AddEnumerateCounter{\asbuk}{\@asbuk}{\cyrm}
\newlist{dotenumerate}{enumerate}{10}
\setlist[enumerate,1]{label=\arabic*., ref=\arabic*} %списки со скобками
\setlist[enumerate,2]{label=\asbuk*), ref=(\asbuk*)} %списки со скобками

\usepackage{environ}

\NewEnviron{LOOP}[1]{
  \newcounter{nclone}
  \setcounter{nclone}{#1} 
  \par
  \loop
    \BODY
  \addtocounter{nclone}{-1}
  \ifnum \value{nclone}>0 \repeat}

%Полуторный интервал
\renewcommand{\baselinestretch}{1.00}
\pagestyle{empty}
\newcolumntype{C}{>{\centering\arraybackslash}X}


\sloppy
\binoppenalty=10000
\relpenalty=10000

\begin{document}
{\centering {\footnotesize Практическое занятие \textnumero~32 \par}
\bfseries   Задачи по второй части курса экономико-математического моделирования
\par\vspace{1mm}
}


\begin{enumerate}
    \item  На овцеводческой ферме из стада 1000 овец произведена бесповторная выборка для взвешивания 36 овец. Их средний вес оказался равным 50 кг, выборочная дисперсия 16. Найти доверительный интервал для оценки среднего веса овцы на всей ферме с надежностью 0,8.

    \item Фирма предлагает автоматы по розливу напитков. При выборке 16 найдена средняя величина 182 г. дозы, наливаемой в стакан автоматом №1. При выборке 9 найдена величина 185 г дозы, наливаемой в стакан автоматом №2. По утверждению изготовителя, случайная величина наливаемой дозы имеет нормальный закон распределения с дисперсией, равной 25 г$^2$ для каждого автомата. Можно ли считать отличия выборочных средних случайной ошибкой при уровне значимости 0,01 (при двусторонней альтернативной гипотезе)?
    
    
    \item  Имеются две выборки значений показателя качества однотипной продукции, изготовленной двумя фирмами. Можно ли на уровне значимости 0,05 считать, что рассматриваемый показатель качества продукции двух фирм описывается одной и той же функцией распределения?
    Интервалы показателя качества

    {\centering
    \begin{tabular}{c|cccc}
        Частоты & 0--10& 10--20   &20--30 & 30--40 \\
        \hline
        1 фирма &  2 & 5 & 2 & 1 \\
        2 фирма & 1 & 4 & 4 & 1  \\
    \end{tabular}\par}

    \item Тюремный покер. Участники одновременно показывают один или два пальца. Потом считают сумму $s$ (она может получиться от двух до четырех). 
    Если $s$ четно, то второй игрок выиграл у первого $s$ долларов, если же $s$ нечетно то есть $s = 3$, то наоборот, $s$ долларов выиграл
    первый. Найдите седловую точку в смешанных стратегиях и цену игры. Справедлива ли игра, и, если нет, то «кому лучше играется», первому и второму?

    \item Решите игру с матрицей выигрышей
    $$\begin{pmatrix}
        5, 2 & 2, 6 & 1, 4 & 0, 3 \\
        4, 1 & 3, 4 & 2, 1 & 1, 2 \\
        1, 0 & 1, 1 & 1, 5 & 5, 1 \\
        2, 3 & 0, 1 & 0, 2 & 4, 4 \\
    \end{pmatrix}
    $$

    \item
    Джон и Бэтти нужно поделить виллу, яхту, неделимые акции и машину. Они условились на следующей процедуре дележа: каждый забирает себе один предмет согласно некоторой очерёдности. 
    Предпочтения выглядят так (в порядке убывания ценности):
    \begin{description}
        \item[Бэтти:] яхта, вилла, акции, машина. 
        \item[Джон:] вилла, акции, машина, яхта.
    \end{description}
    Решите игру, если очередность:Бэтти--Джон--Бэтти--Джон.

    \item Найти выборочное уравнение линейной регрессии $Y$ на $X$ на основании корреляционной таблицы:
    \label{zd}
    
    {\centering
    \begin{tabular}{c|ccc}
        $y_j$ \textbackslash\ $x_i$& 0--1 & 1--2 & 2--3 \\
    \hline
        1--2  & 2    & 1    &      \\
        2--3  & 4    & 5    &      \\
        3--4  &      & 2    & 1    \\
    \end{tabular}\par}

    \item По данным задачи \ref{zd} найти выборочный коэффициент корреляции и выяснить, значим ли он на уровне 0,05.    
\end{enumerate}
    
    

    
    
    
    
    
    
    
    
    
    
    
    
    
    
    

\end{document}	
