%encoding=utf8%
\documentclass[a5paper,14pt]{extarticle}

% Русская кодировка
\usepackage[T2A]{fontenc}
\usepackage[utf8]{inputenc}
\usepackage[russian]{babel}

\usepackage[pdftex]{graphicx}
\usepackage[pdftex,colorlinks,urlcolor=black, linkcolor=black, citecolor=black]{hyperref}
%        \pdfcompresslevel=9 % сжимать PDF

% необходимые модули
\usepackage{amssymb,amsmath,amsthm}
\usepackage{tabularx,multirow,hhline}
\usepackage{indentfirst}
\usepackage{tikz}
\usepackage[margin=9mm]{geometry}
\usepackage[inline]{enumitem}
\usepackage{multicol}
\setlist{itemsep=0pt,leftmargin=\parindent}

\AddEnumerateCounter{\Asbuk}{\@Asbuk}{\CYRM}
\AddEnumerateCounter{\asbuk}{\@asbuk}{\cyrm}
\newlist{dotenumerate}{enumerate}{10}
\setlist[enumerate,1]{label=\arabic*., ref=\arabic*} %списки со скобками
\setlist[enumerate,2]{label=\asbuk*), ref=(\asbuk*)} %списки со скобками

\usepackage{environ}

\NewEnviron{LOOP}[1]{
  \newcounter{nclone}
  \setcounter{nclone}{#1} 
  \par
  \loop
    \BODY
  \addtocounter{nclone}{-1}
  \ifnum \value{nclone}>0 \repeat}

%Полуторный интервал
\renewcommand{\baselinestretch}{1.00}
\pagestyle{empty}
\newcolumntype{C}{>{\centering\arraybackslash}X}


\sloppy
\binoppenalty=10000
\relpenalty=10000
\begin{document}
\begin{enumerate}
  {\item[]\centering {\scriptsize Практическое занятие \textnumero~7 \par}
          \bfseries Симплекс-метод. Двойственная задача
          \par\vspace{1mm}
  }
  
%  \item 
%Решить симплекс-методом:\\
%\textbf{а)}~$F  = -3x_1 + x_2 - x_3 - 2x_4  \to \max$\quad $
%\left\lbrace\begin{aligned}
%  2x_1 + x_2 + 3x_3 + &x_4 \leqslant 5,\\
%  2x_1 - x_2 - x_3 + &2x_4 \leqslant 1;
%\end{aligned}\right.
%$ 

% \textbf{б)}~$F = -x_1 +x_2 + 3x_3 +x_4  \to \min$,\quad $
% \left\lbrace\begin{aligned}
% -x_1 + x_2 - x_3 + 2x_4 &\leqslant 6,\\
% 2x_1 + x_2 + 2x_3 - x_4 &\leqslant 4.
% \end{aligned}\right.
% $

\item Решить задачу линейного программирования:\\
    $F = -4x_1 -18x_2 - 30x_3 -5x_4  \to \max$,
    $\quad
    \left\lbrace\begin{aligned}
    3x_1 + x_2 - 4x_3 - x_4 & \leqslant -3,\\
    2x_1 + 4x_2 + x_3 - x_4 &\geqslant 4.
    \end{aligned}\right.
    $


\item Решить прямую и двойственную задачи:\\
$F(X)= -x_1 + x_2 + x_3 \to \min$,\qquad

$
\left\lbrace
\begin{aligned}
x_1-x_2-x_3 \geqslant 1\\
-2x_1+3x_2 \geqslant 1\\
-3x_1+4x_2-2x_3 \leqslant 1\\
\end{aligned}
\right.
$

% \item 
%Для изготовления четырех видов продукции используют три вида сырья.
%Запасы сырья, нормы его расхода и цены реализации единицы каждого вида продукции
%приведены в таблице.
%
%{\centering
%\begin{tabular}{cccccc}
%\hline
%Тип сырья & \multicolumn{4}{p{4.5cm}}{\centeringНормы расхода сырья на одно изделие} & Запасы сырья\\
%	& A & Б	& В	& Г & 	\\
%\hline
%I 	& 1 & 2	& 1	& 0	& 18\\
%II 	& 1 & 1	& 2	& 1 & 30\\
%III & 1	& 3	& 3 & 2	& 40 \\
%\hline
%Цена изделия & 12 & 7 & 18 & 10 & \\
%\hline
%\end{tabular}
%\par}
%Требуется:
%\begin{enumerate}
%\item 
%Сформулировать прямую оптимизационную задачу на максимум выручки от
%реализации готовой продукции, получить оптимальный план выпуска продукции.
%\item 
%	Сформулировать двойственную задачу и найти ее оптимальный план.
%\item 
%	Пояснить нулевые значения переменных в оптимальном плане.
%\item 
%	На основе свойств двойственных оценок и теорем двойственности:
%	\begin{itemize}
%	\item 
%		проанализировать использование ресурсов в оптимальном плане исходной задачи;
%	\item 
%		определить, как изменятся выручка и план выпуска продукции при увеличении запасов сырья I и II видов на 4 и 3 единицы соответственно и уменьшении на 3~единицы сырья III вида.
%%	\item 
%%		оценить целесообразность включения в план изделия Д ценой 10 ед., на изготовление которого расходуется по две единицы каждого вида сырья.
%\end{itemize}
%\end{enumerate}

\item[]\textbf{Домашнее задание}
\item Изучить подходы к практическому построению линейных моделей и их решению на примере задачи о полках. (см файл Задача о полках.pdf)

\item Изучить подходы к решених задачи линейного программирования в электронных таблицах (MS Excel / LobreOffice Calc) cс помощью инструмента <<Поиск решения>> (см. файл Задача о полках.pdf стр. 12).

\item Решите прямую и двойственную задачу линейного программирования:
$Z = 10y_2-3y_3 \to \min$,  при условиях~$y_i \geqslant 0$ 
и~$
\left\lbrace\begin{aligned}
-2y_1 + y_2 -y_3 & \geqslant 1,\\
y_1 + 2y_2 -y_3 & \geqslant 3,\\
\end{aligned}\right.
$

% \item Решите задачу линейного программирования:
% \begin{multicols}{2}
% \begin{enumerate}
% \item 
% $\begin{aligned}
%  &F=3x_1 +4x_2+x_3 \to \max\\
% % &x_{1} \geqslant 0, x_{2} \geqslant 0, x_{3} \geqslant 0\\
%  &\left\lbrace
%   \begin{aligned}
%     x_{1}+2x_{2}+x_3 &\leqslant 10\\
%     x_{1}+x_{2}+2x_3 &\leqslant 6\\
%     3x_{1}+x_{2}+2x_3 &\leqslant 12\\
%   \end{aligned} 
%   \right.
% \end{aligned}
% $ 


% \item
% $\begin{aligned}
%  &F=3x_1 +4x_2+2x_3 \to \max\\
% % &x_{1} \geqslant 0, x_{2} \geqslant 0, x_{3} \geqslant 0\\
%  &\left\lbrace
%   \begin{aligned}
%     x_{1}+2x_{2}+x_3 &\leqslant 18\\
%     2x_{1}+x_{2}+x_3 &\leqslant 16\\
%     x_{1}+x_{2} &\leqslant 8\\
%     x_{2}+x_{3} &\leqslant 6\\
%   \end{aligned} 
%   \right.
% \end{aligned}
% $ 
% \end{enumerate}
% \end{multicols}
% \item 
% Мебельный комбинат выпускает книжные полки $A$ из натурального дерева со стеклом (2~стекла),полки $P_1$ из полированной ДСП (древесно-стружечной плиты) без стекла и полки $P_2$ из полированной ДСП со стеклом (2 стекла). 
% Габариты полок следующие: длина 1100\,мм, ширина 250\,мм, высота 300\,мм. Размер
% листа ДСП $2\times 3$\,м. Из листов ДСП выкраиваются нижние, верхние и боковые стенки полок $P_1$ и $P_2$.
% Задние стенки для полок $P_1$ и $P_2$ выкраиваются из листов ДВП(древесно-волокнистой плиты).

% При изготовлении полок $A$ выполняются следующие работы: столярные, покрытие лаком, сушка, резка стекла, упаковка. Все операции, производимые в ходе столярных работ и упаковки, выполняются вручную. Полки $P_1$ и $P_2$ поставляются в торговую сеть в разобранном виде. За исключением операции упаковки, все остальные операции (производство комплектующих полки, резка стекла) при изготовлении
% полок $P_1$ и $P_2$, выполняются на специализированных автоматах

% Трудоемкость столярных работ по выпуску одной полки А составляет 4\,ч. Производительность автомата, покрывающего полки $A$ лаком~--- 10 полок в\,час, автомата, режущего стекло~--- 100 стекол в\,час.

% Сменный фонд времени автомата для покрытия лаком~--- 7\,ч, автомата для резки стекла~--- 7.5\,ч. Сушка полок, покрытых лаком, происходит в течение суток в~специальных сушилках, вмещающих 50 полок. На упаковку
% полки $A$ требуется 4~минуты. В производстве полок заняты 40~столяров и 14~упаковщиков.

% Производительность автомата, производящего комплектующие полок $P_1$ и $P_2$, равна 3~полки в час, а его сменный фонд времени равен 7.4\,ч, трудоемкость упаковочных работ составляет 8 мин для полки $P_1$ и 10 мин для полки $P_2$.

% От поставщиков комбинат получает в месяц 400~листов полированной ДСП, 230 листов ДВП а также 260 листов стекла. Из каждого листа ДВП можно выкроить стандартным образом 14 задних стенок
% полок $P_1$ и $P_2$, а из каждого листа стекла~--- 10 стекол для полок $A$ и $P_2$.

% Склад готовой продукции может разместить не более 350 полок и комплектов полок, причем ежедневно в торговую сеть вывозится в среднем 40~полок и комплектов. На начало текущего месяца на складе осталось 100 полок, произведенных ранее.

% Себестоимость полки $А$ равна 205\,руб., полки $P_1$ без стекла~--- 142\,руб., полки $P_2$ со стеклом~--- 160\,руб.

% Маркетинговые исследования показали, что доля продаж полок обоих видов со стеклом составляет не менее 60\% в общем объеме продаж, а емкость рынка полок всех типов, производимых на предприятии, составляет около (не более) 5300 штук в~месяц. Мебельный комбинат заключил договор на поставку заказчику 50 полок типа $P_2$ в текущем месяце.

% Известны цены реализации полок: полка $A$~--- 295\,руб., полка
% $P_1$ без стекла~--- 182 руб., полка $P_2$ со стеклом~--- 220 руб.

% Составьте план производства полок на текущий месяц так, чтобы максимизировать прибыль от продажи полок всех типов в течение месяца. 
\end{enumerate}
\end{document}	


\item Решите задачу линейного программирования:\\
$$\begin{aligned}
 &F=4x_1 +x_2+3x_3 \to \max\\
 &x_{1} \geqslant 0, x_{2} \geqslant 0, x_{3} \geqslant 0\\
 &\left\lbrace
  \begin{aligned}
    2x_{1}+x_{2}+x_3 &\leqslant 10\\
    x_{1}+2x_{2}+2x_3 &\leqslant 6\\
    x_{1}+2x_{2}+3x_3 &\leqslant 12\\
  \end{aligned} 
  \right.
\end{aligned}
$$







