%encoding=utf8%
\documentclass[a4paper,14pt]{extarticle}
%\usepackage{pgfpages}
%\pgfpagesuselayout{2 on 1}[a4paper,border shrink=5mm,landscape]

% Русская кодировка
\usepackage[T2A]{fontenc}
\usepackage[utf8]{inputenc}
\usepackage[russian]{babel}
%\usepackage[pdftex]{graphicx}
%\usepackage[pdftex,colorlinks,urlcolor=black, linkcolor=black, citecolor=black]{hyperref}
%        \pdfcompresslevel=9 % сжимать PDF

% необходимые модули
\usepackage{amssymb,amsmath,amsthm}
\usepackage{tabularx,multirow,hhline}
\usepackage{indentfirst}
\usepackage{tikz}
\usepackage[margin=10mm]{geometry}
\usepackage[inline]{enumitem}
\usepackage{multicol}
\setlist{itemsep=0pt,leftmargin=\parindent}

\AddEnumerateCounter{\Asbuk}{\@Asbuk}{\CYRM}
\AddEnumerateCounter{\asbuk}{\@asbuk}{\cyrm}
\newlist{dotenumerate}{enumerate}{10}
\setlist[enumerate,1]{label=\arabic*., ref=\arabic*} %списки со скобками
\setlist[enumerate,2]{label=\asbuk*), ref=(\asbuk*)} %списки со скобками

\usepackage{environ}

\NewEnviron{LOOP}[1]{
  \newcounter{nclone}
  \setcounter{nclone}{#1} 
  \par
  \loop
    \BODY
  \addtocounter{nclone}{-1}
  \ifnum \value{nclone}>0 \repeat}

%Полуторный интервал
\renewcommand{\baselinestretch}{1.00}
\pagestyle{empty}
\newcolumntype{C}{>{\centering\arraybackslash}X}


\sloppy
\binoppenalty=10000
\relpenalty=10000
\begin{document}
{\centering
{\small Практическое занятие \textnumero~20 \par}
{\bfseries Теоремы о вероятностях}
\par}
\begin{enumerate}
 
%\item
%Проводятся две лотереи. Если событие  ${A_{{1}}=}$[выигрыш по билету
%первой лотереи] и событие  ${A_{{2}}=}$[выигрыш по билету второй
%лотереи], то что означают события: 
%${A_{{1}}A_{{2}}+\overline{{A_{{1}}}}A_{{2}}}$, 
%${A_{{1}}\overline{{A_{{2}}}}+\overline{{A_{{1}}}}A_{{2}}+A_{{1}}A_{{2}}}$?
%\item
%Известно, что события  ${A}$ и  ${B}$ произошли, а событие  ${C}$ не
%наступило. Определите, произошли ли следующие события: 
%${A+BC}$,  ${(A+B)C}$, ${AB+C}$, }
%\item
%Среди студентов наудачу выбирают одного. Рассматриваются события:
%\textit{A}=[выбранный студент старше двадцати лет] и событие
%\textit{B}=[выбранный студент~--- отличник], \textit{C}=[выбранный
%студент живет в общежитии]. Опишите событие 
%${\overline{{A}}{BC}}$. При каком условии имеет
%место равенство \textit{ABC=A}? При каком условии выполняется
%соотношение  ${\overline{{A}}\subset C}$? Наудачу выбранным студентом
%оказался отличник Петя Васильев. Наступило ли событие 
%${\overline{{A}}B}$?
\item
В корзине 5 белых, 8 синих и 7 черных шаров. Шары достают по одному и
возвращают обратно в корзину. Достали два шара. Какова вероятность, что
они одного цвета? Тот же вопрос, если шары достаются без возвращения.
%\item
%Производится сбрасывание одной бомбы по складам боеприпасов. При
%попадании в один склад взрываются все три. Вероятность попадания в
%первый склад равна 0,01, во второй~--- 0,008, в третий~--- 0,025. Найти
%вероятность того, что склады будут взорваны.
\item
Брошены три игральные кости. Найти вероятность событий: $A$~--- на каждой грани
появится 5 очков; 
$B$~--- ровно на двух кубиках выпадет по 5 очков;
$C$~--- На всех кубиках выпадет разное число очков;
$D$~--- на всех кубиках выпадет одинаковое количество очков.

%\item
%Для сигнализации об аварии установили два независимо работающих
%сигнализатора. Вероятность того, что при аварии сработает первый
%сигнализатор, равна 0,95. Вероятность того, что при аварии сработает
%второй сигнализатор, равна 0,9. Найти вероятность того, что при аварии
%сработает ровно один сигнализатор.
\item
Вероятность одного попадания в цель при залпе из двух орудий равна 0,38.
Найти вероятность поражения цели выстрелом из первого орудия, если для
второго орудия эта вероятность равна 0,8.
\item
Среди ста лотерейных билетов есть 5 выигрышных. Найти вероятность того,
что 2 наудачу выбранных билета окажутся выигрышными.
% \item
% В урне имеется 5 шаров с номерами от 1 до 5. Наудачу по одному извлекают
% 3 шара без возвращения. Найти вероятности событий: последовательно
% появятся шары с номерами 1, 4, 5; извлеченные шары будут иметь номера
% 1, 4, 5 не зависимо от того, в каком порядке они появились.
\item
Четыре охотника стреляют в одного зайца. Зайцу достаточно одного попадания. Найти вероятность
того, что охотники, которые поражают движущуюся мишень с  вероятностями 0,3; 0,4; 0,6; 0,2 соответственно,
уйдут с трофеем.
\item 
Один властелин решил избавиться от своего звездочета, дающего ложные предсказания. Но властелин дал шанс звездочету спастись. Властелин предложил звездочету разложить по двум урнам 2 черных и два белых шара. После чего палач наугад вытаскивает шар из произвольной урны. Если шар будет черным, звездочета казнят, а если белым~--- помилуют. Каким образом нужно разложить звездочету шары, чтобы шансы спастись были максимальными?
\item
Студент идет сдавать экзамен н по математике. Он знает 20 билетов из 25. Каким ему выгоднее идти, первым или вторым? (Билетов не меньше, чем студентов).
%\item
%В вычислительной лаборатории имеются 6 клавишных автоматов и 4
%полуавтомата. Вероятность того, что за время выполнения некоторого
%расчета автомат не выйдет из строя равна 0,95, для полуавтомата эта
%вероятность равна 0,8. Студент проводит расчет на наудачу выбранной
%машине. Найти вероятность того, что до окончания расчета машина не
%выйдет из строя.

\item
В пирамиде 10 винтовок, из которых 4 снабжены оптическими прицелами.
Вероятность того, что стрелок поразит мишень при выстреле из винтовки с
оптическим прицелом, равна 0,95. Для винтовки без оптического прицела
эта вероятность равна 0,8. Стрелок поразил цель из наудачу выбранной
винтовки. Что вероятнее: стрелок стрелял из винтовки с оптическим
прицелом или без него?

\item
Проводится медицинское обследование на некоторое заболевание.
Пусть доля больных по отношению ко всему населению равна 0,0001
Ошибочность теста~--- 2\%. 
Тест дважды показал, что пациент здоров. Какова вероятность того, что пациент действительно здоров?

{\item[]\centering\bfseries Домашнее задание\par}
\item
Три исследователя, независимо друг от друга производят измерения
некоторой физической величины. Вероятность того, что первый
исследователь допустит ошибку при считывании показаний прибора, равна
0,1. Для второго и третьего исследователя эта вероятность
соответственно равна 0,15 и 0,2. Найти вероятность того, что при
однократном измерении хотя бы один из исследователей допустит ошибку.
\item
В ящике 10 деталей, из которых 4 окрашены. Сборщик наугад взял 3 детали.
Найти вероятность того, что хотя бы одна из них окрашена.
% \item
% Брошены три игральные кости. Найти вероятности следующих событий: а) на
% двух выпавших гранях одно очко, а на третьей~--- другое количество очков;
% б) на двух выпавших гранях появится одинаковое количество очко, а на
% третьей~--- другое; в) на всех выпавших гранях~--- разное количество
% очков.
\item
Вероятность попадания в мишень стрелка при одном выстреле равна 0,8.
Сколько выстрелов должен произвести стрелок, чтобы вероятность
отсутствия промахов была меньше 0,4?
\item
В урну, содержащую два шара, каждый из которых равновероятно может быть
как черным так и белым, опускают белый шар, после чего извлекают один
шар. Найти вероятность того, что извлеченный шар окажется белым.
\item
Имеется две урны. В первой лежит 3 белых и 4 черных шара, во второй~--- 2 белых и 3 черных шара. 
Из первой урны наудачу берут два шара и перекладывают во вторую, а затем из второй урны вынимают один шар. Он оказался белым. 
Какой состав переложенных шаров является наиболее вероятным?


\end{enumerate}

\end{document}	