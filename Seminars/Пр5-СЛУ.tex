%encoding=utf8%
\documentclass[a5paper,10pt]{extarticle}

% Русская кодировка
\usepackage[T2A]{fontenc}
\usepackage[utf8]{inputenc}
\usepackage[russian]{babel}

\usepackage[pdftex]{graphicx}
\usepackage[pdftex,colorlinks,urlcolor=black, linkcolor=black, citecolor=black]{hyperref}
%        \pdfcompresslevel=9 % сжимать PDF

% необходимые модули
\usepackage{amssymb,amsmath,amsthm}
\usepackage{tabularx}
\usepackage{indentfirst}
\usepackage{tikz}
\usepackage[margin=9mm]{geometry}
\usepackage[inline]{enumitem}
\usepackage{multicol}
\setlist{itemsep=1pt,leftmargin=\parindent}

\AddEnumerateCounter{\Asbuk}{\@Asbuk}{\CYRM}
\AddEnumerateCounter{\asbuk}{\@asbuk}{\cyrm}
\newlist{dotenumerate}{enumerate}{10}
\setlist[enumerate,1]{label=\arabic*., ref=\arabic*} %списки со скобками
\setlist[enumerate,2]{label=\textbf{\asbuk*)}, ref=(\asbuk*)} %списки со скобками

\usepackage{environ}

\NewEnviron{LOOP}[1]{
  \newcounter{nclone}
  \setcounter{nclone}{#1} 
  \par
  \loop
    \BODY
  \addtocounter{nclone}{-1}
  \ifnum \value{nclone}>0 \repeat}

%Полуторный интервал
\renewcommand{\baselinestretch}{1.00}
\pagestyle{empty}
\newcolumntype{C}{>{\centering\arraybackslash}X}

%\usepackage{ccfonts,eulervm}
%\usepackage[math]{iwona}
%\renewcommand{\sfdefault}{iwona}\normalfont
%\renewcommand{\rmdefault}{pplx}\normalfont

\sloppy
\binoppenalty=10000
\relpenalty=10000


\newcommand{\isShare}{\url{https://cloud.mail.ru/public/5tTJ/4DyNiAiTB}}
\newcommand{\HEAD}[2]{
	{\noindent\footnotesize Занятие \textnumero~#1  \hfill  Материалы курса: \isShare\par}
	\hrule
	\vspace{3mm}
	{
		\centering 
		\bfseries \large #2\par
		\par
	}
}
\begin{document}
{\centering \small Практические занятия \textnumero~6 \par\bfseries \large Системы линейных уравнений\par}

\begin{enumerate}
\item Решить СЛУ:\\
\begin{enumerate*}
	\item 
	$\left\lbrace\begin{aligned}
	2x-y-z &= 4,\\ 
	3x+4y-2z &= 11,\\
	3x-2y+4z &=11;
      \end{aligned}\right.$
	\item 
 $\left\lbrace\begin{aligned}2x-3y+z-2 &=0,\\x+5y-4z+5 &=0,\\4x+y-3z+4
	&=0; \end{aligned}\right.$ 
	\item 
	$\left\lbrace\begin{aligned}
    -x_{1}+x_{2}+2x_{3} &= -5,\\ 
    6x_{1}-2x_{2}+3x_{3} &= -7,\\
    x_{1}+8x_{2}+5x_{3} &=  2;\\
  \end{aligned}\right.$
  
%	\item 
%	$\left\lbrace\begin{aligned}
%    2x_{1}+2x_{2}-x_{3}+x_{4} &= 4,\\
%    4x_{1}+3x_{2}-x_{3}+2x_{4} &= 6,\\
%    8x_{1}+5x_{2}-3x_{3}+4x_{4} &= 12,\\
%    3x_{1}+3x_{2}-2x_{3}+2x_{4} &= 6.
%  \end{aligned}\right.$
\end{enumerate*}
\item
	Из  некоторого  листового  материала  необходимо  выкроить  360  заготовок типа~\texttt{А},  300  заготовок  типа~\texttt{Б}  и  675  заготовок  типа~\texttt{В}.  При  этом можно применять  три  способа  раскроя.  При  первом  способе  раскроя  получается  3 заготовки типа~\texttt{А}, 1 заготовка типа~\texttt{Б} и 4 заготовки типа~\texttt{В}, при втором способе раскроя получается 2 заготовки типа~\texttt{А}, 6 заготовок типа~\texttt{Б} и 1 заготовка  типа~\texttt{В},  при  третьем  способе  раскроя  получается  1  заготовка  типа~\texttt{А},  2  заготовки типа~\texttt{Б} и 5 заготовок типа~\texttt{В}. Найти расход материала при каждом из указанных способов раскроя.
%\item Решить методом Гаусса:\\
%\begin{enumerate*}
%\item 
%	$\left\lbrace\begin{aligned}
%	3x-5y &=13, \\
%	2x+7y &= 81;
%	\end{aligned}\right.$
%\item 
%	$\left\lbrace\begin{aligned}
%		2x+y &= 5,\\
%		x+3z &=16,\\
%		5y-z &=10.
%	\end{aligned}\right.$
%\item 
%   $\left\lbrace\begin{aligned}
%    -x_{1}+x_{2}+2x_{3} &= -5\\ 
%    6x_{1}-2x_{2}+3x_{3} &= -7\\
%    x_{1}+8x_{2}+5x_{3} &=  2\\
%  \end{aligned}\right.$;
%  %(1,2,-3)
%\end{enumerate*}

\item Исследовать систему линейных уравнений и найти её решения:\\
\begin{enumerate*}
\item 
	$\left\lbrace\begin{aligned}
    9x_{1}-3x_{2}+5x_{3}+x_4 &= 6,\\
    6x_{1}-2x_{2}+3x_{3}+2x_4 &= 4,\\
    3x_{1}-x_{2}+3x_{3}+7x_4 &= 2;
  \end{aligned}\right.$
	%%(a,13+3a,-7,0)
\item 
	$\left\lbrace\begin{aligned}
    x_{1}+x_{2}+3x_{3}-2x_4+3x_5 &= 1,\\
    2x_{1}+2x_{2}+4x_{3}-x_4+3x_5 &= 2,\\
    3x_{1}+3x_{2}+5x_{3}-2x_4+3x_5 &= 1,\\
    2x_{1}+2x_{2}+8x_{3}-3x_4+9x_5 &= 2;\\
  \end{aligned}\right.$\\
	%%нет решений
%\item 
%	$\left\lbrace\begin{aligned}
%    2x_{1}- x_{2}+ x_{3}+2x_4+ 3x_5 &= 2,\\
%    6x_{1}-3x_{2}+2x_{3}+4x_4+ 5x_5 &= 3,\\
%    6x_{1}-3x_{2}+4x_{3}+8x_4+13x_5 &= 9,\\
%    4x_{1}-2x_{2}+ x_{3}+ x_4+ 2x_5 &= 1;\\
%  \end{aligned}\right.$
%	%% (a, b, 5-8a+4b,-3, 1+2a-b)
\end{enumerate*}
\item 
	На предприятии имеется четыре технологических способа изготовления изделий~$A$ и~$B$ из некоторого сырья. В таблице указано количество изделий, которое может быть произведено из единицы сырья каждым из технологических способов:

{\centering
\begin{tabularx}{12cm}{>{\bfseries}c|CCCC}
\hline
 & \multicolumn{4}{c}{Выход из единицы сырья}\\
Изделие  & I & II & III & IV\\
\hline
$A$ & 2 & 1 & 7 & 4\\
$B$ & 6& 12& 2 & 3\\
\hline
\end{tabularx}
\par}

Найти количество сырья, которое следует переработать по каждой технологии, чтобы произвести 574~изделия~$A$ и~328~изделий~$B$ из 94\;ед. сырья.

\item 
	Три судна доставили в порт 6000\;т чугуна, 4000\;т железной руды и 3000\;т апатитов. Разгрузку можно производить как непосредственно в железнодорожные вагоны для последующей доставки потребителям, так и на портовые склады. В вагоны можно разгрузить 8000\,т, а остаток груза придётся направить на склады.  Стоимость выгрузки одной тонны чугуна, железной руды и апатитов в~вагоны составляет соответственно $4{,}30$, $5{,}25$ и $2{,}20$\,ден.\,ед.
Описать план разгрузки судов, если затраты на нее должны составить 58850\,ден.\,ед. Учтите, что поданные в порт вагоны не приспособлены для перевозки апатитов.

\end{enumerate}
\newpage

\end{document}
