%encoding=utf8%
\documentclass[a4paper,12pt]{extarticle}

% Русская кодировка
\usepackage[T2A]{fontenc}
\usepackage[utf8]{inputenc}
\usepackage[russian]{babel}

\usepackage[pdftex]{graphicx}
\usepackage[pdftex,colorlinks,urlcolor=blue, citecolor=magenta]{hyperref}
%        \pdfcompresslevel=9 % сжимать PDF

% необходимые модули
\usepackage{amssymb,amsmath,amsthm}
\usepackage{indentfirst}
\usepackage{tikz}
\usepackage[margin=9mm]{geometry}
\usepackage[inline]{enumitem}
\setlist{noitemsep,leftmargin=\parindent}

\AddEnumerateCounter{\Asbuk}{\@Asbuk}{\CYRM}
\AddEnumerateCounter{\asbuk}{\@asbuk}{\cyrm}
\newlist{dotenumerate}{enumerate}{10}
\setlist[enumerate,1]{label=\arabic*., ref=\arabic*} %списки со скобками
\setlist[enumerate,2]{label=\asbuk*), ref=(\asbuk*)} %списки со скобками

\usepackage{tabularx}
\newcolumntype{C}{>{\centering\arraybackslash}X}


%Полуторный интервал
\renewcommand{\baselinestretch}{1.00}
\pagestyle{empty}
\begin{document}
{\centering \small Практическое занятие \textnumero~1 \par\bfseries \large Графы \par}

\begin{enumerate}
\item Узел 7~--- склад, остальные узлы~--- строительные площадки компании. 
Показатели на дугах --- расстояния в километрах. 
Какова длина кратчайшего пути от склада до строительной площадки 1? Проходит ли кратчайший путь от склада к строительной площадке 1 через строительную площадку 2? Какова длина кратчайшего пути от склада до строительной площадки 2? Проходит ли кратчайший путь от склада к строительной
площадке 2 через строительную площадку 4?

{\centering
\begin{tikzpicture}[y=7.5mm, x=8mm,above,sloped]
\node[draw,circle] (1) at (1,1) {1};
\node[draw,circle] (2) at (3,3) {2};
\node[draw,circle] (3) at (4,0) {3};
\node[draw,circle] (4) at (6.5,1.5) {4};
\node[draw,circle] (5) at (9,0) {5};
\node[draw,circle] (6) at (10,1.5) {6};
\node[draw,rectangle] (7) at (8,3) {7};
\draw 
	(1) -- node{15} (2)
	(1) -- node{10} (3)
	(2) -- node{3}  (3)
	(2) -- node{6}  (4)
	(2) -- node{17} (7)
	(3) -- node{4}  (5)
	(4) -- node{5}  (7)
	(4) -- node{4}  (5)
	(5) -- node{2}  (6)
	(6) -- node{6}  (7)
	;
\end{tikzpicture}
\par}

\item Необходимо провести свет в 8 поселков района. Стоимость прокладки ЛЭП между населенными пунктами показана в таблице. Разработать наиболее экономичную схему электрификации.

{\centering \begin{tabular}{|c|c|c|c|c|c|c|c|c|}
\hline
Поселок & 1 & 2 & 3 & 4 & 5  & 6  & 7 & 8\\
\hline
1 &  & 13 & 9 & 14 & 14 & -  & 20 & 18 \\
\hline
2 &  &    & 6 & -  & 15 & 9  & 21 & - \\
\hline
3 &  &    &   & 12 & -  & 11 & 17 & - \\
\hline
4 &  &    &   &    & 8  & 17 & -  & - \\
\hline
5 &  &    &   &    &    & 16 & -  & - \\
\hline
6 &  &    &   &    &    &    & 19 & - \\
\hline
7 &  &    &   &    &    &    &    & 31 \\
\hline
%8 &  &    &   &    &    &    &    &    \\
%\hline
\end{tabular}\par}
\item 
Руководство компании решает, создавать ли для выпуска новой продукции крупное производство, малое предприятие или продать патент другой фирме. Размер выигрыша, который компания может получить, зависит от благоприятного или неблагоприятного состояния рынка.

{\centering \begin{tabular}{|c|l|c|c|}
	\hline
	Вид  & Действия компании & \multicolumn{2}{c|}{Выигрыш при состоянии рынка}\\
	стратегии & & благоприятном & неблагоприятном\\
	\hline
	1 & крупное производство  & 200\,000  & -180\,000\\
	\hline
	2 & малое предприятие  & 100\,000 &  -20\,000\\
	\hline
	3 & продажа патента &  10\,000 &  10\,000\\
	\hline
\end{tabular}\par}

Перед принятием решения о строительстве руководство компании может
обратиться к специализированной фирме, которая проведет исследование состояния рынка и уточнит вероятности исходов. 

Стоимость этого исследования составляет 10000 руб. 
Возможности фирмы в части достоверности ее прогнозов представлены в следующей таблице.

{\centering
\begin{tabular}{|l|c|c|}
\hline 
Прогноз  & \multicolumn{2}{c|}{фактически} \\
 фирмы& благоприятный & неблагоприятный\\
\hline 
благоприятный & 0,78 & 0,22 \\
\hline 
неблагоприятный & 0,27 & 0,73 \\
\hline 
\end{tabular}\par}

Когда фирма выдает прогноз, что рынок будет благоприятный, то этот прогноз оправдывается с~вероятностью 0,78

\begin{itemize}
\item \textit{Целесообразно ли проведение дополнительного исследования?}

\item Предположим, исследование состоялось и фирма выдала прогноз, что рынок будет благоприятный с вероятностью 0,45. \textit{Требуется определить оптимальную стратегию предприятия?}
\end{itemize}

\item Фермер может выращивать либо кукурузу, либо соевые бобы. Вероятность того, что цены на будущий урожай этих культур повысятся, останутся на том же уровне или понизятся, равна соответственно 0,25, 0,30 и 0,45. Если цены возрастут, урожай кукурузы даст 30\,000 долл. чистого дохода, а урожай соевых бобов~--- 10\,000 долл. Если цены останутся неизменными, фермер лишь покроет расходы. Но если цены станут ниже, урожай кукурузы и соевых бобов приведет к потерям в 35\,000 и 5\,000 долл. соответственно. Какую культуру следует выращивать фермеру? Каково ожидаемое значение его прибыли? 

\item \textbf{На дом:} задания 1,2,3 из пособия Ряттель А.\,В. Основы экономико-математического
моделирования стр. 9--10. Ищите здесь: \url{https://cloud.mail.ru/public/3soD/y91HFQd3H}

\end{enumerate}
\end{document}
