%encoding=utf8%
\documentclass[a4paper,14pt]{extarticle}
%\usepackage{pgfpages}
%\pgfpagesuselayout{2 on 1}[a4paper,border shrink=5mm,landscape]

% Русская кодировка
\usepackage[T2A]{fontenc}
\usepackage[utf8]{inputenc}
\usepackage[russian]{babel}
%\usepackage[pdftex]{graphicx}
%\usepackage[pdftex,colorlinks,urlcolor=black, linkcolor=black, citecolor=black]{hyperref}
%        \pdfcompresslevel=9 % сжимать PDF

% необходимые модули
\usepackage{amssymb,amsmath,amsthm}
\usepackage{tabularx,multirow,hhline}
\usepackage{indentfirst}
\usepackage{tikz}
\usepackage[margin=12mm]{geometry}
\usepackage[inline]{enumitem}
\usepackage{multicol}
\setlist{itemsep=1pt,leftmargin=\parindent}

\AddEnumerateCounter{\Asbuk}{\@Asbuk}{\CYRM}
\AddEnumerateCounter{\asbuk}{\@asbuk}{\cyrm}
\newlist{dotenumerate}{enumerate}{10}
\setlist[enumerate,1]{label=\arabic*., ref=\arabic*} %списки со скобками
\setlist[enumerate,2]{label=\asbuk*), ref=(\asbuk*)} %списки со скобками

\usepackage{environ}

\NewEnviron{LOOP}[1]{
  \newcounter{nclone}
  \setcounter{nclone}{#1} 
  \par
  \loop
    \BODY
  \addtocounter{nclone}{-1}
  \ifnum \value{nclone}>0 \repeat}

%Полуторный интервал
\renewcommand{\baselinestretch}{1.00}
\pagestyle{empty}
\newcolumntype{C}{>{\centering\arraybackslash}X}


\sloppy
\binoppenalty=10000
\relpenalty=10000
\begin{document}
{\centering {\scriptsize Практическое занятие \textnumero~17 \par}
\bfseries Многокритериальные задачи
\par\vspace{1mm}
}
\begin{enumerate}
    \item 
    Химический комбинат планирует внедрить комплекс средств автоматизации (КСА) для системы управления технологическими процессами. Имеется возможность выбрать один из семи вариантов КСА (КСА1, КСА2, ..., КСА7). При выборе учитываются четыре критерия: затраты, связанные с изготовлением КСА и его вводом в эксплуатацию; срок ввода КСА в эксплуатацию; срок гарантийного обслуживания предприятием изготовителем; удобство КСА в эксплуатации. Характеристики КСА приведены в таблице:

    {\centering \small
    \begin{tabular}{|p{6cm}|c|c|c|c|c|c|c|}
        \hline
        Критерий & КСА1 & КСА2 & КСА3 & КСА4 & КСА5 & КСА6 & КСА7\\
        \hline
        Затраты, млн ден. ед. &
        40 &  30 & 40 & 60 & 45 & 25 & 55 \\
        \hline
        Срок ввода в эксплуатацию, мес. & 
        8 & 8 & 6 & 6 & 7 & 8 & 6 \\
        \hline
        Срок гарантийного обслуживания, лет &
        4 & 4 & 5 & 7 & 4 & 4 & 5 \\
        \hline
        Удобство в эксплуатации &
        Хор. & Отл.  & Удовл.  & Отл.  & Плохо  & Оч.~хор.  & Хор. \\
        \hline
    \end{tabular}
    \par}

    Критерии <<затраты>> и <<срок ввода в эксплуатацию>> подлежат минимизации, 
    критерий <<срок гарантийного обслуживания>>~--- максимизации). 
    Выбрать множество Парето.


    \item  Решить линейную многокритериальную задачу методом идеальной точки:
        $f_1=-x_1+3x_2 \to \max$, $f_2(x)=4x_1-x_2 \to \max$, при ограничениях
        $-x_1+x_2 \leqslant 1$, $x_1+x_2 \geqslant 3$, $x_1-2x_2 \leqslant 0$ , $x_1 \leqslant 4$, $x_2 \leqslant 3$. 
    
    \item[] \textbf{Домашнее задание}
   \item 
    Решить линейную многокритериальную задачу:
   $f_1=-2x+1 \to \max$, $f_2(x)=2y+3 \to \max$, при ограничениях
   $0 \leqslant x \leqslant 1$, $ 0 y \leqslant 1$.  И при условии, что идеальная точка (точка утопии) известна и имеет координаты $M^*(5,7)$.
   \item 
    Решить предыдущую задачу при условии, что $f_2 \to \min$.
   \item Задание 1 стр. 31 пособия А.\,В. Ряттель.
%    \item Задание 2 стр. 32 пособия А.\,В. Ряттель.
\end{enumerate}

\end{document}	