%encoding=utf8%
\documentclass[a4paper,14pt]{extarticle}

% Русская кодировка
\usepackage[T2A]{fontenc}
\usepackage[utf8]{inputenc}
\usepackage[russian]{babel}

\usepackage[pdftex]{graphicx}
\usepackage[pdftex,colorlinks,urlcolor=blue, citecolor=magenta]{hyperref}
%        \pdfcompresslevel=9 % сжимать PDF

% необходимые модули
\usepackage{amssymb,amsmath,amsthm}
\usepackage{indentfirst}
\usepackage{tikz}
\usepackage[margin=9mm]{geometry}
\usepackage[inline]{enumitem}
\setlist{noitemsep,leftmargin=\parindent}

\AddEnumerateCounter{\Asbuk}{\@Asbuk}{\CYRM}
\AddEnumerateCounter{\asbuk}{\@asbuk}{\cyrm}
\newlist{dotenumerate}{enumerate}{10}
\setlist[enumerate,1]{label=\arabic*., ref=\arabic*} %списки со скобками
\setlist[enumerate,2]{label=\asbuk*), ref=(\asbuk*)} %списки со скобками

\usepackage{tabularx}
\newcolumntype{C}{>{\centering\arraybackslash}X}


%Полуторный интервал
\renewcommand{\baselinestretch}{1.00}
\pagestyle{empty}
\begin{document}
{\centering \small Практическое занятие \textnumero~3 \par
\bfseries \large Сетевые модели \par}

\begin{enumerate}

\item 
По имеющимся данным требуется:

{\centering
\begin{tabular}{c>{\centering}p{4cm}>{\centering}p{4cm}>{\centering}p{4cm}}
\hline
Работа, & Длительность работы, дн. $t_{ij}$ & Минимальное время работы,$d_{ij}$ дн. & Коэффициент использования дополнительных средств, $k_{ij}$\tabularnewline
\hline
1,2 & 10 &  6 & 0,6  \tabularnewline
1,3 &  8 &  5 & 0,1 \tabularnewline
2,3 & 14 & 10 & 0,3 \tabularnewline
2,4 &  6 &  2 & 0,8 \tabularnewline
3,4 &  5 &  4 & 0,9 \tabularnewline
3,5 & 12 &  7 & 0,5 \tabularnewline
4,5 &  4 &  2 & 0,3 \tabularnewline
\hline
\end{tabular}
\par}
Ограничение на затраты: 210 ден.ед.

\begin{enumerate}
\item построить сетевой график
\item выделить критический путь и найти его длину;
\item определить резервы времени каждого события;
\item определить резервы времени (полные, частные первого вида, свободные и независимые) всех работ и коэффициенты напряженности работ, не лежащих на критическом пути;
\item выполнить оптимизацию сетевого графика по времени.
\end{enumerate}
\item 
Для проведения некоторых работ предприятие арендует производственное помещение. Стоимость аренды 10 ден. ед в день. В таблице приведён расчет продолжительностей выполняемых работ. Оптимизируйте работы по времени выполнения.

{\centering
\begin{tabular}{c>{\centering}p{3cm}>{\centering}p{3cm}>{\centering}p{3cm}>{\centering}p{3cm}}
\hline
& \multicolumn{2}{>{\bfseries}c}{Нормальный режим работ}& \multicolumn{2}{>{\bfseries}c}{Максимальный режим работ}\tabularnewline
Операция & Длительность, дн. & Стоимость, ден. ед. &Длительность, дн. & Стоимость, ден. ед.\tabularnewline
\hline
1,2 & 4	&  80 & 2 & 150 \tabularnewline
1,3 & 2	&  50 & 1 &  70 \tabularnewline
1,4 & 3	&  60 & 2 &  80 \tabularnewline
2,4 & 2	&  60 & 1 &  70 \tabularnewline
2,6 & 6	& 100 & 3 & 160 \tabularnewline
3,4 & 2	&  40 & 1 &  60 \tabularnewline
3,5 & 3	&  70 & 2 &  90 \tabularnewline
4,6 & 4	&  90 & 2 & 170 \tabularnewline
5,6 & 4	&  80 & 2 & 160 \tabularnewline
\hline
\end{tabular}
\par}















%Для улучшения финансового состояния фирме необходимо увеличить спрос на выпускаемый
%цемент марки М400 и расширить потребительский рынок. Фирма считает целесообразным размещать
%цемент в специализированной таре. Для переоснащения цеха необходимо установить оборудование по
%производству специализированной тары. Предполагается выполнить следующее:
%
%
%%\begin{tabular}{|cp{8cm}|cc|cc|}
%%%\hline
%%%	№ п.\,п. & Yfbvtyjdf
%%\hline
%%	1 & Подготовка и выпуск технического задания на переоборудование цеха 
%%	  & 20 & 20 & 18 & 26 \\
%%	2 & Разработка мероприятий по технике безопасности						
%%	  & 25 & 30 & 20 & 37 \\
%%	3 & Подбор кадров
%%	  & 10 & 5 & 9 & 7 \\
%%	4 & Заказ и поставка необходимого оборудования
%%	  & 30 & 60 & 23 & 64 \\ 
%%	5 & Заказ и поставка электрооборудования
%%	  & 40 & 65 & 32 & 78 \\
%%	6 & Установка оборудования
%%	  & 50 & 90 & 43 & 100 \\
%%	7 & Установка электрооборудования
%%	  & 45 & 80 & 41 & 85 \\
%%	8 & Обучение персонала
%%	  & 15 & 5 & 9 & 10 \\
%%	9 & Испытание и сдачу в эксплуатацию линии	
%%	  & 25 & 50 & 21 & 57 \\
%%\hline
%%\end{tabular}
%%Ожидается, что производительность вводимой линии по производству тары составит 1000
%%мешков в день при односменном режиме работы. 
%%Стоимость 1 мешка~--- 25 р., 
%%выручка от реализации тары в смену составит 25 тыс. р., 
%%из которых чистая прибыль фирмы равна 50 тыс. р. 
%%Деньги на покупку оборудования и переоснащение цеха в размере 5 500 тыс. р. взяты в банке под 30% годовых из расчета
%%5000 тыс. р. на оборудование и 500 тыс. р. на его установку.
%%Затраты на проведение работ и их продолжительность в нормальном и максимальном режимах
%%указаны в табл. 30.8

\medskip
\item \textbf{На дом:} См. файл \texttt{Пр3-Сети. Оптимизация. На дом.pdf}.  Ищите здесь: \url{https://cloud.mail.ru/public/3soD/y91HFQd3H}
 


\end{enumerate}
\end{document}
