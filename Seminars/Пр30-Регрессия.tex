%encoding=utf8%
\documentclass[a4paper,12pt]{extarticle}
% \usepackage{pgfpages}

% Русская кодировка
\usepackage[T2A]{fontenc}
\usepackage[utf8]{inputenc}
\usepackage[russian]{babel}
%\usepackage[pdftex]{graphicx}
%\usepackage[pdftex,colorlinks,urlcolor=black, linkcolor=black, citecolor=black]{hyperref}
%        \pdfcompresslevel=9 % сжимать PDF
% необходимые модули
\usepackage{amssymb,amsmath,amsthm}
\usepackage{tabularx,multirow,hhline}
\usepackage{indentfirst}
\usepackage{tikz}
\usepackage{icomma}
\usepackage[margin=8mm]{geometry}
\usepackage[inline]{enumitem}
\usepackage{multicol}
\setlist{nosep,leftmargin=\parindent}

\AddEnumerateCounter{\Asbuk}{\@Asbuk}{\CYRM}
\AddEnumerateCounter{\asbuk}{\@asbuk}{\cyrm}
\newlist{dotenumerate}{enumerate}{10}
\setlist[enumerate,1]{label=\arabic*., ref=\arabic*} %списки со скобками
\setlist[enumerate,2]{label=\asbuk*), ref=(\asbuk*)} %списки со скобками

\usepackage{environ}

\NewEnviron{LOOP}[1]{
  \newcounter{nclone}
  \setcounter{nclone}{#1} 
  \par
  \loop
    \BODY
  \addtocounter{nclone}{-1}
  \ifnum \value{nclone}>0 \repeat}

%Полуторный интервал
\renewcommand{\baselinestretch}{1.00}
\pagestyle{empty}
\newcolumntype{C}{>{\centering\arraybackslash}X}


\sloppy
\binoppenalty=10000
\relpenalty=10000

\begin{document}
{\centering {\footnotesize Практическое занятие \textnumero~30 \par}
\bfseries   Эконометрические модели. Парная регрессия
\par\vspace{1mm}
}

{\small
\begin{itemize}
\item Уравнение регрессии \(y_x = b_1x+b_2\), где \(b_1 = \frac{\overline{xy}-\overline{x}\overline{y}}{s^2_x} \)
\item Коэффициент корреляции: $r = \frac{\overline{xy}-\overline{x}\overline{y}}{s_x\cdot s_y}$ значим на уровне $\alpha$, если 
$|r|\frac{\sqrt{n-2}}{\sqrt{1-r^2}}>t_{1-\alpha,n-2}$
\item Доверительный интервал для значений \(y\):\quad
\(|y^*_0-\hat{y}_0| \leqslant t_{1-\alpha; n-2} \cdot \frac{\sum\limits_{i=1}^n (\hat{y}_i-y_i)^2}{n-2} \cdot\left(
          1 + \frac{1}{n} 
            + \frac{(x_0-\bar{x})^2}{\sum\limits_{i=1}^n (x_i-\bar{x})}
        \right)
\)
\end{itemize}
}
\medskip
\hrule
\medskip 

\begin{enumerate}
  \item Получены эмпирические данные связи между производительностью труда~$Y$ (тыс. руб.) и энерговооруженностью труда~$X$ (кВт) (в расчёте на одного работающего) для 14 предприятий региона:
  \[
    \begin{array}{|c|c|c|c|c|c|c|c|c|c|c|c|c|c|c|}
      \hline
      x & 2,8 & 2,2 & 3,0 & 3,5 & 3,2 & 3,7 & 4,0 & 
        4,8 & 6,0 & 5,4 & 5,2 & 5,4 & 6,0 & 9,0 \\
      \hline
      y & 6,7 & 6,9 & 7,2 & 7,3 & 8,4 & 8,8 & 9,1 &
        9,8 & 10,6 & 10,7 & 11,1 & 11,8 & 12,1 & 12,4 \\
      \hline
    \end{array}
  \]
  \begin{enumerate*}
    \item Найти коэффициент корреляции и установить тесноту корреляционной зависимости;
    \item оценить значимость этого коэффициента;
    \item построить уравнения линейной регрессии;
    \item сравнить графически эмпирические и теоретические данные.
  \end{enumerate*}
 
  
  \item
    Распределение 60 предприятий химической промышленности по энерговооруженности труда Y (кВТч) и фондовооруженности Х (млн.руб.) дано в таблице.
    \[
    \begin{array}{|c|c|c|c|c|c|}
      \hline
       y \backslash x & 0,0-4,5 & 4,5-9 & 9-13,5 & 13,5-18 & 18-22,5 \\
      \hline
      0,0-1,4 & 4 & 1 &  &  & \\
      \hline
      1,4-2,8 & 4 & 2 &  &  & \\
      \hline
      2,8-4,2 & 2 & 8 & 1 &  & \\
      \hline
      4,2-5,6 &  & 1 & 20 & 4 & \\
      \hline
      5,6-7,0 &  &  & 3 & 3 & 3 \\
      \hline
      7,0-8,4 &  &  &   & 1 & 3 \\
      \hline
    \end{array}
    \]
    Необходимо: 
    \begin{enumerate*}
      \item найти групповые средние и построить эмпирические линии регрессии; 
      \item найти уравнения прямых регрессии, построить их графики; 
      \item оценить тесноту и направление связи между переменными с помощью коэффициента корреляции; 
      \item проверить значимость коэффициента корреляции.
    \end{enumerate*}

  \item[] \textbf{Домашнее задание}
  \item 
     При исследовании корреляционной зависимости между объемом продукции \(X\) (единиц) и ее себестоимости \(Y\) (тыс. руб.) 
     получено следующее уравнение регрессии \(Y\) по \(X\): \( y_x = -0,0004x + 4,22 \). 
     Составить уравнение регрессии \(X\) по \(Y\), если коэффициент корреляции между этими признаками оказался равным \(0,8\), а средний объем продукции \(\bar{x} = 3000\) единиц.

  \item 
    По следующим эмпирическим данным
    \[
      \begin{array}{|c|c|c|c|c|c|}
      \hline
      y \backslash x & 10-20 & 20-30 & 30-40 & 40-50 & 50-60 \\
      \hline
      11-21 & 4 & 6 &  &  &  \\
      \hline
      21-31 &  & 8 & 10 &  & \\
      \hline
      31-41 &  &  & 32 & 3 & 9 \\
      \hline
      41-51 &  &  & 4 & 12 & 6 \\
      \hline
      51-61 &  &  &  & 1 & 5 \\
      \hline
    \end{array}
    \]
    необходимо: 
    \begin{enumerate*}
      \item 
      найти групповые средние и построить эмпирические линии регрессии; 
      \item 
      найти уравнения прямых регрессии, построить их графики; 
      \item 
      оценить тесноту и направление связи между переменными с помощью коэффициента корреляции; 
      \item 
      проверить значимость коэффициента корреляции.
    \end{enumerate*}  
  \item 
  Имеются следующие данные об уровне механизации работ \(X\) (\%) и производительности труда \(Y\) (т/ч) для 14~однотипных предприятий:
  \[
    \begin{array}{|c|c|c|c|c|c|c|c|c|c|c|c|c|c|c|}
      \hline
      х & 32 & 30 & 36 & 40 & 41 & 47 & 56 & 54 & 60 
        & 55 & 61 & 67 & 69 & 76 \\
      \hline
      y & 20 & 24 & 28 & 30 & 31 & 33 & 34 & 37 & 38 
        & 40 & 41 & 43 & 45 & 48 \\
      \hline
    \end{array}
    \]  
  Необходимо: 
  \begin{enumerate*}
    \item оценить тесноту и направление связи между переменными с помощью коэффициента корреляции; 
    \item проверить значимость коэффициента корреляции; 
    \item найти уравнения прямых регрессий; 
    \item сравнить результаты графически.
  \end{enumerate*}
\end{enumerate}

\end{document}	
