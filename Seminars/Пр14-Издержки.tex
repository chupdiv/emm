%encoding=utf8%
\documentclass[a5paper,11pt]{extarticle}
%\usepackage{pgfpages}
%\pgfpagesuselayout{2 on 1}[a4paper,border shrink=5mm,landscape]

% Русская кодировка
\usepackage[T2A]{fontenc}
\usepackage[utf8]{inputenc}
\usepackage[russian]{babel}
%\usepackage[pdftex]{graphicx}
%\usepackage[pdftex,colorlinks,urlcolor=black, linkcolor=black, citecolor=black]{hyperref}
%        \pdfcompresslevel=9 % сжимать PDF

% необходимые модули
\usepackage{amssymb,amsmath,amsthm}
\usepackage{tabularx,multirow,hhline}
\usepackage{indentfirst}
\usepackage{tikz}
\usepackage[margin=6mm]{geometry}
\usepackage[inline]{enumitem}
\usepackage{multicol}
\setlist{itemsep=0pt,leftmargin=\parindent}

\AddEnumerateCounter{\Asbuk}{\@Asbuk}{\CYRM}
\AddEnumerateCounter{\asbuk}{\@asbuk}{\cyrm}
\newlist{dotenumerate}{enumerate}{10}
\setlist[enumerate,1]{label=\arabic*., ref=\arabic*} %списки со скобками
\setlist[enumerate,2]{label=\asbuk*), ref=(\asbuk*)} %списки со скобками

\usepackage{environ}

\NewEnviron{LOOP}[1]{
  \newcounter{nclone}
  \setcounter{nclone}{#1} 
  \par
  \loop
    \BODY
  \addtocounter{nclone}{-1}
  \ifnum \value{nclone}>0 \repeat}

%Полуторный интервал
\renewcommand{\baselinestretch}{1.00}
\pagestyle{empty}
\newcolumntype{C}{>{\centering\arraybackslash}X}


\sloppy
\binoppenalty=10000
\relpenalty=10000
\begin{document}
{\centering {\scriptsize Практическое занятие \textnumero~14 \par}
\bfseries Производная в задачах экономической оптимизации
\par\vspace{1mm}
}
\begin{enumerate}
    \item 
        Требуется оградить забором прямоугольный участок земли площадью 294 кв. м. и затем разделить его на две равные части перегородкой. Каковы должны быть размеры участка, чтобы на постройку забора и перегородки было истрачено наименьшее количество материала?    
    \item 
        Фирма минимизирует средние издержки, которые получаются в результате равными 30 руб/ед. продукции. Чему равны при этом предельные издержки?

    \item 
    Цементный завод производит $x$ тонн цемента в день. По договору он должен ежедневно поставлять строительной фирме не менее 20 т цемента. Производительные мощности завода таковы. Что выпуск цемента не может превышать 90 т в день. Определить, при каком объеме производства средние издержки будут наименьшими, если функция издержек имеет вид:
    $$
        C(x) = -x^3+98x^2+200x
    $$

    % \item 
    %     Товар реализуется по фиксированной цене $p=8$.
    %     Издержки на производство товара заданы функцией $C(x)=10+x+\frac{1}{3}x\sqrt{x}$
    %     Определить оптимальный объем выпуска продукции~$x$.
    \item 
        Предприятие (монополия) устанавливает фиксированную цену $p=380$ за единицу товара. 
        Издержки при произвожстве $x$ единиц товара равны $C(x)=292x+x^2$
        Количество реализуемого товара зависит от производства следующим образом:
        $K(x)=x+\sqrt{x_0}-\sqrt{x}$. При каком объеме производства предприятие получает максимальную прибыль.



        \item Начальный капитал может быть размещён в банке под 50\% годовых или
инвестирован в производство, причем эффективность вложения ожидается в
размере 100\%, а издержки задаются квадратичной зависимостью. Прибыль
облагается налогом в~$p$\%. При каких значениях~$p$
вложение в производство является более выгодным, чем размещение
капитала в банке.

    \item 
        Объем выпущенной заводом продукции  $x$  и выручка  $z$ , полученная от ее реализации, связаны следующей зависимостью:
        $z=10x+\frac 3 2x^2-\frac 1{15}x^3$ .
       Найдите предельную выручку. При каком объеме производства выручка максимальна. 

    \item 
       Имеется портфель акций стоимостью в  $C$  рублей. Известно, что с течением времени стоимость акций повышается по закону  $V=Ce^{t/2}$. С другой стороны, если акции продать, а деньги положить в банк, то на вырученную сумму непрерывно будут начисляться 10\% годовых (это означает, что сумма $V_0$, положенная в банк, через  $t$ лет станет равной $V_1=V_0e^{0.1t}$).
       
       Определите момент времени  $t_0$, в который наиболее выгодно продать имеющийся портфель акций и положить деньги в банк, чтобы через  $t$  лет накопленная сумма была максимальной.
       
    %    \item 
    %    Цена бриллианта пропорциональна квадрату его массы. Если бриллиант разбить на две части. То в каком случае общая стоимость двух частей будет наименьшей?
\end{enumerate}

\end{document}	