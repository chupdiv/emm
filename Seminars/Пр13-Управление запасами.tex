%encoding=utf8%
\documentclass[a5paper,11pt]{extarticle}
%\usepackage{pgfpages}
%\pgfpagesuselayout{2 on 1}[a4paper,border shrink=5mm,landscape]

% Русская кодировка
\usepackage[T2A]{fontenc}
\usepackage[utf8]{inputenc}
\usepackage[russian]{babel}
%\usepackage[pdftex]{graphicx}
%\usepackage[pdftex,colorlinks,urlcolor=black, linkcolor=black, citecolor=black]{hyperref}
%        \pdfcompresslevel=9 % сжимать PDF

% необходимые модули
\usepackage{amssymb,amsmath,amsthm}
\usepackage{tabularx,multirow,hhline}
\usepackage{indentfirst}
\usepackage{tikz}
\usepackage[margin=6mm]{geometry}
\usepackage[inline]{enumitem}
\usepackage{multicol}
\setlist{itemsep=0pt,leftmargin=\parindent}

\AddEnumerateCounter{\Asbuk}{\@Asbuk}{\CYRM}
\AddEnumerateCounter{\asbuk}{\@asbuk}{\cyrm}
\newlist{dotenumerate}{enumerate}{10}
\setlist[enumerate,1]{label=\arabic*., ref=\arabic*} %списки со скобками
\setlist[enumerate,2]{label=\asbuk*), ref=(\asbuk*)} %списки со скобками

\usepackage{environ}

\NewEnviron{LOOP}[1]{
  \newcounter{nclone}
  \setcounter{nclone}{#1} 
  \par
  \loop
    \BODY
  \addtocounter{nclone}{-1}
  \ifnum \value{nclone}>0 \repeat}

%Полуторный интервал
\renewcommand{\baselinestretch}{1.00}
\pagestyle{empty}
\newcolumntype{C}{>{\centering\arraybackslash}X}


\sloppy
\binoppenalty=10000
\relpenalty=10000
\begin{document}
{\centering {\scriptsize Практическое занятие \textnumero~13 \par}
\bfseries Управление запасами
\par\vspace{1mm}
}
\begin{enumerate}
    \item Ежедневный спрос на некоторый продукт составляет 100 единиц.
    Затраты на приобретение каждой партии продукта равны 100 ден.\;ед. не зависимо от объема партии.
    Затраты на хранение единицы продукта 0.02~ден.~ед. в~сутки. 
    Определить наиболее экономичный объем партии и интервал между поставками.
    \item Предположим теперь, что возможен дефицит товара, который влечет штраф в размере 0,03~ден.~ед. на~единицу продукции. Определить наиболее экономичный объем хранимой на складах продукции.
    \item Фирма может производить изделие или покупать его. Если фирма сама выпускает изделие, то каждый запуск его в производство обходится в~20~руб. Интенсивность производства составляет 120~шт. в день. Если изделие закупается, то затраты на осуществление заказа равны 15 руб. Затраты на содержание изделия в запасе независимо от того, закупается оно или производится, равны 2 коп. в день. Потребление изделия фирмой оценивается в 26000~шт. в~год.
    Предполагая, что фирма работает без дефицита, определите, что выгоднее: закупать или производить изделие (в~месяце 22~рабочих дня).
    \item На некотором станке производятся детали в количестве 2000 штук в месяц. Эти детали используются для производства продукции на другом станке с интенсивностью 500 шт. в месяц. По оценкам специалистов компании, издержки хранения составляют 50 коп. в год за одну деталь. Стоимость производства одной детали равна 2,50~руб., а~стоимость на подготовку производства составляет 1000 руб. Каким должен быть размер партии деталей, производимой на первом станке, с какой частотой следует запускать производство этих партий?
\end{enumerate}

\end{document}	