%encoding=utf8%
\documentclass[a4paper,14pt]{extarticle}
% \usepackage{pgfpages}
% \pgfpagesuselayout{2 on 1}[a4paper,border shrink=0mm,landscape]

% Русская кодировка
\usepackage[T2A]{fontenc}
\usepackage[utf8]{inputenc}
\usepackage[russian]{babel}


% необходимые модули
\usepackage{amssymb,amsmath,amsthm}
\usepackage{tabularx,multirow,hhline}
\usepackage{indentfirst}
\usepackage{tikz}
\usepackage[margin=20mm]{geometry}
\usepackage{paralist}
\usepackage[inline]{enumitem}
\usepackage{multicol}
\setlist{noitemsep,leftmargin=\parindent}

\AddEnumerateCounter{\Asbuk}{\@Asbuk}{\CYRM}
\AddEnumerateCounter{\asbuk}{\@asbuk}{\cyrm}
\newlist{dotenumerate}{enumerate}{10}
\newlist{denumerate}{enumerate}{1}
\setlist[enumerate,1]{itemsep=3pt,label=\arabic*., ref=\arabic*} %списки со скобками
\setlist[denumerate,1]{noitemsep,label={Д.\,\arabic*.}, ref=\arabic*} %списки со скобками

\usepackage{environ}

\NewEnviron{LOOP}[1]{
  \newcounter{nclone}
  \setcounter{nclone}{#1} 
  \par
  \loop
    \BODY
  \addtocounter{nclone}{-1}
  \ifnum \value{nclone}>0 \repeat}

%Полуторный интервал
\renewcommand{\baselinestretch}{1.00}
\pagestyle{empty}
\newcolumntype{C}{>{\centering\arraybackslash}X}


\sloppy
\binoppenalty=10000
\relpenalty=10000

\newcommand{\bitem}{\textbf{\item}}
\begin{document}
{\centering 
{\centering {\scriptsize Практическое занятие \textnumero~20 \par}
\bfseries Повторение независимых испытаний\par}
\begin{enumerate}
    \item Два равносильных шахматиста играют в шахматы. Что вероятнее выиграть: две партии из четырех или три партии из шести?

\item Кубик подбросили 4 раза. Какова вероятность того, что 6 очков выпадет хотя бы 3 раза?


% \item Кубик подбросили 10 раз. Каково наивероятнейшее число появления шестерки?
% \item  Пусть всхожесть семян пшеницы составляет 90\%. Чему равна вероятность того, что из~7 посеянных семян взойдут~5?

\item Каждый день акции корпорации <<Бананза>> поднимаются в цене или падают в цене на один пункт с вероятностями соответственно 0,75 и~0,25. Найти вероятность того, что акции после шести дней вернутся к своей первоначальной цене, полагая, что изменения цены акции~--- независимые события.
% \item Моторы, устанавливаемые на самолёт имеют надёжность~$p$. Многомоторный самолёт продолжает лететь, если работает не менее половины его моторов. При каких значениях~$p$ двухмоторный самолёт надёжней четырёхмоторного самолёта?

\item Примерно 70\% клиентов банка расплачиваются по кредитам вовремя. а)~Найти вероятность того, что из 20-ти случайным образом выбранных клиентов банка вовремя расплатятся по кредитам более 15-ти клиентов.
b)~Найти наивероятнейшее число клиентов из выбранных 20-ти, которые вовремя погасят долги по кредитам. с)~Найти вероятность того, что именно наивероятнейшее число клиентов вовремя погасит долги по кредитам.

\item ОТК проверяет на стандартность 900 деталей. Вероятность того, что деталь стандартна, равна 0,9. Найти с вероятностью 0,95 границы, в которых будет заключено число $m$ стандартных деталей среди проверенных.


\item Бригада из десяти человек идёт обедать. Имеются две одинаковые столовые, и каждый член бригады независимо один от другого идёт обедать в любую из этих столовых. Если в одну из столовых случайно придёт больше посетителей, чем в ней имеется мест, то возникает очередь. Какое наименьшее число мест должно быть в каждой из столовых, чтобы вероятность возникновения очереди была меньше 0,15?

% \item В тарифе Связь Z от БиБи-лайн в 1~Гб трафика предоставляется абсолютно бесплатно. Абонент считает что в 75\% случаев ему этого трафика хватит на день. Какова вероятность, что в течение ровно 4~суток в ближайшие 6~дней абонент не понесёт дополнительных расходов на интернет?


% \item Капитан корабля перед высадкой десанта приказал выпустить по береговой полосе длиной 200 метров 20~реактивных снарядов, опасаясь замаскированных огневых точек. Вдоль берега в землю был врыт бункер длиной 20~метров.
% а)~Найти вероятность того, что 4 снаряда попали в бункер. b)~Найти наивероятнейшее число снарядов, попавших в бункер.


% \item Предприятие производит полиэтиленовые 5-литровые бутыли. Компания <<Живая Вода>> покупает их, наполняет и запускает в~торговлю. При покупке бутылок для контроля качества бутылей из партии отбирается случайным образом 10~бутылок. Если среди этих бутылок только две или менее оказываются дефектными, вся партия принимается и направляется в производство. Какова вероятность того, что вся партия будет принята, если предприятие-производитель выпускает 10\% дефектных бутылок?

% \item Тест содержит 10~вопросов с вариантами ответов: да, нет. Какова вероятность получения не менее 80\% правильных ответов, если отвечать наугад?
% \item Если в семье четыре ребёнка, что вероятнее: это два мальчика и две девочки, или три ребёнка одного пола и один другого пола? Принять вероятность того, что данный ребёнок~--- мальчик, равной 0,5.

%\item Вероятность отказа каждого прибора при испытании равна 0,2. Приборы испытывают независимо друг от друга. Что вероятнее: отказ 10 из~80 испытуемых приборов или отказ 15 приборов из 120?

% \item Какое число раз необходимо подбросить игральный кубик, чтобы наивероятнейшее число выпадения шестерки было равно 10?

% \item Два студента вдвоем решают задачи. Вероятность того, что первый не сможет решить задачу, равна 0,2, второй~--- 0,4. Найти наивероятнейшее число нерешенных задач, если всего задач 25.

% \item Начинающий баскетболист  делает четыре независимых броска в кольцо. Вероятность попадания при одном броске равна 0,3. Чтобы войти в состав команды, ему достаточно попасть в кольцо три раза. При двух попаданиях в кольцо вероятность стать членом команды равна 0,8, а при одном попадании~--- 0,2. Найти вероятность того, что указанный баскетболист войдёт в состав команды.

% \item Известно, что в среднем из 100 студентов 60 живут в общежитии. Найти вероятность того, что из 500~студентов живут в общежитии: а)~300 студентов; б) от 280 до 310 студентов.

% \item Вероятность встретить на улице знакомого равна 0,1. Сколько среди первых 100 случайных прохожих можно надеяться встретить знакомых с~вероятностью 0,95?

% \item Завод отправил на базу 10000 стандартных изделий. Среднее число изделий, повреждаемых при транспортировке, составляет 0,02\%. Найти вероятность того, что из 10000 изделий не будет повреждено: а)~9997; б)~хотя бы 9997.

\item По результатам проверок налоговыми инспекциями установлено, что в среднем каждое второе малое предприятие региона имеет нарушение финансовой дисциплины. Найти вероятность того, что из~1000 зарегистрированных в регионе малых предприятий имеют нарушения финансовой дисциплины: а) наивероятнейшее число предприятий; б) от~480 до~520; в) не менее 480 предприятий.


% \item Вероятность сложить пасьянс в каждом из 625 опытов равна 0,8. Найти вероятность того, что относительная частота сложения пасьянса отклонится от его вероятности по абсолютной величине не более, чем на 0,04.

\item Вероятность появления герба в каждом из независимых испытаний равна 0,5. Найти число испытаний n, при котором с вероятностью 0,7698 можно ожидать, что относительная частота появления герба отклонится от его вероятности по абсолютной величине не более чем на 0,02.



% \item Вероятность выиграть в лотерею в каждом билете 0,8. Сколько нужно купить билетов, чтобы с вероятностью 0,9 можно было ожидать, что не менее 75 билетов окажутся выигрышными?

\item В страховой компании 10 тыс. клиентов. Страховой взнос каждого клиента составляет 500 рублей. При наступлении страхового случая, вероятность которого по имеющимся данным и~оценкам экспертов можно считать равной 0,005, страховая компания обязана выплатить клиенту страховую сумму размером 50 тыс. руб. На какую прибыль может рассчитывать страховая компания с~надежностью 0,95?


% \item Вероятность появления на занятиях студента равна 0,2. В семестре всего 385 занятий. 
% Какова вероятность того, что студент будет присутствовать не менее чем на~76 занятиях?
% \item В банк отправлено 40000 пакетов денежных знаков. Вероятность того, что пакет содержит недостаточное или избыточное число денежных знаков, равна 0,0001. Найти вероятность того, что при проверке будет обнаружено: а)~три ошибочно укомплектованных пакета; б)~не более трех пакетов.
% \item Вероятность появления положительного результата в каждом из~$n$ опытов равна~0,9. Сколько нужно произвести опытов, чтобы с вероятностью~0,98 можно было ожидать,  что не менее 150 опытов дадут положительный результат?
% \item Игральную кость бросают 80~раз. Найти с вероятностью 0,99 границы, в которых будет заключено число $m$ выпадений шестерки.
% \item Монета брошена $2n$ раз (n велико). Найти вероятность того, что число выпадений герба будет заключено между числами $n-\frac{\sqrt{2n}}{2}$ и $n+\frac{\sqrt{2n}}{2}$.

\end{enumerate}
\end{document}
