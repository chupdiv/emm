%encoding=utf8%
\documentclass[a4paper,14pt]{extarticle}
% \usepackage{pgfpages}
% \pgfpagesuselayout{2 on 1}[a4paper,border shrink=0mm,landscape]

% Русская кодировка
\usepackage[T2A]{fontenc}
\usepackage[utf8]{inputenc}
\usepackage[russian]{babel}


% необходимые модули
\usepackage{amssymb,amsmath,amsthm}
\usepackage{tabularx,multirow,hhline}
\usepackage{indentfirst}
\usepackage{tikz}
\usepackage[margin=20mm]{geometry}
\usepackage{paralist}
\usepackage[inline]{enumitem}
\usepackage{multicol}
\setlist{noitemsep,leftmargin=\parindent}

\AddEnumerateCounter{\Asbuk}{\@Asbuk}{\CYRM}
\AddEnumerateCounter{\asbuk}{\@asbuk}{\cyrm}
\newlist{dotenumerate}{enumerate}{10}
\newlist{denumerate}{enumerate}{1}
\setlist[enumerate,1]{itemsep=5pt,label=\arabic*., ref=\arabic*} %списки со скобками
\setlist[enumerate,2]{label=\textbf{(\asbuk*)}, ref=\arabic*} %списки со скобками
\setlist[denumerate,1]{noitemsep,label={Д.\,\arabic*.}, ref=\arabic*} %списки со скобками

\usepackage{environ}

\NewEnviron{LOOP}[1]{
  \newcounter{nclone}
  \setcounter{nclone}{#1} 
  \par
  \loop
    \BODY
  \addtocounter{nclone}{-1}
  \ifnum \value{nclone}>0 \repeat}

%Полуторный интервал
\renewcommand{\baselinestretch}{1.00}
\pagestyle{empty}
\newcolumntype{C}{>{\centering\arraybackslash}X}


\sloppy
\binoppenalty=10000
\relpenalty=10000

\usepackage{diagbox}

\newcommand{\bitem}{\textbf{\item}}
\begin{document}
{\centering 
{\centering {\scriptsize Практическое занятие \textnumero~22 \par}
\bfseries Дискретные случайные величины --- 2\par}
\begin{enumerate}
\item 
	Составить закон распределения ДСВ  $X+2Y$, найти ее математическое ожидание и дисперсию (двумя способами), если ДСВ $X$ и $Y$ заданы законами распределения:
$$
\begin{array}{c|ccc}
	x_i & 0 & 1 & 3 \\
	\hline
	p_i & 0,3 & 0,6 & 0,1\\
\end{array}
\qquad
\begin{array}{c|cc}
	y_i & 0 & 2 \\
	\hline
	p_i & 0,2 & 0,8\\
\end{array}
$$

%\item 
%	Найти математическое ожидание случайной величины $Z=X+2Y$, если известны математические ожидания $M(X)=5$, $M(Y)=3$.

\item 
	Случайные величины $X$ и $Y$ независимы. Найти дисперсию случайной величины $Z=3X+2Y$, если известно, что $D(X)=5$, $D(Y)=6$.
\item 
	Вероятность того, что стрелок попадет в мишень при одном выстреле, равна 0,8. Стрелку выдаются патроны до тех пор, пока он не промахнется. Требуется: а) составить закон распределения ДСВ Х – числа патронов, выданных стрелку; б) найти моду $X$.
	
\item Задана дискретная двумерная случайная величина $(X,Y)$: 
$$\begin{array}{c|ccc}
			& \multicolumn{3}{c}{X}\\
	Y  		& x_1=2 & x_2=4 & x_3=8 \\
	\hline
	y_1=4  	& 0,05 	& 0,15 	& 0,20\\
	y_2=8 	& 0,15 	& 0,25 	& 0,20\\
\end{array}
$$
Найти: 
	а) безусловные законы распределения составляющих; 
	б) условный закон распределения составляющей X при условии, что составляющая $Y$ приняла значение $y_2=8$; 
	в) условный закон распределения Y при условии, что $X=x_2=4$.
\item
	Два бомбардировщика поочередно сбрасывают бомбы на цель до первого попадания. Вероятность попадания  в цель первым бомбардировщиком равна 0,6, вторым~--- 0,7. У каждого бомбардировщика имеется по 4 бомбы. Найти функцию распределения ДСВ $Х$~--- число сброшенных бомб.
\item[] \textbf{Домашнее задание}
\item
	Даны две ДСВ: $X$~--- число появлений орла при двух бросках монеты, $Y$~--- число выпавших очков при бросании игральной кости. 
	Составить их законы распределения. Найти $M(X-Y)$, $D(X-Y)$  и $M(XY)$, $D(XY)$.
%\item 
%	Два консервных завода поставляют продукцию в магазин в пропорции $2:3$. Доля продукции высшего качества на первом заводе составляет 90\%, а на втором~--- 80\%. 
%	В магазине куплено 3 банки консервов. Найти функцию распределения СВ $X$~--- число банок с продукцией высшего качества.

\item Задана дискретная двумерная случайная величина $(X,Y)$: 
$$\begin{array}{c|ccc}
			& \multicolumn{3}{c}{X}\\
	Y  		& x_1=2 & x_2=3 & x_3=8 \\
	\hline
	y_1=2  	& 0,10 & 0,20 & 0,40\\
	y_2=5 	& 0,15 & 0,05 & 0,10\\
\end{array}
$$
Найти: 
	а) безусловные законы распределения составляющих; 
	б) условный закон распределения составляющей $X$ при условии, что составляющая $Y$ приняла значение $y_1=2$.	
\end{enumerate}
\end{document}
