%encoding=utf8%
\documentclass[a4paper,14pt]{extarticle}

% Русская кодировка
\usepackage[T2A]{fontenc}
\usepackage[utf8]{inputenc}
\usepackage[russian]{babel}

\usepackage[pdftex]{graphicx}
\usepackage[pdftex,colorlinks,urlcolor=blue, citecolor=magenta]{hyperref}
%        \pdfcompresslevel=9 % сжимать PDF

% необходимые модули
\usepackage{amssymb,amsmath,amsthm}
\usepackage{indentfirst}
\usepackage{tikz}
\usepackage[margin=8mm]{geometry}
\usepackage[inline]{enumitem}
\setlist{noitemsep,leftmargin=\parindent}

\AddEnumerateCounter{\Asbuk}{\@Asbuk}{\CYRM}
\AddEnumerateCounter{\asbuk}{\@asbuk}{\cyrm}
\newlist{dotenumerate}{enumerate}{10}
\setlist[enumerate,1]{label=\textbf{\arabic*.}, ref=\arabic*} %списки со скобками
\setlist[enumerate,2]{label=\textbf{(\asbuk*)}, ref=(\asbuk*)} %списки со скобками

\usepackage{tabularx}
\newcolumntype{C}{>{\centering\arraybackslash}X}


%Полуторный интервал
\renewcommand{\baselinestretch}{1.00}
\pagestyle{empty}
\begin{document}
{\centering \small Практические занятия \textnumero~4--5 \par\bfseries \large Матрицы и определители\par}

\begin{enumerate}
\item 
	Найдите~$3A-4B$ и $AB^T$для матриц $A=\begin{pmatrix} 1& 4& 0 \\-1& 2& -3 \end{pmatrix}$, $B=\begin{pmatrix} 1& 3& 5 \\ 4&-2& -9 \end{pmatrix}$.

\item 
	Три предприятия выпускают четыре вида продукции. Заданы матрицы помесячных выпусков:
	$$
	\begin{pmatrix}
		2 & 3 & 1 & 2 \\
		4 & 2 & 2 & 1 \\
		5 & 4 & 4 & 2
	\end{pmatrix},
	\quad
	\begin{pmatrix}
		1 & 4 & 2 & 2 \\
		3 & 3 & 3 & 2 \\
		4 & 5 & 4 & 3
	\end{pmatrix},
	\quad
	\begin{pmatrix}
		2 & 5 & 3 & 1 \\
		3 & 4 & 3 & 1 \\
		4 & 4 & 4 & 4
	\end{pmatrix}
	$$
Необходимо: 
	\begin{enumerate*}
	\item 
		найти матрицу выпуска продукции за квартал;
	\item 
		найти матрицы $B_1$ и $B_2$ прироста выпуска продукции за каждый месяц и~проанализировать результаты.		
	\end{enumerate*}

\item Найдите 
	\begin{enumerate*}
	\item 
		$\begin{pmatrix}2&1\\1&3\end{pmatrix}^3$;
	\item 
		$\begin{pmatrix}0&0&1\\1&1&2\\2&2&3\\3&3&4\end{pmatrix}
		\begin{pmatrix}-1&-1\\2&2\\1&1\end{pmatrix}
		\begin{pmatrix}4\\1\end{pmatrix}$
	\end{enumerate*}

\item 
	В мастерскую, специализирующуюся на ремонте телефонов, поступают аппараты, из которых 70\% требуют малого ремонта, 20\%~--- среднего, 10\%~--- сложного.
	Статистически установлено, что через год из аппаратов, прошедших малый ремонт, 10\% требуют малого ремонта, 60\%~--- среднего, 30\%~--- сложного; из аппаратов, прошедших средний ремонт,~--- 20\% малого, 50\%~--- среднего, 30\%~--- сложного; из аппаратов, прошедших сложный ремонт,~--- 60\% малого, 40\%~--- среднего. Найти доли из отремонтированных в начале года аппаратов, которые будут требовать ремонта того или иного вида через один, два, три года.

%\item 
%	Найдите все матрицы, перестановочные с $\begin{pmatrix} 1 & 2 \\ 3 &
%		4\end{pmatrix}$.
\item Для матриц 
  $A=\begin{pmatrix}4&3&\\1&3\end{pmatrix}$ и 
  $B=\begin{pmatrix}1&-2&\\-3&1\end{pmatrix}$
  вычислите $|A|\cdot |B|$ и $|A\cdot B|$.
\item Вычислите определители третьего порядка разными способами:\\
\begin{enumerate*}
	\item 
		${\begin{vmatrix}-6&5&0\\0&-6&-2\\3&2&2\end{vmatrix}}$,
	\item 
		$\begin{vmatrix}2  &1  &3  \\5  &8  &2  \\3  &4  &2  \end{vmatrix}$,
	\item 
		$\begin{vmatrix}\sin^{2}\alpha &1&\cos^{2}\alpha \\ \sin^{2}\beta &1&\cos^{2}\beta \\\sin^{2}\gamma &1&\cos^{2}\gamma \end{vmatrix}$.
\end{enumerate*}

\item Решите уравнение $\begin{vmatrix}1  &x  &x  \\x  &1  &x  \\x  &x  &-2  \end{vmatrix}=0$.
\item Решить СЛУ:\\
\begin{enumerate*}
	\item 
	$\left\lbrace\begin{aligned}
	2x-y-z &= 4,\\ 
	3x+4y-2z &= 11,\\
	3x-2y+4z &=11;
      \end{aligned}\right.$
	\item 
 $\left\lbrace\begin{aligned}2x-3y+z-2 &=0,\\x+5y-4z+5 &=0,\\4x+y-3z+4
	&=0; \end{aligned}\right.$ 
	\item 
	$\left\lbrace\begin{aligned}
    -x_{1}+x_{2}+2x_{3} &= -5,\\ 
    6x_{1}-2x_{2}+3x_{3} &= -7,\\
    x_{1}+8x_{2}+5x_{3} &=  2;\\
  \end{aligned}\right.$
  
%	\item 
%	$\left\lbrace\begin{aligned}
%    2x_{1}+2x_{2}-x_{3}+x_{4} &= 4,\\
%    4x_{1}+3x_{2}-x_{3}+2x_{4} &= 6,\\
%    8x_{1}+5x_{2}-3x_{3}+4x_{4} &= 12,\\
%    3x_{1}+3x_{2}-2x_{3}+2x_{4} &= 6.
%  \end{aligned}\right.$
\end{enumerate*}
\item 
	На предприятии имеется четыре технологических способа изготовления изделий~$A$ и~$B$ из некоторого сырья. В таблице указано количество изделий, которое может быть произведено из единицы сырья каждым из технологических способов:

{\centering
\begin{tabularx}{12cm}{c|CCCC}
\hline
 & \multicolumn{4}{c}{Выход из единицы сырья}\\
Изделие  & I & II & III & IV\\
\hline
$A$ & 2 & 1 & 7 & 4\\
$B$ & 6& 12& 2 & 3\\
\hline
\end{tabularx}
\par}

Найти количество сырья, которое следует переработать по каждой технологии, чтобы произвести 574~изделия~$A$ и~328~изделий~$B$ из 94\;ед. сырья.
\end{enumerate}
\newpage
{\centering\bfseries Ответы\par}
\begin{enumerate}[itemsep=5mm]
\item %1 
	 $\begin{pmatrix} -1 & 0 & -20\\ -19 & -2 & 45\end{pmatrix}$,\quad
	$\begin{pmatrix}13 & 12\\ 20 & -27\end{pmatrix}$
\item %2
	$M_1+M_2+M_3
	=\begin{pmatrix}
		5 & 12 & 6 & 5\\
		10 & 9 & 8 & 4\\
		13 & 13 & 12 & 9
	\end{pmatrix}$, 
	\quad
	$B_1
	=\begin{pmatrix}
		-1 & 1 & 1 & 0\\
		-1 & 1 & 1 & 1\\
		-1 & 1 & 0 & 1
	\end{pmatrix}$,
	\quad
	$B_2=
	\begin{pmatrix}
		1 & 1 & 1 & -1\\
		0 & 1 & 0 & -1\\
		0 & -1 & 0 & 1
	\end{pmatrix}$
\item %3
	\begin{enumerate*}
	\item 
		$\begin{pmatrix}15 & 20\\ 20 & 35\end{pmatrix}$,
		\quad~
	\item 
		$\begin{pmatrix}5 & 15 & 25 & 35\end{pmatrix}^T$.

	\end{enumerate*}
\item %4
	$P=\begin{pmatrix}
		0.1 & 0.6 & 0.3\\
		0.2 & 0.5 & 0.3\\
		0.6 & 0.4 & 0
		\end{pmatrix}$,\quad
	$X_0=\begin{pmatrix}
		0.7 & 0.2 & 0.1
		\end{pmatrix}$,
		
$P^2=\begin{pmatrix}
		0.31 & 0.48 & 0.21\\
		0.3 & 0.49 & 0.21\\
		0.14 & 0.56 & 0.3
	\end{pmatrix}$,\quad
$P^3=\begin{pmatrix}
	0.253 & 0.51 & 0.237\\
	0.254 & 0.509 & 0.237\\
	0.306 & 0.484 & 0.21
	\end{pmatrix}$
	
$X_1 = \begin{pmatrix}0.17 & 0.56 & 0.27\end{pmatrix}$\quad
$X_2 = \begin{pmatrix}0.291 & 0.49 & 0.219\end{pmatrix}$\quad
$X_3 = \begin{pmatrix}0.2585 & 0.5072 & 0.2343\end{pmatrix}$
\item %5
$\begin{pmatrix}15 & -9\\ -15 & 24\end{pmatrix}$.
\item %6
	\begin{enumerate*}
	\item 
		18\quad~
	\item 
		0\quad~
	\item 
		0
	\end{enumerate*}
\item %7
$2x^3 - 2 = 0$, $x=1$
\item %8
	\begin{enumerate}
	\item 
		\underline{$(x,y,z)=(3,1,1)$},\quad
		$\Delta=60$, \quad
		$A^{-1}
			=\frac{1}{60}
			\begin{pmatrix}
				12 & 6 & 6\\
				-18 & 11 & 1\\
				-18 & 1 & 11
			\end{pmatrix}$
		
	\item 
		\underline{$(x,y,z)=(1,2,-3)$},\quad
		$\Delta=107$, \quad
		$A^{-1}
			=\frac{1}{-2}
			\begin{pmatrix}
				-11 & -8 & 7\\
				-13 & -10 & 9\\
				-19 & -14 & 13
			\end{pmatrix}$
	\item 
		\underline{$(x,y,z)=(1,2,-3)$},\quad
		$\Delta=60$, \quad
		$A^{-1}
			=\frac{1}{60}
			\begin{pmatrix}
				-34 & 11 & 7\\
				-27 & -7 & 15\\
				50 & 9 & -4
			\end{pmatrix}
			$
\end{enumerate}
\item %9
\underline{$
(x_1,x_2,x_3,x_4) = (-\frac{12}{13} a, 14 + \frac{7}{26}a, 80 - \frac{9}{26} a, a)
$}

$
	\left\lbrace
	\begin{aligned}
		x_1 + x_2 + x_3 + x_4 &= 94,\\
		2x_1 + x_2 + 7x_3 + 4x_4 &= 574,\\
		6x_1 +12x_2 +2x_3 + 3x_4 &= 328.\\
	\end{aligned}
	\right.
$


\end{enumerate}
\end{document}
