%encoding=utf8%
\documentclass[a5paper,10pt]{extarticle}

% Русская кодировка
\usepackage[T2A]{fontenc}
\usepackage[utf8]{inputenc}
\usepackage[russian]{babel}

\usepackage[pdftex]{graphicx}
\usepackage[pdftex,colorlinks,urlcolor=black, linkcolor=black, citecolor=black]{hyperref}
%        \pdfcompresslevel=9 % сжимать PDF

% необходимые модули
\usepackage{amssymb,amsmath,amsthm}
\usepackage{tabularx}
\usepackage{indentfirst}
\usepackage{tikz}
\usepackage[margin=9mm]{geometry}
\usepackage[inline]{enumitem}
\usepackage{multicol}
\setlist{itemsep=1pt,leftmargin=\parindent}

\AddEnumerateCounter{\Asbuk}{\@Asbuk}{\CYRM}
\AddEnumerateCounter{\asbuk}{\@asbuk}{\cyrm}
\newlist{dotenumerate}{enumerate}{10}
\setlist[enumerate,1]{label=\arabic*., ref=\arabic*} %списки со скобками
\setlist[enumerate,2]{label=\textbf{\asbuk*)}, ref=(\asbuk*)} %списки со скобками

\usepackage{environ}

\NewEnviron{LOOP}[1]{
  \newcounter{nclone}
  \setcounter{nclone}{#1} 
  \par
  \loop
    \BODY
  \addtocounter{nclone}{-1}
  \ifnum \value{nclone}>0 \repeat}

%Полуторный интервал
\renewcommand{\baselinestretch}{1.00}
\pagestyle{empty}
\newcolumntype{C}{>{\centering\arraybackslash}X}

%\usepackage{ccfonts,eulervm}
%\usepackage[math]{iwona}
%\renewcommand{\sfdefault}{iwona}\normalfont
%\renewcommand{\rmdefault}{pplx}\normalfont

\sloppy
\binoppenalty=10000
\relpenalty=10000


\newcommand{\isShare}{\url{https://cloud.mail.ru/public/5tTJ/4DyNiAiTB}}
\newcommand{\HEAD}[2]{
	{\noindent\footnotesize Занятие \textnumero~#1  \hfill  Материалы курса: \isShare\par}
	\hrule
	\vspace{3mm}
	{
		\centering 
		\bfseries \large #2\par
		\par
	}
}
\begin{document}
{\centering \small Практическое занятие \par\bfseries \large Линейные модели. Разные задачи\par}

\begin{enumerate}

 \item 
Для изготовления четырех видов продукции используют три вида сырья.
Запасы сырья, нормы его расхода и цены реализации единицы каждого вида продукции
приведены в таблице.

{\centering
\begin{tabular}{cccccc}
\hline
Тип сырья & \multicolumn{4}{p{4.5cm}}{\centeringНормы расхода сырья на одно изделие} & Запасы сырья\\
	& A & Б	& В	& Г & 	\\
\hline
I 	& 1 & 2	& 1	& 0	& 18\\
II 	& 1 & 1	& 2	& 1 & 30\\
III & 1	& 3	& 3 & 2	& 40 \\
\hline
Цена изделия & 12 & 7 & 18 & 10 & \\
\hline
\end{tabular}
\par}
Требуется:
\begin{enumerate}
\item 
Сформулировать прямую оптимизационную задачу на максимум выручки от
реализации готовой продукции, получить оптимальный план выпуска продукции.
\item 
	Сформулировать двойственную задачу и найти ее оптимальный план.
\item 
	Пояснить нулевые значения переменных в оптимальном плане.
\item 
	На основе свойств двойственных оценок и теорем двойственности проанализировать использование ресурсов в оптимальном плане исходной задачи;
\item 
	определить, как изменятся выручка и план выпуска продукции при увеличении запасов сырья I и II видов на 4 и 3 единицы соответственно и уменьшении на 3~единицы сырья III вида.
%	\item 
%		оценить целесообразность включения в план изделия Д ценой 10 ед., на изготовление которого расходуется по две единицы каждого вида сырья.
\end{enumerate}

\hrule
	\item Решить СЛУ:
	$\left\lbrace\begin{aligned}
     x_1 &= 2,\\
     x_1+x_2+3x_3 &= 2,\\
     x_1+3x_2+5x_3 &= 6.
    \end{aligned}\right.$
		%ответ (2,3,-1).

\item 
	Найдите все матрицы~$X$ размером $2\times 2$, такие, что 
	$X^2=E$,\quad 
	$E=\begin{pmatrix}
		1 & 0\\
		0 & 1
	\end{pmatrix}.
	$	
	\item Вычислить определитель: 
		$\begin{vmatrix}
				1  & 1  & 2  & 0  \\
				3  & 0  & 4  & 1  \\
			 -2  & 2  & 5  & 6  \\
				4  & 0  & 1  & 0
			\end{vmatrix}$.


\item Преобразуйте систему ограничений к общей форме (к неравенствам): 
$
\left\lbrace
\begin{aligned}
  x_1-x_2+4x_3-2x_4&=2,\\
  3x_1+2x_2-x_3+4x_4&=3.\\
\end{aligned}
\right.
$

  \item Найти оптимальный план и оптимальное значение целевой функции
		$F(x_{1},x_{2})=2x_{1}-x_{2} \to \max$, $x_{1},x_{2}\geqslant 0$
		при условии
 $
  \left\lbrace \begin{aligned}
    6x_{1}-10x_{2}	 &\geqslant	2;\\
    3x_{1}+	x_{2}	&\leqslant 16;\\
    2x_{1}	+x_{2} &\geqslant	5.
  \end{aligned} \right.
 $  

%\item После получения долгожданной зарплаты семья собирается поехать на мелкооптовый рынок за мясом. В семье (муж, жена и мать жены) из мяса готовят пельмени, котлеты, голубцы и гуляш. У каждого члена семьи свои соображения о том, на какие блюда лучше использовать мясо. Муж хочет, чтобы на голубцы пошло не менее 1~кг., а на пельмени и котлеты~--- не более 5~кг. 
%Жена считает, что на пельмени и голубцы нужно выделить не менее 4~кг., а на гуляш~--- как минимум в два раза меньше, чем на пельмени.
%Ее мама хочет на котлеты выделить минимум 2~кг., а на голубцы не более 3~кг. Все они согласны в том, что на котлеты и пельмени нужно отвести не меньше половины всего мяса.
%Так как мясо в наше время дорогое, то не хочется покупать лишнего мяса. 
%Сколько его купить, чтобы удовлетворить все пожелания всех членов семьи?  
 
\end{enumerate}




\end{document}



\item 
$
f(x) = 3x_1 + 2x_2 \longrightarrow \max,
$
\quad
$
\left\lbrace\begin{aligned}
x_1 + 2x_2 &\leqslant 11,\\
2x_1 - x_2 &\geqslant 5,\\
x_1 + 3x_2 &\geqslant 14,\\
x_1 , x_2 &\geqslant 0.\\
\end{aligned}\right.
$
\item 
$
 f(x) = 3x_1 + 2x_2\longrightarrow \max,
$
\quad
$
\left\lbrace\begin{aligned}
 x_1 + 2x_2 &\leqslant 12,\\
 2x_1 - x_2 &\geqslant 7,\\
 x_1 + 3x_2 &\geqslant 14,\\
 x_1 , x_2 &\geqslant 0.\\
\end{aligned}\right.
$



\item 
$
f(x) = 3x_1 + 2x_2\longrightarrow \max,
$
\quad
$
\left\lbrace\begin{aligned}
 x_1 + 2x_2 &\geqslant 10,\\
 2x_1 - x_2 &\leqslant 18,\\
 x_1 + 3x_2 &\leqslant 13,\\
 x_1 , x_2 &\geqslant 0.\\
\end{aligned}\right.
$

\item 
$
 f(x) = 3x_1 + 2x_2\longrightarrow \min,
$
\quad
$
\left\lbrace\begin{aligned}
 x_1 + 2x_2 &\geqslant10,\\
 2x_1 - x_2 &\geqslant 10,\\
 x_1 + 3x_2 &\leqslant 13,\\
 x_1 , x_2 &\geqslant 0.\\
\end{aligned}\right.
$

{\footnotesize \item На имеющихся у фермера 400 га земли он планирует посеять кукурузу и сою.
Сев и уборка кукурузы требуют на каждый гектар 200 ден. ед. затрат, а сои~--- 100 ден.~ед. 
На покрытие расходов, связанных с севом и уборкой, фермер получил ссуду в 60 тыс.~ден.~ед. 
Каждый гектар, засеянный кукурузой, принесёт 30 центнеров, а каждый гектар, засеянный соей,~--- 60 центнеров. 
Фермер заключил договор на продажу, по которому каждый центнер кукурузы принесёт ему 3~ден.~ед., а каждый центнер сои~--- 6~ден.~ед. Однако согласно этому договору фермер обязан хранить убранное зерно в течение нескольких месяцев на складе, максимальная вместимость которого равна 21~тыс.~цент\-неров.
Фермеру хотелось бы знать, сколько гектаров нужно засеять каждой из этих
культур, чтобы получить максимальную прибыль.\par}

Вариант 1
$
f(x) = 3x_1 + 2x_2 \longrightarrow \max
\left\lbrace\begin{aligned}
x_1 + 2x_2 &\leqslant 11,\\
2x_1 - x_2 &\geqslant 5,\\
x_1 + 3x_2 &\geqslant 14,\\
x_1 , x_2 &\geqslant 0.\\
\end{aligned}\right.
$
Вариант 2
$
 f(x) = 3x_1 + 2x_2\longrightarrow \max 
\left\lbrace\begin{aligned}
 x_1 + 2x_2 &\leqslant 12,\\
 2x_1 - x_2 &\geqslant 7,\\
 x_1 + 3x_2 &\geqslant 14,\\
 x_1 , x_2 &\geqslant 0.\\
\end{aligned}\right.
$

Вариант 3
$
f(x) = 3x_1 + 2x_2\longrightarrow \max
\left\lbrace\begin{aligned}
 x_1 + 2x_2 &\geqslant 10,\\
 2x_1 - x_2 &\leqslant 18,\\
 x_1 + 3x_2 &\leqslant 13,\\
 x_1 , x_2 &\geqslant 0.\\
\end{aligned}\right.
$
Вариант 4
$
 f(x) = 3x_1 + 2x_2\longrightarrow \min
\left\lbrace\begin{aligned}
 x_1 + 2x_2 &\geqslant10,\\
 2x_1 - x_2 &\geqslant 10,\\
 x_1 + 3x_2 &\leqslant 13,\\
 x_1 , x_2 &\geqslant 0.\\
\end{aligned}\right.
$

Вариант 5
$
f(x) = 4х1+ 3х2 \longrightarrow \max
\left\lbrace\begin{aligned}
 х1 + 2х2 &\leqslant 10
 х1 + 2х2 &\geqslant 2
2х1 + х2 &\leqslant 10
 х1 &\geqslant 0, х2 &\geqslant 0
\end{aligned}\right.
$
Вариант 6
$
 f(x) = 3x_1 + 2x_2\longrightarrow \min
\left\lbrace\begin{aligned}
 x_1 + 2x_2 &\geqslant12
 2x_1 - x_2 &\geqslant 12
 x_1 + 3x_2 &\leqslant 14
 x_1 , x_2 &\geqslant 0
\end{aligned}\right.
$
Вариант 7
$
f(x) = 3х1+ 5х2 \longrightarrow \max
\left\lbrace\begin{aligned}
 х1 + х2 &\leqslant 5
3х1 + 2 х2 &\leqslant 8
х1 &\geqslant 0, х2 &\geqslant 0
\end{aligned}\right.
$

Вариант 8
$
 f(x) = 3x_1 + 2x_2 \longrightarrow \min
\left\lbrace\begin{aligned}
 x_1 + 2x_2 &\leqslant 11
 2x_1 - x_2 &\geqslant 5
 x_1 + 3x_2 &\geqslant 14
 x_1 , x_2 &\geqslant 0
\end{aligned}\right.
$

Вариант 9
$
f(x) = 3x_1+ x_2  \longrightarrow \max
\left\lbrace\begin{aligned}
2х1 + 3х2 &\geqslant 12
-х1 + х2 &\leqslant 2
2х1 - х2 &\leqslant 2
х1 &\geqslant 0, х2 &\geqslant 0
\end{aligned}\right.
$

Вариант 10
$
f(x) = 3x_1+ x_2  \longrightarrow \max
\left\lbrace\begin{aligned}
х1 + х2 &\leqslant 5
0.5х1 + х2 &\geqslant 3
х1 - х2 &\geqslant 1
\end{aligned}\right.
$
