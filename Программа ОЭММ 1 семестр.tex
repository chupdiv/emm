%encoding=utf8%
\documentclass[a4paper,12pt]{extarticle}

% Русская кодировка
\usepackage[T2A]{fontenc}
\usepackage[utf8]{inputenc}
\usepackage[russian]{babel}

\usepackage[pdftex]{graphicx}
\usepackage[pdftex,colorlinks,urlcolor=black, linkcolor=black, citecolor=black]{hyperref}
%        \pdfcompresslevel=9 % сжимать PDF
\usepackage{qrcode}


% необходимые модули
\usepackage{amssymb,amsmath,amsthm}
\usepackage{tabularx}
\usepackage{indentfirst}
\usepackage{tikz}
\usepackage[margin=10mm,bottom=20mm]{geometry}
\usepackage[inline]{enumitem}
\usepackage{multicol}
\usepackage{colortbl}

\setlist{noitemsep,leftmargin=\parindent}

\AddEnumerateCounter{\Asbuk}{\@Asbuk}{\CYRM}
\AddEnumerateCounter{\asbuk}{\@asbuk}{\cyrm}
\newlist{dotenumerate}{enumerate}{10}
\setlist[enumerate,1]{itemsep=4pt,label=\arabic*., ref=\arabic*} %списки со скобками
\setlist[enumerate,2]{label=\textbf{\asbuk*)}, ref=(\asbuk*)} %списки со скобками

\usepackage{environ}

\NewEnviron{LOOP}[1]{
  \newcounter{nclone}
  \setcounter{nclone}{#1} 
  \par
  \loop
    \BODY
  \addtocounter{nclone}{-1}
  \ifnum \value{nclone}>0 \repeat}

%\pagestyle{empty}

	
\DeclareMathOperator{\rang}{rang}
%\usepackage{ccfonts,eulervm}
% \usepackage[math]{iwona}
%\renewcommand{\sfdefault}{iwona}\normalfont
%\renewcommand{\rmdefault}{pplx}\normalfont
\setlength{\multicolsep}{6.0pt plus 2.0pt minus 1.5pt}% 50% of original values
\sloppy
\binoppenalty=10000
\relpenalty=10000
\begin{document}
\begin{center}
	\large
	\bfseries
    Программа первого семестра\\
     дисциплины 
	,,Основы экономико-математического моделирования``
\end{center}


\begin{center}
\bfseries
Темы, выносимые на зачет по математике 
\end{center}


\begin{enumerate}
    \item 
        Понятие модели и экономического моделирования.
    \item 
        Виды моделирования. Понятие математической модели.
    \item 
       Классификация экономико-математических моделей.
    \item 
        Место и роль математического моделирования в экономической науке.
    \item 
        Основные понятия и определения теории графов.
    \item 
        Задача нахождения кратчайшего маршрута.
    \item 
        Дерево решений.
    \item 
        Сеть и сетевая модель.
        \begin{enumerate*}
            \item Критический маршрут
            \item Временные параметры
            \item Коэффициент напряженности работ
            \item Оптимизация <<время -- затраты>>
        \end{enumerate*}
    \item 
        Матрицы и действия над ними.
        \begin{enumerate*}
            \item Виды матриц.  
            \item Линейные операции над матрицами.
            \item Умножение матриц
            \item Обратная матрица
        \end{enumerate*}
    \item 
        Определители квадратных матриц. 
        \begin{enumerate*}
            \item Свойства определителей.
            \item Правило Саррюса
            \item Миноры и алгебраические дополнения
            \item Разложение по строке.
        \end{enumerate*}
  
    \item 
        Системы линейных алгебраических уравнений.
        \begin{enumerate*}
            \item Число решений СЛУ.
            \item Правило Крамера
            \item Элементарные преобразования
            \item Метод Гаусса---Жордана
            \item Базисные решения
        \end{enumerate*}
    \item 
        Системы неравенств
        \begin{enumerate*}
            \item Переход от системы неравенств к системе уравнений.
            \item Графическое представление системы неравенств с двумя переменными.
        \end{enumerate*}

    \item 
        Линейная модель оптимального планирования.
        \begin{enumerate*}
            \item Графический метод решения.
            \item Примеры задач.
            \item Теоремы двойственности
        \end{enumerate*}
    \item 
         Транспортная задача.
    \item 
         Балансовые модели. Отчетная и прогнозная модели Леонтьева
         Модель Леонтьева
    \item 
        Динамическое программирование. 
    \item 
        Многокритериальные задачи. Метод идеальной точки.
    \item 
        Иерархии и приоритеты. 
        \begin{enumerate*}
            \item Построение графа иерархических связей
            \item Матрица попарных сравнений и ее согласованность.
            \item Выбор оптимального решения.
        \end{enumerate*}
    \item 
        События и их виды.
    \item 
        Различные подходы к понятию вероятности.
        \begin{enumerate*}
            \item Статистическая вероятность
            \item Классическая вероятность
            \item Геометирческая вероятность.
        \end{enumerate*}
    \item 
        Теоремы сложения и умножения вероятностей.
    \item 
        Формула полной вероятности.
    \item 
        Формула Байеса.
    \item 
        Повторение независимых испытаний.
        \begin{enumerate*}
            \item Формула Бернулли
            \item Теоремы Муавра---Лапласа
        \end{enumerate*}
    \item 
        Дискретные случайные величины, их числовые характеристики.
    \begin{enumerate*}
        \item Закон распределения в табличной и графической формах.
        \item Мода, математическое ожидание, дисперсия. Их вычисление.
    \end{enumerate*}
    \item 
        Сумма и произведение дискретных случайных величин.
    \item 
        Законы распределения случайных величин.    
        \begin{enumerate*}
            \item биномиальный
            \item геометрический
            \item гипергеометирческий законы
        \end{enumerate*}
    \item 
        Непрерывные случайные величины. Нормальный закон распределения.

\end{enumerate}

\end{document}