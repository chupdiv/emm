%encoding=utf8%
\documentclass[a5paper,14pt]{extarticle}
%\nofiles

% Русская кодировка
\usepackage[T2A]{fontenc}
\usepackage[utf8]{inputenc}
\usepackage[russian]{babel}

\usepackage[pdftex]{graphicx}
\usepackage[pdftex,colorlinks,urlcolor=black, linkcolor=black, citecolor=black]{hyperref}
%        \pdfcompresslevel=9 % сжимать PDF

% необходимые модули
\usepackage{amssymb,amsmath,amsthm}
\usepackage{tabularx}
\usepackage{indentfirst}
\usepackage{tikz}
\usepackage[margin=10mm]{geometry}
\usepackage{enumitem}
\usepackage{multicol}
\setlist{itemsep=4pt,leftmargin=\parindent}

\AddEnumerateCounter{\Asbuk}{\@Asbuk}{\CYRM}
\AddEnumerateCounter{\asbuk}{\@asbuk}{\cyrm}
\newlist{dotenumerate}{enumerate}{10}
\setlist[enumerate,1]{label=\arabic*., ref=\arabic*} %списки со скобками
\setlist[enumerate,2]{label=\asbuk*), ref=(\asbuk*)} %списки со скобками

\usepackage{environ}

\NewEnviron{LOOP}[1]{
  \newcounter{nclone}
  \setcounter{nclone}{#1} 
  \par
  \loop
    \BODY
  \addtocounter{nclone}{-1}
  \ifnum \value{nclone}>0 \repeat}


  \newcounter{ncard}
\newcommand{\num}{\refstepcounter{ncard}\arabic{ncard}}  
%Полуторный интервал
\renewcommand{\baselinestretch}{1.00}
\pagestyle{empty}
\newcolumntype{C}{>{\centering\arraybackslash}X}
\sloppy
\begin{document}
\begin{LOOP}{20}
\begin{enumerate}
{\item[]\centering \bfseries  Контрольная работа <<Линейные модели>>\par\vspace{1mm} Карточка № \num\par\vspace{1mm}}

\item Вычислить: $5A-B^2-E$,\\ если
$A=\begin{pmatrix}0&-3\\4&-1\end{pmatrix}$,
$B=\begin{pmatrix}2&4\\-5& 0\end{pmatrix}$.

\item Решить СЛУ:\\
$
\left\lbrace\begin{aligned}
x_1+2x_2-x_3 & =1,\\
3x_1+4x_2-2x_3 & =1,\\
5x_1+x_3 & =-1.\\
\end{aligned}\right.
$

\item Найти общее решение СЛУ, указать свободные и зависимые переменные:\\
$
\left\lbrace\begin{aligned}
x_1-4x_2+3x_3 +7x_4&=4,\\
-2x_1+x_2-x_3+3x_4 &=6.\\
%4x_1-3x_2+x_3+5x_4 &=2,\\
%-x_1+2x_2-x_3-x_4 &=4.\\
\end{aligned}\right.
$

\item  Найти оптимальный план и оптимальное значение функции  $F(x,y)=2x+3y\to \max$, $x,y\geqslant 0$ при условии
$
\left \lbrace
\begin{aligned}
  -2x+y &\leqslant 2\\
  x-3y &\geqslant -9\\
  4x+3y &\leqslant 24
\end{aligned}
\right.
$

\item Решить симплекс-методом:\\
$F(x_1, x_2, x_3, x_4) = -2x_1 - x_2 + x_3 -2x_4  \to \max$,\\
при условиях 
$
\left\lbrace\begin{aligned}
x_1 + 3x_2 - x_3 + 4x_4 &\leqslant 5\\
-x_1 - x_2 + x_3 + 2x_4 &\leqslant 3
\end{aligned}\right.
$


\item Найти $A^{-1}$, если
$A=\begin{pmatrix}1 & 2 \\ 2 & 3\end{pmatrix}$.

\end{enumerate}
\newpage
\begin{enumerate}
{\item[]\centering \bfseries  Контрольная работа <<Линейные модели>>\par\vspace{1mm} Карточка № \num\par\vspace{1mm}}


\item Вычислить: $2(A-E)-B^2$, если\\
$A=\begin{pmatrix}10&4\\4&-15\end{pmatrix}$,
$B=\begin{pmatrix}0&-3\\-5&6\end{pmatrix}$.

\item Решить СЛУ:
$
\left\lbrace\begin{aligned}
2x_1-5x_2-x_3 & =-1,\\
3x_1+x_3 & =-4,\\
4x_1+x_2+2x_3 & =-5.\\
\end{aligned}\right.
$

\item Найти общее решение СЛУ, указать свободные и зависимые переменные:\\
$
\left\lbrace\begin{aligned}
x_1+2x_2-x_3 +x_4&=-6,\\
%-x_1+3x_3-x_4 &=-4,\\
2x_1+4x_2+x_3+3x_4 &=-5,\\
%3x_1+x_2+2x_4 &=2.\\
\end{aligned}\right.
$

\item Найти оптимальный план и оптимальное значение функции $F(x,y)= 5x-y \to \min$, $x,y\geqslant 0$ в области
$
\left \lbrace
\begin{aligned}
  2x+3y &\leqslant 18,\\
  -5x+9y &\leqslant 45,\\
  x-2y &\leqslant 4
\end{aligned}
\right.
$

%~\hfill\textbf{Ответ: $F_{\min}(0,5)=-5$.}

\item Решить симплекс-методом:\\
$F(x_1, x_2, x_3, x_4) = -x_1 +x_2 + 3x_3 +x_4  \to \min$,\\
при условиях 
$
\left\lbrace\begin{aligned}
-x_1 + x_2 - x_3 + 2x_4 &\leqslant 6\\
2x_1 + x_2 + 2x_3 - x_4 &\leqslant 4
\end{aligned}\right.
$

\item Найти $A^{-1}$, если
$A=\begin{pmatrix}-3 & -1 \\ 1 & 2 \end{pmatrix}$.
\end{enumerate}
\newpage
\begin{enumerate}
{\item[]\centering \bfseries  Контрольная работа <<Линейные модели>>\par\vspace{1mm} Карточка № \num\par\vspace{1mm}}


\item Вычислить $A^2+3B+E$,\\ если
	$A=\begin{pmatrix}3&1\\4&-2\end{pmatrix}$,
	$B=\begin{pmatrix}0&2\\-5&7\end{pmatrix}$.
	
\item Решить СЛУ:
$
\left\lbrace\begin{aligned}
3x_1-2x_2+4x_3 & =-5,\\
x_2-2x_3 & =4,\\
2x_1-x_2+x_3 & =-1.\\
\end{aligned}\right.
$

\item Найти общее решение СЛУ, указать свободные и зависимые переменные:\\
$
\left\lbrace\begin{aligned}
x_1-x_2+3x_3 +2x_4&=5,\\
3x_1+x_2-4x_3-x_4 &=-5,\\
%5x_1+3x_2-x_3+x_4 &=5,\\
%x_2+5x_3-x_4 &=-2.\\
\end{aligned}\right.
$

\item  Найти оптимальный план и оптимальное значение функции $F(x,y)=2x+3y \to \max$, $x,y\geqslant 0$ при условии
$
\left \lbrace
\begin{aligned}
  -6x+y &\leqslant 3,\\
  -5x+9y &\leqslant 45,\\
  x-3y &\leqslant 3.
\end{aligned}
\right.
$

%~\hfill\textbf{Ответ: Функция бесконечно максимизируется.}


\item Решить симплекс-методом:\\
$F(x_1, x_2, x_3, x_4) = 3x_1 - 2x_2 + x_3 +2x_4  \to \min$,\\
при условиях 
$
\left\lbrace\begin{aligned}
x_1 - x_2 + 3x_3 + 2x_4 &\leqslant 6\\
2x_1 + x_2 - x_3 + 4x_4 &\leqslant 4
\end{aligned}\right.
$

\item Найти $A^{-1}$, если
$A=\begin{pmatrix}1 & 2\\ 3 & 5\end{pmatrix}$.

\end{enumerate}
\newpage
\end{LOOP}
\end{document}	
%Как изменится определитель с комплексными элементами, если каждый его элемент заменить сопряженным числом? 






% Модель Леонтьева
\item Дана матрица прямых затрат 
$A=\begin{pmatrix}
  0.3& 0.2\\
  0.4& 0.1\\ 
\end{pmatrix}$.\\
Найти \textbf{а)}~вектор конечного потребления $Y$, если вектор производства 
$X=\begin{pmatrix}
  200\\
  100\\ 
\end{pmatrix}$;
\textbf{б)}~вектор производства~$X$, если вектор конечного потребления
$Y=\begin{pmatrix}
  55\\
  110\\ 
\end{pmatrix}$.

\item Дана матрица прямых затрат 
$A=\begin{pmatrix}
  0.1& 0.4\\
  0.5& 0.2\\ 
\end{pmatrix}$.\\
Найти \textbf{а)}~вектор конечного потребления $Y$, если вектор производства 
$X=\begin{pmatrix}
  140\\
  100\\ 
\end{pmatrix}$;
\textbf{б)}~вектор производства~$X$, если вектор конечного потребления
$Y=\begin{pmatrix}
  52\\
  104\\ 
\end{pmatrix}$.


\item Дана матрица прямых затрат 
$A=\begin{pmatrix}
  0.2& 0.3\\
  0.6& 0.2\\ 
\end{pmatrix}$.\\
Найти \textbf{а)}~вектор конечного потребления $Y$, если вектор производства 
$X=\begin{pmatrix}
  100\\
  120\\ 
\end{pmatrix}$;
\textbf{б)}~вектор производства~$X$, если вектор конечного потребления
$Y=\begin{pmatrix}
  92\\
  138\\ 
\end{pmatrix}$.