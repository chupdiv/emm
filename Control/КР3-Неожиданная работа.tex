%encoding=utf8%
\documentclass[a5paper,12pt]{extarticle}

% Русская кодировка
\usepackage[T2A]{fontenc}
\usepackage[utf8]{inputenc}
\usepackage[russian]{babel}

\usepackage[pdftex]{graphicx}
\usepackage[pdftex,colorlinks,urlcolor=black, linkcolor=black, citecolor=black]{hyperref}
%        \pdfcompresslevel=9 % сжимать PDF

% необходимые модули
\usepackage{amssymb,amsmath,amsthm}
\usepackage{tabularx}
\usepackage{indentfirst}
\usepackage{tikz}
\usepackage[margin=5mm]{geometry}
\usepackage{enumitem}
\usepackage{multicol}
\setlist{noitemsep,leftmargin=\parindent}

\AddEnumerateCounter{\Asbuk}{\@Asbuk}{\CYRM}
\AddEnumerateCounter{\asbuk}{\@asbuk}{\cyrm}
\newlist{dotenumerate}{enumerate}{10}
\setlist[enumerate,1]{label=\arabic*., ref=\arabic*} %списки со скобками
\setlist[enumerate,2]{label=\asbuk*), ref=(\asbuk*)} %списки со скобками

\usepackage{environ}

\NewEnviron{LOOP}[1]{
  \newcounter{nclone}
  \setcounter{nclone}{#1} 
  \par
  \loop
    \BODY
  \addtocounter{nclone}{-1}
  \ifnum \value{nclone}>0 \repeat}

  \newcounter{nvar}
  \newcommand{\showvar}{\refstepcounter{nvar} \arabic{nvar}}
%Полуторный интервал
\renewcommand{\baselinestretch}{1.00}
\pagestyle{empty}
\newcolumntype{C}{>{\centering\arraybackslash}X}
\begin{document}
\begin{LOOP}{11}
\begin{enumerate}
{\item[]\centering \bfseries  Самостоятельная работа\\ <<Детерминированные модели>> Карточка №\showvar\par\vspace{1mm}}

\item 
Построить сетевой график, рассчитать наиболее ранние и наиболее поздние сроки наступления событий, найти критический путь 
\par{\centering
\begin{tabular}{ccc}
\hline
Работа & Опирается на работы & Время исполнения \\
\hline
$b_1$    & ---             & 2 \\
$b_2$    & ---             & 8 \\
$b_3$    & ---             & 5 \\
$b_4$    & $b_1$           & 5 \\
$b_5$    & $b_1$           & 4 \\
$b_6$    & $b_3$           & 3 \\
$b_7$    & $b_2, b_5$      & 3 \\
$b_8$    & $b_2, b_4, b_5$ & 6 \\
$b_9$    & $b_6, b_7$      & 6 \\
$b_{10}$ & $b_6, b_7$      & 9 \\
$b_{11}$ & $b_8, b_9$      & 5 \\
\hline
\end{tabular}\par}


\item Дана матрица прямых затрат межотраслевого баланса модели Леонтьева;
$A=\begin{pmatrix}
  0.3& 0.2\\
  0.4& 0.1\\ 
\end{pmatrix}$.\\
Найдите \textbf{а)}~вектор конечного потребления $Y$, если вектор производства 
$X=\begin{pmatrix}
  200\\
  100\\ 
\end{pmatrix}$;
\textbf{б)}~вектор производства~$X$, если вектор конечного потребления
$Y=\begin{pmatrix}
  55\\
  110\\ 
\end{pmatrix}$.

\item
Решить транспортную задачу, заданную матрицей стоимостей перевозок. В левой колонке указаны мощности поставщиков, в верхней строке - спрос потребителей. \( \begin{array}{c|cccc} & 20 & 25 & 35 & 20 \\ \hline 40 & 2 & 3 & 5 & 4 \\ 40 & 3 & 2 & 4 & 1 \\ 20 & 4 & 3 & 2 & 6 \\ \end{array} \)
\end{enumerate}



\newpage
\begin{enumerate}
{\item[]\centering \bfseries  Самостоятельная работа\\ <<Детерминированные модели>> Карточка №\showvar\par\vspace{1mm}}




\item 
Для подготовки финансового плана на следующий год фирме необходимо получить данные от отделов сбыта, производства, финансов и бухгалтерии. Ниже указаны соответствующие работы и их продолжительность.
	\begin{itemize}
	\item 
    А - Разработка прогноза сбыта - 10 дней
	\item 
    В - Изучение конъюнктуры рынка - 7 дней
	\item 
    С - Подготовка рабочих чертежей изделия и технологии его производства - 5 дней (выполняется после A)
	\item 
    D - Разработка календарных планов производства - 3 дня (выполняется после С)
	\item 
    Е - Оценка себестоимости производства - 2 дня (Выполняется после D)
	\item 
    F - Определение цены изделия - 1 день (Выполняется, если завершены B, E)
	\item 
    G - Разработка финансового плана - 14 (Выполняется, если завершены E, F)
	\end{itemize}
Построить сетевую модель, найти ранние и поздние сроки событий, определить критический путь и показать его на сетевом графике.
\item 
Дана матрица прямых затрат межотраслевого баланса модели Леонтьева;
$A=\begin{pmatrix}
  0.1& 0.4\\
  0.5& 0.2\\ 
\end{pmatrix}$.\\
Найдите \textbf{а)}~вектор конечного потребления $Y$, если вектор производства 
$X=\begin{pmatrix}
  140\\
  100\\ 
\end{pmatrix}$;
\textbf{б)}~вектор производства~$X$, если вектор конечного потребления
$Y=\begin{pmatrix}
  52\\
  104\\ 
\end{pmatrix}$.

\item 
Решить транспортную задачу, заданную матрицей стоимостей перевозок. В левой колонке указаны мощности поставщиков, в верхней строке - спрос потребителей. \( \begin{array}{c|cccc} & 30 & 35 & 35 & 10 \\ \hline 40 & 2 & 3 & 5 & 4 \\ 50 & 3 & 2 & 4 & 1 \\ 20 & 4 & 3 & 2 & 6 \\ \end{array} \)
\end{enumerate}



\newpage
%\begin{enumerate}
%{\item[]\centering \bfseries  Неожиданная самостоятельная работа\\ <<Детерминированные модели>> Карточка №\showvar\par\vspace{1mm}}
%
%
%\item 
%При подключении абонента к телефонной сети выполняются следующие работы:
%	\begin{itemize}
%
%	\item[(1)] 
%	получение от абонента заявки на подключение;
%	\item[(2)] 
%    выяснение возможности подключения на АТС;
%	\item[(3)] 
%    проверка технических возможностей на месте;
%	\item[(4)] 
%    резервирование рабочего канала связи в магистральной сети АТС;
%	\item[(5)] 
%    резервирование рабочего канала связи в местном распределительном устройстве;
%	\item[(6)] 
%    получение нужного оборудования;
%	\item[(7)] 
%    монтаж проводки на месте;
%	\item[(8)] 
%    проверка новой цепи;
%	\item[(9)] 
%    принятие абонента к обслуживанию;
%	\item[(10)] 
%    установка оборудования;
%	\item[(11)] 
%    регулировка работы оборудования.
%    \end{itemize}
%
%Порядок выполнения работ следующий:
%    работы 2 и 3 следуют за работой 1;
%    работы 4, 5 и 6 – за 2 и 3;
%    работа 7 следует за 5 и предшествует 8;
%    8 предшествует работы 9;
%    6 предшествует работы 10;
%    работа 11 следует за 10;
%    9 и 11 являются последними;
%    работа 4 предшествует работе 8.
%
%Построить сетевую модель, найти ранние и поздние сроки событий, определить критический путь и показать его на сетевом графике.
%
%\item Дана матрица прямых затрат межотраслевого баланса модели Леонтьева;
%$A=\begin{pmatrix}
%  0.2& 0.3\\
%  0.6& 0.2\\ 
%\end{pmatrix}$.\\
%Найдите \textbf{а)}~вектор конечного потребления $Y$, если вектор производства 
%$X=\begin{pmatrix}
%  100\\
%  120\\ 
%\end{pmatrix}$;
%\textbf{б)}~вектор производства~$X$, если вектор конечного потребления
%$Y=\begin{pmatrix}
%  92\\
%  138\\ 
%\end{pmatrix}$.
%
%\item  Решить транспортную задачу, заданную матрицей стоимостей перевозок.
%В левой колонке указаны мощности поставщиков, в верхней строке - спрос потребителей.
% \( \begin{array}{c|cccc} & 20 & 25 & 35 & 10 \\ \hline 30 & 2 & 3 & 5 & 4 \\ 40 & 3 & 2 & 4 & 1 \\ 20 & 4 & 3 & 2 & 6 \\ \end{array} \)
%\end{enumerate}
\newpage
\end{LOOP}
\end{document}	
%Как изменится определитель с комплексными элементами, если каждый его элемент заменить сопряженным числом? 
