%encoding=utf8%
\documentclass[a4paper,14pt]{extarticle}
%\usepackage{pgfpages}
%\pgfpagesuselayout{2 on 1}[a4paper,border shrink=5mm,landscape]

% Русская кодировка
\usepackage[T2A]{fontenc}
\usepackage[utf8]{inputenc}
\usepackage[russian]{babel}
%\usepackage[pdftex]{graphicx}
%\usepackage[pdftex,colorlinks,urlcolor=black, linkcolor=black, citecolor=black]{hyperref}
%        \pdfcompresslevel=9 % сжимать PDF

% необходимые модули
\usepackage{amssymb,amsmath,amsthm}
\usepackage{tabularx,multirow,hhline}
\usepackage{indentfirst}
\usepackage{tikz}
\usepackage[margin=10mm]{geometry}
\usepackage[inline]{enumitem}
\usepackage{multicol}
\setlist{nosep,leftmargin=\parindent}

\AddEnumerateCounter{\Asbuk}{\@Asbuk}{\CYRM}
\AddEnumerateCounter{\asbuk}{\@asbuk}{\cyrm}
\newlist{dotenumerate}{enumerate}{10}
\setlist[enumerate,1]{label=\arabic*., ref=\arabic*} %списки со скобками
\setlist[enumerate,2]{label=\asbuk*), ref=(\asbuk*)} %списки со скобками

\usepackage{environ}

\NewEnviron{LOOP}[1]{
  \newcounter{nclone}
  \setcounter{nclone}{#1} 
  \par
  \loop
    \BODY
  \addtocounter{nclone}{-1}
  \ifnum \value{nclone}>0 \repeat}

%Полуторный интервал
\renewcommand{\baselinestretch}{1.00}
\pagestyle{empty}
\newcolumntype{C}{>{\centering\arraybackslash}X}

  \newcounter{ncard}
\newcommand{\num}{\refstepcounter{ncard}\arabic{ncard}}  

\sloppy
\binoppenalty=10000
\relpenalty=10000
\begin{document}

\begin{LOOP}{10}
\begin{enumerate}
{\item[]\centering \bfseries  Контрольная работа <<Теория вероятностей --- 1>> \par\vspace{1mm} Карточка № \num\par\vspace{1mm}}

\item 
В магазине имеются 30 телевизоров, причем 20 из них импортных. Найти вероятность того, что среди 5 проданных в
течение дня телевизоров окажется 3 импортных телевизора.

\item 
Вероятность своевременного выполнения студентом контрольной работы по каждой из трех дисциплин равна соответственно
0,6, 0,5 и 0,8. Найти вероятность своевременного выполнения контрольной работы студентом: а) по трем дисциплинам; б)
хотя бы по одной дисциплине.

\item 
В среднем пятая часть поступающих в продажу автомобилей некомплектны. Найти вероятность того, что среди 500
автомобилей m имеют некомплектность, где: а) $m=120$; б)  $120\leqslant m\leqslant 140$.

\item 
На некоторой фабрике 30\% продукции производится машиной I, 25\% продукции --- машиной II, остальная часть 45\% ---
машиной III. У машины I в брак идет 1\% всей производимой ею продукции, у машины II --- 1,5\%, у машины III --- 2\%. Наугад
выбранная единица продукции оказалась браком. Какова вероятность того, что она произведена машиной I?

\item 
Какова вероятность того, что сумма трех наудачу взятых отрезков, длина каждого из которых не превосходит $l$, будет
больше $l$?
\end{enumerate}

\vspace{2cm}
\begin{enumerate}
{\item[]\centering \bfseries  Контрольная работа <<Теория вероятностей --- 1>> \par\vspace{1mm} Карточка № \num\par\vspace{1mm}}

\item 
Из 20 отделений банка 10 расположены за чертой города. Для обследования случайным образом отобрано 5 отделений. Какова вероятность того, что среди отобранных окажется в черте города 3 отделения?
\item 
В студии телевидения 3 телевизионных камеры. Для каждой камеры вероятность того, что она включена в данный момент,
равна для 1-ой, 2-ой, 3-ей камеры соответственно 0,6, 0,7, 0,4. Какова вероятность того, что в данный момент: а)
включена хотя бы одна камера; б) все камеры отключены.

\item 
Вероятность того, что стодолларовая  купюра фальшивая, равна 0,01. Найти вероятность того, что из 100 купюр: а) 5 фальшивых; б) от 5 до 20 фальшивых.
\item 
Для участия в студенческих спортивных отборочных соревнованиях выделено из первой группы курса 4, из второй --- 6, из
третьей группы --- 5 студентов. Вероятности того, что студент первой, второй, третьей группы попадает в сборную
института, соответственно равны 0,9, 0,7 и 0,8. Наудачу выбранный студент в итоге соревнования попал в сборную
института. Какова вероятность того, что этот студент принадлежал ко 2-ой группе?

\item 
Какова вероятность, не целясь, попасть бесконечно малой пулей в прутья квадратной решетки, если толщина прутьев равна
a, а расстояние между их осями равно $l$ ($l > a$)?

\end{enumerate}

\newpage
\begin{enumerate}
{\item[]\centering \bfseries  Контрольная работа <<Теория вероятностей --- 1>> \par\vspace{1mm} Карточка № \num\par\vspace{1mm}}


\item 
На полке стоят 10 книг, среди которых 3 книги по теории вероятностей. Наудачу берутся 4 книги. Какова вероятность
того, что среди отобранных две книги по теории вероятностей?

\item 
Мастер обслуживает 4 станка, работающих независимо друг от друга. Вероятность того, что первый станок  в течение
смены потребует внимания рабочего, равна 0,3, второй --- 0,6, третий --- 0,4 и четвертый --- 0,25. Найти вероятность того,
что в течение смены потребует внимания мастера: а) хотя бы один станок; б) четыре станка.

\item 
Вероятность того, перфокарта набита оператором неверно, равна 0,1. Найти вероятность того, что из 200 перфокарт
правильно набитых будет: а) 10; б) от 100 до 150.

\item 
При отклонении от нормального режима работы автомата срабатывает сигнализатор первого типа с вероятностью 0,8, для
второго и третьего типа сигнализатора эта вероятность равна соответственно 0,7 и 0,6. Вероятности того, что автомат
снабжен сигнализатором первого, второго или третьего типа, равны соответственно 0,3, 0,5, 0,2. Получен сигнал об
отклонении от нормального режима. Какова вероятность того, что на автомате установлен сигнализатор второго типа?

\item 
Из отрезка $[-1;2]$ наудачу взяты два числа. Какова вероятность того, что их сумма больше 1, а произведение меньше 1?


\end{enumerate}


\vspace{2cm}
\begin{enumerate}
{\item[]\centering \bfseries  Контрольная работа <<Теория вероятностей --- 1>> \par\vspace{1mm} Карточка № \num\par\vspace{1mm}}

\item 
В магазине имеются 10 женских и 6 мужских шуб. Для анализа качества отобрали три шубы случайным образом. Определить
вероятность того, что среди отобранных шуб окажутся 2 женских.

\item 
Три автомашины направлены на перевозку груза. Вероятность исправного состояния первой из них составляет 0,8, второй ---
0,9, третьей --- 0,4. Найти вероятность того, что находятся в исправном состоянии: а) все три машины; б) хотя бы одна
машина.

\item 
Обувной магазин продал 200 пар обуви. Вероятность того, что в магазин будет возвращена бракованная пара равна 0,01.
Найти вероятность того, что из проданных пар обуви будет возвращено: а) 10 пар; б) от 10 до 50 пар.

\item 
Цех изготовляет кинескопы для телевизоров, причем 20\% всех кинескопов предназначены для кинескопов с диагональю 37
см, 50\% --- с диагональю 51 см, 30\% --- с диагональю 75 см. Известно, что 50\% всей продукции кинескопов с диагональю 37
см отправляется на экспорт, для телевизоров с диагональю 51 см и 75 см этот показатель равен соответственно 40\% и
70\%. Наудачу взятый кинескоп отправлен на экспорт. Какова вероятность того, что этот кинескоп имеет диагональ 37 см?

\item 
Из отрезка [0;2] наудачу выбраны два числа x и y. Найдите вероятность того, что эти числа удовлетворяют неравенствам 
$x^2\leqslant 4y\leqslant 4x$.
\end{enumerate}
\newpage
\end{LOOP}
\end{document}

В-5

1. Известно, что 100 изделий некоторой фирмы 5 бракованные. Взяли наугад на проверку три изделия. Какова вероятность
того, что бракованными окажутся два изделия?

2. Вероятности своевременного выполнения задания тремя независимо работающими предприятиями соответственно равны 0,5,
0,6 и 0,7. Найти вероятность того, что своевременно выполнят задание: а) все три предприятия; б) хотя бы одного
предприятие.

3. В среднем 15\% жильцов дома имеют задолженность по квартирной плате. Найти вероятность того, что из 300 жильцов дома
имеют задолженность: а) 80 жильцов; б) от 200 до 240 жильцов.

4. В среднем из каждых 100 клиентов отделения банка 60 обслуживаются первым операционистом, 30 --- вторым, 10 --- третьим.
Вероятность того, что клиент будет обслужен без помощи заведующего отделением, только самим операционистом, составляют
0,9, 0,8, 0,5 соответственно для первого, второго и третьего служащих банка. Клиент обслужен без помощи заведующего
отделением. Какова вероятность того, что клиента обслуживал первый операционист?

5. На отрезок АВ длиной 12 см наугад ставят точку М. Найдите вероятность того, что площадь квадрата, построенного на
отрезке АМ заключена между 36 см2 и 81 см2.

В-6

1. Среди 20 поступающих в ремонт часов 8 нуждаются в общей чистке механизма. Какова вероятность того, что среди взятых
наудачу 6 часов в общей чистке механизма нуждаются 4?

2. Вероятность правильного оформления счета первым клерком равна 0,9, вторым --- 0,8, третьим --- 0,95. Каждый клерк оформил
по одному счету. Какова вероятность того, что среди оформленных счетов: а) ни один не оформлен правильно; б) хотя бы
один оформлен правильно?

3. На станциях отправления поездов находится 1000 автоматов для продажи билетов. Вероятность выхода из строя одного
автомата в течение часа равна 0,004. Какова вероятность того, что в течение часа из строя выйдут: а) 10 автоматов; б)
от 10 до 30 автоматов?

4. В автохозяйстве имеется три вида автоцистерн: две --- первого типа, четыре --- второго типа, три --- третьего типа.
Вероятность технической исправности этих машин составляет, соответственно, 0,9, 0,8, 0,7. Случайно выбранная
автоцистерна выполнила заказ. Какова вероятность того, что эта цистерна относилась ко 2-му типу?

5. На паркет, составленный из правильных треугольников со стороной а, случайно брошена монета радиуса r. Найдите
вероятность того, что монета не заденет границы ни одного из треугольников.

В-7

1. В магазине имеются 20 телевизоров, причем 11 из них отечественного производства. Найти вероятность того, что среди 6
проданных в течение дня телевизоров окажется 3 отечественных телевизоров. 

2. Предприятие обеспечивает регулярный выпуск продукции при безотказной поставке комплектующих от трех смежников.
Вероятность безотказных поставок от 1-го смежника равна 0,7, от 2-го и третьего --- 0,9 и 0,8 соответственно. Найти
вероятность того, что из трех смежников безотказно будут работать: а) все три смежника; б) хотя бы один смежник.



3. Засеяно 4000 семян. Вероятность того, что семя не взойдет, равна 0,0001. Найти вероятность того, что из
посеянных семян не взойдут: а) 3 семечка; б) от 10 до 100 семечек.]{3. Засеяно 4000 семян. Вероятность того, что семя
не взойдет, равна 0,0001. Найти вероятность того, что из посеянных семян не взойдут: а) 3 семечка; б) от 10 до 100
семечек.

4. Из заготовленной для посева пшеницы зерно первого сорта составляет 40\%, второго сорта --- 50\%, третьего сорта --- 10\%.
Вероятность того, что взойдет зерно первого сорта равна 0,8, второго --- 0,5, третьего --- 0,3. Случайно взятое зерно
взошло. Какова вероятность того, что зерно было второго сорта?

5. Стержень длины а наудачу разломан на 3 части. Найдите вероятность того, что длина каждой части окажется больше а/4.

В-8

1. Контролер ОТК, проверив качество сшитых 40 пальто, установил, что 28 из них первого сорта, а остальные - второго.
Найти вероятность того, что среди взятых наудачу из этой партии четырех пальто 2 будет второго сорта.

2. В порт приходят корабли только из трех пунктов отправления: А, В, С. Вероятность прихода лайнера из пункта А равна
0,4, из пункта В --- 0,3, из пункта С --- 0,1. В порт из каждого пункта прибыл один корабль. Какова вероятность того, что:
а) все три прибывших корабля --- лайнеры; б) хотя бы один из прибывших кораблей --- лайнер?

3. В результате проверки качества приготовленных для посева семян  гороха установлено, что в среднем 90\% всхожи. Какова
вероятность того, что из 200 посеянных сеян взойдут: а) 90 семян; б) не менее 150 семян?

4. В мастерской на трех станках изготавливаются однотипные детали. Вероятность безотказной работы первого станка равна
0,8, второго --- 0,7, третьего --- 0,9. Вероятность изготовления небракованной детали на первом станке равна 0,8, на втором
--- 0,7, на третьем 0,9. Случайно выбранная деталь оказалась небракованной. Найти вероятность того, что эта деталь была
изготовлена вторым станком?

5. Расстояние от пункта А до В автобус проходит за 2 минуты, а пешеход --- за 15 минут. Интервал движения автобусов 25
минут. Вы подходите в случайный момент времени к пункту А и отправляетесь в В пешком. Найдите вероятность того, что в
пути вас догонит очередной автобус.

В-9

1. Из 22 досок в кубометре обрезной доски 3 бракованных. Какова вероятность того, что из 5 взятых досок 2 бракованных?

2. Магазин получает продукцию в ящиках с трех оптовых складов. Вероятность того, что в прибывающих ящиках находятся
яблоки, равна 0,3, 0,5, 0,6 соответственно для 1-го, 2-го, 3-го склада. В магазин прибыло по одному ящику с каждого
склада. Какова вероятность того, что яблоки находятся: а) во всех трех прибывших ящиках; б) хотя бы в одном прибывшем
ящике?

3. Вероятность того, что деталь стандартна, равна 0,8. Найти вероятность того, что из 500 проверенных деталей,
стандартных будет: а) 200 деталей; б) по крайней мере 450 деталей.

4. На автозавод поступили двигатели от трех моторных заводов. От первого завода поступило 10 двигателей, от второго --- 6
и от третьего --- 4 двигателя. Вероятности безотказной работы этих двигателей в течение гарантийного срока соответственно
равны 0,9, 0,8, 0,7. Двигатель проработал в течение гарантийного срока безотказно. Какова вероятность того, что этот
двигатель был изготовлен на втором заводе?

5. Точка (a,b) наудачу выбирается из квадрата с вершинами (0,0), (1,0), (1,1), (0,1). Найдите вероятность того, что
корни уравнения  $x^2+\normalsubformula{\text{ax}}+b=0$  окажутся действительными и одного знака.
\end{document}
